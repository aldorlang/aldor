%%%%%%%%%%%%%%%%%%%%%%%%%%%%%%%%%%%%%%%%%%%%%%%%%%%%%%%%%%%%%%%%%%%%%%%%%%%%%%%%
\chapter*{\textit{Preface}}
%\addcontentsline{toc}{chapter}{Preface}

\asharpfull{} is a programming language that attempts to achieve
power through the uniform treatment of all values.
Rather than build a language by adding features, we have tried instead
to build a language by removing restrictions.
While the design of \asharp{} emphasizes generality and composibility,
it also emphasizes efficiency.
Usually these objectives seem to pull in different directions.
%A significant
An achievement of \asharp{}'s implementation is its ability
to attain both simultaneously.

\asharp{} is not at its foundation an object-oriented language.
Instead, object semantics are reconstructed from the primitive treatment
of functions and types as first-class values.  Similarly, aspect-oriented
programming arises as a natural use of general language primitives.
While the initial ro\^le of \asharp{} was to replace the compiler component 
of the computer algebra system \axiom{}, \asharp{} is not a reimplentation 
of the \axiom{} programming language.  Rather, \asharp{} reconstructs 
the essential aspects of the \axiom{} programming language from more 
primitive notions.

\aldor{} has been, over the period 1995-2001, available from the 
Numerical Algorithms Group (NAG) as part of the commercial \axiom{} system.
Over this period, \aldor{}'s users began using it more and more outside 
of this original context, to the point where now most \aldor{} code
is unrelated to \axiom{}.   

It is now appropriate that \aldor{} have its own means of distribution 
for those who wish to use it in a general context, and {\tt Aldor.org} 
has been formed for this purpose.  The Numerical Algorithms Group has 
graciously consented to allow free distribution of \aldor{} this way.

Bringing \asharp{} from a gleam in the mind's eye to a concrete 
compiler has been a substantial task.  Considerable program analysis 
and optimization is required to reduce high-level source programs to 
efficient machine code.  For the current release, the source of the 
compiler is approximately 135,000 lines of code, not including the 
run-time system, base library or other associated software.   

Here we have described \aldor{} 1.0, the first release of \aldor{}
independent of \axiom{}.   Those who have used \aldor{} earlier will
note that the present document has been adapted to refer to a new library,
{\tt libaldor}.   This library has been used as there is now a considerable
body of \aldor{} code based upon it.  Manuel Bronstein and Marc Moreno Maza
are to be thanked for having invested considerable efforts in its development.
Marc Moreno Maza and Yannis Chicha have updated all the examples in this
document to work with {\tt libaldor}.

Numerous individuals have contributed to \aldor{} over its development
at IBM Research, the Numerical Algorithms Group, and elsewhere.   
These contributors should be listed in the acknowledgements section of 
this document.  Particular thanks are due to Martin Dunstan
who served as \aldor{}'s steward at the Numerical Algorithms Group.
It has been a true pleasure to work with such a collegial and insightful
group of people.  

{\it London, Ontario}\hfill{\it SMW}\\
{\it January, 2002}\hfill
