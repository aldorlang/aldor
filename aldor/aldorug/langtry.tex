%\documentstyle[fullpage,a4]{article}
%\begin{document}
%\newcommand{\SrcQuote}[1]{``{\tt #1}''}
%\setlength{\parindent}{0pt}
%\setlength{\parskip}{1em}
%
%\begin{abstract}
% {\em Anything you read here may change, as the syntax is improved or 
%  various semantic difficulties are found and fixed.  However, the 
%  overall structure will remain the same, and hopefully syntax and 
%  related details won't change that much}
%
% Exceptions are domains, exception types are categories.  Exceptions
% are thrown by \SrcQuote{throw} statements, and caught within 'try'
% blocks.  An \SrcQuote{finally} construct is provided to allow a place
% for resource deallocation.
%
% There is an except operator indicating what exceptions may
% be thrown by a function (following C++'s semantics, because they
% don't require too much in the way of work). Exceptions may be
% parameterised, and may pass information back to their handler.
% Since exception types are categories, all the usual inheritance
% rules apply. 
%
% Exceptions are not (currently) resumable, as it adds a bit more
% expense to each throw, and there isn't a reasonable syntax
% for doing so just yet (think type-safety).  
%\end{abstract}
%
%%%%%%%%%%%%%%%%%%%%%%%%%%%%%%%%%%%%%%%%%%%%%%%%%%%%%%%%%%%%%%%%%%%%%%%%%%%%%%
%%%%%%%%%%%%%%%%%%%%%%%%%%%%%%%%%%%%%%%%%%%%%%%%%%%%%%%%%%%%%%%%%%%%%%%%%%%%%%
%%%                                                                        %%%
%%% :: Exceptions
%%%                                                                        %%%
%%%%%%%%%%%%%%%%%%%%%%%%%%%%%%%%%%%%%%%%%%%%%%%%%%%%%%%%%%%%%%%%%%%%%%%%%%%%%%
%%%%%%%%%%%%%%%%%%%%%%%%%%%%%%%%%%%%%%%%%%%%%%%%%%%%%%%%%%%%%%%%%%%%%%%%%%%%%%
\newcommand{\SrcQuote}[1]{``{\tt #1}''}
\head{chapter}{Exceptions}{asugExcept}
\index{exceptions!{\em see \chapref{asugExcept}}}
\head{widesection}{Introduction}{asugExcIntro}

What happens when something goes wrong during a call to a function?
For example, the system detects an error in its arguments (\eg{}~divide
by zero), or that it can't perform a particular operation (\eg{}~opening
a file).

Using normal call and return code, you have two options:
the function can cause the program to halt, \eg{}~by a call to
``error''; or it can return a partial or union type which includes a flag to
say that the operation failed.
The former may be a bit extreme, especially if the program is in some
way interactive.
The latter option then causes every function that calls it to have to
be aware of and check for the failure case, which may or may not be useful.  It
certainly gets long-winded if the failure case has to be propagated
through a deep call-chain.

A more concise way of signalling error conditions is provided by the
exceptions mechanism.  There are four parts to the mechanism:
\begin{enumerate}
\item Throwing (or raising) an exception
\item Catching exceptions that we are interested in
\item Declaring which exceptions may be thrown by an expression
\item Defining exception objects
\end{enumerate}

\head{section}{Throwing Exceptions}{asugExcThrow}
\index{except}
The syntax for raising an error is:

% The original syntax is deprecated and generates a warning if used:
%    \egindent{\tt except {\em exception}}
\egindent{\tt throw {\em exception}}

where {\em exception} is an expression evaluating to an exception object.
When an exception is thrown, execution of the current function halts,
and control is passed to the most recent handler (note that this uses
dynamic scope for determining ``most recent'').  The type of an throw
expression is {\tt Exit} (but see later)~--- \ie{}~no value is created by
the expression, but control does not pass to the expression (compare this
with the return statement, for example).  A throw statement is
therefore a way of causing a function to terminate abnormally.


\head{section}{Catching Exceptions}{asugExcCatch}
\index{try}
\index{catch}
\index{in}
\index{finally}
To catch an exception the syntax is:

% The original syntax is deprecated and generates a warning if used:
%    \egindent{\tt try {\em Expr} but {\em id} in {\em handler} always {\em stmts}}
\egindent{\tt try {\em Expr} catch {\em id} in {\em handler} finally {\em stmts}}

%% I've implemented the above syntax, but I'm beginning to prefer:
%% try expr but {
%%    (e: ZeroDivide) => ...
%%    (e: BadReduction) => ...
%%    (e: FileException) => ...
%% }
%% Hmm....

where {\em Expr} is the expression we are protecting, {\em id} is an
identifier, {\em handler} is an (optional) error handler and {\em
stmts} are some (optional) statements to do any resource freeing on
abnormal exits.

When a handler is executed, the {\em id} is bound to whatever exception
was raised.  The handler block is then evaluated.

Typically it will look something like:

% The original syntax has changed ...
% \begin{verbatim}
% try ... but E in {
%         E has ZeroDivide(Integer) => ...
%         E has BadReduction => ...
%         E has FileException => ...
%         true => except E;
%         never;
% }
% \end{verbatim}
\begin{small}
\begin{verbatim}
   try ... catch E in {
        E has ZeroDivide(Integer) => ...
        E has BadReduction => ...
        E has FileException => ...
        true => throw E;
        never;
   }
\end{verbatim}
\end{small}

Where each item on the right hand side of the {\tt has} denotes an
exception type.  The last line is to ensure that the statement
compiles successfully.

Each branch of the handler block (the parts after {\tt =>}) should
evaluate to the same type as that of the expression protected by the
handler.

This is so that one can do:

% \begin{verbatim}
% n := try divide(a, x) but E in { E has ZeroDivide => 22; ... }
% \end{verbatim}
\begin{small}
\begin{verbatim}
n := try divide(a,x) catch E in {E has ZeroDivide => 22; ... }
\end{verbatim}
\end{small}

which will attempt the division, and if successful assign the result
to {\tt n}, otherwise if a division by zero exception is raised,
then {\tt n} will have the value 22.

After the handler has been executed, the {\em stmts} in the 
{\tt finally} part of the try expression are evaluated.  These are
guaranteed to be evaluated even if the handler itself raises an exception.

The typical reason to use an {\tt finally} block is to deallocate any
resources the function may have allocated, for example:

\begin{small}
\begin{verbatim}
   f := open(file);
   try doWonderfulThings(f, data) finally close!(f);
\end{verbatim}
\end{small}

Note that the '{\tt catch {\em id} in ...}' part is also optional
provided that a {\tt finally} part is supplied.

This will ensure that the file is always closed regardless of what
exceptions are thrown by {\tt doWonderfulThings}.

\head{section}{Specifying Exceptions}{asugExcSpec}
\index{throw}

It is possible to declare what exceptions are thrown by a particular
expression.  This is done using the {\tt throw} keyword as an
infix operator (the two uses are rarely confused).  The operator takes
two arguments, a base type and a (possible empty) comma-separated list
of exception types.  Typically, the {\tt throw} keyword is applied to the
return types of functions to indicate which exceptions they can raise,
for example:

\egindent {\em foo{\tt :(}args{\tt) ->} returns {\tt throw (} x1,x2,x3,...{\tt)}}

This indicates that {\em foo} may only raise exceptions of types {\em x1,x2,x3}
etc.  The programs behaviour is undefined if other exceptions are
raised.  To indicate that a function raises no exceptions, the tuple
should be empty.  If there is no {\tt throw} clause on the return type,
then the compiler assumes that any exception may be raised by the
function.  This does lead to some unsafe code --- for example:

\begin{small}
\begin{verbatim}
    justDie(): Integer == throw ZeroDivide;

    badIdea(): Integer throw () == justDie();
\end{verbatim}
\end{small}

Here {\tt badIdea} indicates that it will not raise an exception,
while {\tt justDie} will always raise an error.  The compiler may
warn the user in this situation.

The compiler will check if any exceptions are explicitly
raised that do not satisfy the functions signature, for example:
\begin{small}
\begin{verbatim}
   foo(): Integer throw ZeroDivide == {
        throw FileError;
   }
\end{verbatim}
\end{small}
will not compile.  This also works within try blocks:
\begin{small}
\begin{verbatim}
   bar(): () throw Ex1 == {
       try zzz() catch E in {
           E has ExA => throw ZeroDivide; --- error
           E has ExB => throw Ex1; --- OK
           true => throw E --- error
           never
       }
   }
\end{verbatim}
\end{small}

The {\tt throw} qualifier on types works just like any other type
constructor, and so you can use it as part of a type as normal:
\begin{small}
\begin{verbatim}
foo(fn: Integer -> Integer throw ()): () == ...
\end{verbatim}
\end{small}

indicates that the argument to {\tt foo} must be a function that never
raises an exception.  Naturally, there are only a few places where it
makes sense to use these types.

NB: We simplified things earlier when we said that the type of an {\tt throw}
statement was {\tt Exit} --- the actual type of \SrcQuote{throw X} is
{\tt Exit throw {\em typeof({\tt X})}}.  Where {\em typeof({\tt
X})} indicates the exact type of the exception expression.  This
indicates that flow of control stops, but the exception {\tt X}
may be raised.

\head{section}{Defining Exceptions}{asugExcDef}
\index{exceptions, definition of}

An exception definition is made up of two parts --- a category
definition and a domain definition.  The category definition provides
a means to specify related exceptions (so that {\tt ZeroDivideException}
may inherit from {\tt ArithmeticException} for example), and the
domain definition provides a mechanism for creating the exception.

For example, 
\begin{small}
\begin{verbatim}
define ZeroDivideException: Category == ArithmeticException with;
ZeroDivide: ZeroDivideException == add;
\end{verbatim}
\end{small}

If {\tt ZeroDivide} is defined this way, then any handler with a clause 
\SrcQuote{E has ArithmeticException} will also catch {\tt ZeroDivide} exceptions.
There is an {\tt is} keyword in the language if you need to do exact
matching, but it shouldn't be needed that often.
\index{is}

This mechanism also allows one to create parameterised exceptions:
\begin{small}
\begin{verbatim}
define ZeroDivideException(R:Ring):Category == ArithmeticException with;
ZeroDivide(R:Ring):ZeroDivide(R)@Category == add;
\end{verbatim}
\end{small} 
and to have values defined within the exception:
\begin{small}
\begin{verbatim}
define FileException: Category == ArithmeticException with {
        name(): String;
}

FileError(vvv: String): FileException == add { 
        name(): String == vvv;
}
\end{verbatim}
\end{small} 

In fact, there is a myriad of ways to insert values into exceptions:
\begin{small}
\begin{verbatim}
foo(): () == {
        ...
        throw (add {name(): String == <bizzare-calculation>})_
		@FileException
        ...
}
\end{verbatim}
\end{small} 

is one, which has the single advantage that {\tt name} will only
be calculated if it is actually used.  Defining a 'lazy' version of 
{\tt FileException} is a much better thing to do in this situation.


Another example that keeps the {\tt add} definition simple, but 
defines rather a lot of objects, is:

\begin{small}
\begin{verbatim}
define FileException: Category == Exception with {
        name(): String;
}

define FileException(vvv: String): Category == 
	FileException@Category with {
        default name(): String == vvv;
}

FileError(vvv: String): FileException(vvv) == add;

\end{verbatim}
\end{small}


