%%%%%%%%%%%%%%%%%%%%%%%%%%%%%%%%%%%%%%%%%%%%%%%%%%%%%%%%%%%%%%%%%%%%%%%%%%%%%%
%%%%%%%%%%%%%%%%%%%%%%%%%%%%%%%%%%%%%%%%%%%%%%%%%%%%%%%%%%%%%%%%%%%%%%%%%%%%%%
%%%                                                                        %%%
%%% :: Language-Defined Types
%%%                                                                        %%%
%%%%%%%%%%%%%%%%%%%%%%%%%%%%%%%%%%%%%%%%%%%%%%%%%%%%%%%%%%%%%%%%%%%%%%%%%%%%%%
%%%%%%%%%%%%%%%%%%%%%%%%%%%%%%%%%%%%%%%%%%%%%%%%%%%%%%%%%%%%%%%%%%%%%%%%%%%%%%
\head{chapter}{Language-defined types}{asugLangTypes}

%This section describes the types which are defined by the \asharp{} language.
The \asharp{} language defines only those types which
are required in specifying what the language does.  
Most of the types which are usually found in high-level programming languages
are delegated to libraries in \asharp{}.  This allows the library designer
maximum flexibility in dressing the basic types with desired operations.

The language defined types are listed below.
Here, $n, m \geq 0$.
\begin{itemize}
\item Type
\item ($S_1$,...,$S_n$) {\tt ->} ($T_1$,...,$T_m$)
\item Tuple $T$
\item Cross($T_1$,...,$T_n$)
\item Enumeration($x_1$,...,$x_n$)
\item Record($T_1$,...,$T_n$)
\item TrailingArray(($U_1$,...,$U_n$),($V_1$,...,$V_m$))
\item Union($T_1$,...,$T_n$)
\item Category
\item Join($C_1$,...,$C_n$)
%\item Meet($C_1$,...,$C_n$)
\item Boolean
\item Literal
\item Generator $T$
\item Exit
\item Foreign $I$
\item Machine
\item Ref $T$
\item ``Magic Types''
\end{itemize}
These types are described in the sections which follow.

%%%%%%%%%%%%%%%%%%%%%%%%%%%%%%%%%%%%%%%%%%%%%%%%%%%%%%%%%%%%%%%%%%%%%%%%%%%%%%
\head{section}{Type}{asugLangTypesType}%
\Export{Type}{Type}{}%
\index{Type}%
\ttin{Type} is the type of all data type objects, including itself.
Sometimes it it not possible to tell whether a value is a type.
Unless a value has been explicitly asserted to be a type, then
it is not treated as one.

In the example below, the parameter \ttin{t} of the function
\ttin{higher} is a type in some possible calls but not in others.

\begin{small}
\begin{verbatim}
#include "aldor"

higher(T: Type, f: T->T, t: T): T == f f t;

-- next next 2
n: Integer   == higher(Integer, next, 2);

-- List List Integer
L: PrimitiveType == higher(PrimitiveType, List, Integer);
\end{verbatim}
\end{small}

%%%%%%%%%%%%%%%%%%%%%%%%%%%%%%%%%%%%%%%%%%%%%%%%%%%%%%%%%%%%%%%%%%%%%%%%%%%%%%
\head{widesection}{{\tt ($S_1$,..,$S_n$)->($T_1$,..,$T_m$)}}{asugLangTypesFunction}%
\Export{->}{(Tuple Type, Tuple Type) -> Type}{}%
\index{-@{\tt ->}}%
\index{function types}%
``{\tt ($S_1$,..,$S_n$)->($T_1$,..,$T_m$)}''
is the type for functions which have $n$ arguments and produce $m$
results of the given types.  
The types $S_i$ may be mutually dependent and may have default values.
The types $T_j$ may depend on $S_i$ as well as on each other.

These are a few examples of function types:

\begin{small}
%\begin{tabular}{@{}p{1.5in}p{3.0in}@{}}
\begin{tabular}{ll}
{\tt I ==> Integer}
   & A macro used in these examples. \\
   &    \\
{\tt (I, I) -> ()} 
   & No result --- useful for side-effecting functions. \\
{\tt (I, I) -> I} 
   & One result --- most common case. \\
{\tt (I, I) -> (I, I)}
   & Two results. \\
{\tt Tuple I -> Tuple I}
   & Any number of arguments and any number of \\
   & results.  \\
{\tt (i:~I, n:~I) -> I}
   & The arguments may be passed by keyword. \\
{\tt (i:~I, n:~I == 0) -> I}
   & The arguments may be passed by keyword, and \\
   &  the second one has a default value. \\
{\tt (I, n:~I)-> IntegerMod n}
   & The return type\footnote{{\ttin IntegerMod} is defined on page \pageref{imodnSample}} depends on an argument value. \\
{\tt (n:~I, IntegerMod n)-> I}
   & One argument type depends on another argument\\
   & value.
\end{tabular}
\end{small}

%\pagebreak
%%%%%%%%%%%%%%%%%%%%%%%%%%%%%%%%%%%%%%%%%%%%%%%%%%%%%%%%%%%%%%%%%%%%%%%%%%%%%%
\head{section}{Tuple $T$}{asugLangTypesTuple}%
\Export{Tuple}{Type -> Type}{}%
\index{Tuple}%
Tuples provide n-ary, homogeneous products.  
The language allows explicit multiple values of the same type or
a \ttin{Cross} of values of the same type to be converted to a Tuple.
For example,
\verb"()", \verb"(1)", \verb"(1, 2)", and \verb"(1,3,7,8)" may all be
used where a \verb"Tuple(Integer)" is expected.
The base \asharp{} library extends Tuple
to provide operations to count the elements or extract particular ones:

\Export{length}{Tuple S -> MachineInteger}{}%
\ExportToo{element}{(Tuple S, MachineInteger) -> S}{}%
These operations are named differently to the standard
\ttin{\#} and \ttin{apply} to avoid ambiguity between 
values and singleton tuples.
The \ttin{element} operation uses 1-based indexing.
Tuple values are not updatable.

%%%%%%%%%%%%%%%%%%%%%%%%%%%%%%%%%%%%%%%%%%%%%%%%%%%%%%%%%%%%%%%%%%%%%%%%%%%%%%
\head{widesection}{Cross($T_1$,...,$T_n$)}{asugLangTypesCross}%
\index{Cross}
\Export{Cross}{Tuple Type -> Type}{}
This is the constructor for cross-product types.
The language allows explicit multiple values to be converted
to a single cross product value and {\em vice versa}.
Here is an example:

\begin{small}%
\begin{verbatim}

-- Conversion of multiple values to a cross product:
ij: Cross(Integer, Integer) := (1, 2);

-- Conversion of a cross product to multiple values:
(i, j) := ij;

-- This gives the same result as  n := i + j.
n := + ij;
\end{verbatim}
\end{small}

There are no operations for counting or selecting product components,
and product values are not updatable.
The products may be cartesian or dependent:

\begin{small}
\begin{verbatim}

Cross(Integer, Integer)          -- cartesian product
Cross(n: Integer, IntegerMod(n)) -- dependent product
\end{verbatim}
\end{small}


%%%%%%%%%%%%%%%%%%%%%%%%%%%%%%%%%%%%%%%%%%%%%%%%%%%%%%%%%%%%%%%%%%%%%%%%%%%%%%
\head{widesection}{Enumeration($x_1$,...,$x_n$)}{asugLangTypesEnumeration}%
\index{Enumeration}
\Export{Enumeration}{Tuple Type -> Type}{}%
Enumerations are types consisting of a fixed set of symbolic values.

A list of names enclosed in single quotes
is a short-hand for a call to \verb+Enumeration+.
For example,

\begin{small}
\begin{verbatim}
Colour == 'red, green, blue';

x: Colour := red
\end{verbatim}
\end{small}

One of the common uses of enumerations is for selector functions.
The \verb+List(S)+ domain from the standard \asharp{} library exports
functions called \verb+first+, \verb+rest+, \verb+setFirst!+ and \verb+setRest!+.
These operations could be replaced by the following ones:

\begin{small}%
\begin{verbatim}
apply: (%, 'first') -> S
set!:  (%, 'first', S) -> S
apply: (%, 'rest') -> %
set!:  (%, 'rest', %) -> %
\end{verbatim}
\end{small}%

This allows expressions of the form:

\begin{small}%
\begin{verbatim}
l.first; l.first := s
l.rest;  l.rest  := l
\end{verbatim}
\end{small}%

Separate types \verb"'first'" and \verb"'rest'" are used,
rather than one \\ \verb"'first, rest'", to allow strong type checking.

The precise way in which enumerations work may seem a bit strange at first:
the form 

\begin{small}
\begin{verbatim}
'red, green, blue'
\end{verbatim}
\end{small}%

is actually a short-hand for the call 

\begin{small}
\begin{verbatim}
Enumeration(red: Type, green: Type, blue: Type)
\end{verbatim}
\end{small}%

Notice that this has the same form as a typical call to \verb+Record+
or a call to \ttin{->} with keyword arguments.
This provides enumerations without introducing 
any extra fundamental ideas into the language.

%%%%%%%%%%%%%%%%%%%%%%%%%%%%%%%%%%%%%%%%%%%%%%%%%%%%%%%%%%%%%%%%%%%%%%%%%%%%%%
\head{widesection}{Record($T_1$,...,$T_n$)}{asugLangTypesRecord}%
\Export{Record}{Tuple Type -> Type}{}%
\index{Record}%
Records provide the basic updatable structure for aggregate data. Each type argument to \verb"Record" may be given in any of the following forms:

$ T $ \, or \,
$ id : T $ \, or \,
$ id : T $ \verb"=="$\; v $.

%\pagebreak
A record type 
{\tt Record($T_1$,...,$T_n$)}
exports the following operations:

\Export{bracket}{($T_1$,...,$T_n$) -> \%}{}
\ExportToo{record}{($T_1$,...,$T_n$) -> \%}{}
\ExportToo{explode}{\% -> ($T_1$,...,$T_n$)}{}
\ExportToo{dispose!}{\% -> ()}{}

and, for each argument of the form 
$ id : T $ or
$ id : T $ \verb"=="$\; v $,
the record type also exports the operations:

\Export{apply}{(\%, '$id_i$') -> $T_i$}{}
\ExportToo{set!}{(\%, '$id_i$', $T_i$) -> $T_i$}{}

The \ttin{bracket} and \ttin{record} operations have the same function
and construct new record values.  

The \ttin{bracket} operation allows
records to be constructed with the syntax \verb"[1,1]", which is nice
and concise.
More importantly, it allows record types to be used generically --- that
is, without disclosing that the heterogeneous aggregate
is a record type, as opposed to a table or other structure.

The  \ttin{record} operation allows documented construction
of record values.  
This is convenient if there are many aggregate types in scope
(lists, lists of records, records of lists \etc)
and the bracket operation becomes too heavily overloaded for clarity.

The \ttin{explode} operation allows record values to be deconstructed
into their constituent parts.  This is convenient for use with
multiple assignment or passing all the components to a function of
an equal number of arguments.

The \ttin{dispose!} operation promises that its argument will
no longer be referenced,
and on certain platforms permits the memory to be reused immediately.
It is not necessary to use this function:
if you don't, storage will be garbage collected periodically.
The choice often boils down to debugability {\em versus} speed.

The \ttin{apply} and \ttin{set!} operations allow the record
fields to be extracted and reset.  This may be done with either
explicit calls to these functions or implicit calls arising
from the forms \ttin{r.i} or \ttin{r.i := j}.

The substitution of the type arguments into the exported operations
uses the original form in which the argument is given.
As a consequence, record constructors support keyword arguments
and arguments with default values.

%\pagebreak
To be concrete, we give a few examples.

\subsubsection{{\tt Record(I, DF)}}

The simplest (and least useful) way to use \verb"Record"
is to call it with simple type expressions, giving neither
field names nor default values with the arguments.

An example would be \verb"Record(Integer, DoubleFloat)".
In this case no \ttin{apply} or \ttin{set!} operations are exported,
and the behaviour of
the record type is very similar to that of a corresponding call
to \verb"Cross".

\Export{bracket}{(Integer, DoubleFloat) -> \%}{}
\ExportToo{record}{(Integer, DoubleFloat) -> \%}{}
\ExportToo{explode}{\% -> (Integer, DoubleFloat)}{}
\ExportToo{dispose!}{\% -> ()}{}


\subsubsection{{\tt Record(i:I, x:DF)}}

The call 
\verb"Record(i: Integer, x: DoubleFloat)"
provides a record type with the following operations:

\Export{bracket}{(i: Integer, x: DoubleFloat) -> \%}{}
\ExportToo{record}{(i: Integer, x: DoubleFloat) -> \%}{}
\ExportToo{explode}{\% -> (Integer, DoubleFloat)}{}
\ExportToo{dispose!}{\% -> ()}{}
\ExportToo{apply}{(\%, 'i') -> Integer}{}
\ExportToo{apply}{(\%, 'x') -> DoubleFloat}{}
\ExportToo{set!}{(\%, 'i', Integer) -> Integer}{}
\ExportToo{set!}{(\%, 'x', DoubleFloat) -> DoubleFloat}{}

This allows expressions of the form

\begin{small}%
\begin{verbatim}
r := [3, 4.0];
r := [i == 3, x == 4.0]
r := [x == 4.0, i == 3]

(vi, vx) := explode r

r.i := 7
r.x := 32.0
\end{verbatim}
\end{small}

To be painstakingly correct, the exports are not precisely as shown above.
The actual exports are:

\Export{bracket}{(i: Integer, x: DoubleFloat) -> \%}{}
\ExportToo{record}{(i: Integer, x: DoubleFloat) -> \%}{}
\ExportToo{explode}{\% -> (i: Integer, x: DoubleFloat)}{}
\ExportToo{dispose!}{\% -> ()}{}
\ExportToo{apply}{(\%, 'i') -> (i: Integer)}{}
\ExportToo{apply}{(\%, 'x') -> (x: DoubleFloat)}{}
\ExportToo{set!}{(\%, 'i', i: Integer)) -> (i: Integer)}{}
\ExportToo{set!}{(\%, 'x', x: DoubleFloat) -> (x: DoubleFloat)}{}

What is happening is that the exports from the record type
all support keyword arguments as  a consequence of uniform substitution.
This really only useful for the \ttin{bracket} and \ttin{record}
operations, but it is cleaner to be completely uniform.

\subsubsection{{\tt Record(i:I==7, x:DF==0)}}

If argument types to \verb"Record" are given with default values,
for example
\verb"Record(i: Integer == 7, x: DoubleFloat == 0)",
then the uniform substitution yields the following construction operations:

\Export{bracket}{(i: Integer == 7, x: DoubleFloat == 0) -> \%}{}
\ExportToo{record}{(i: Integer == 7, x: DoubleFloat == 0) -> \%}{}

Now it is no longer necessary to supply all the fields when constructing
new values:

\begin{small}%
\begin{verbatim}
r := [5];      -- Same as  r := [5, 0]
r := [.01];    -- Same as  r := [7, .01]
r := [];       -- Same as  r := [7, 0]
\end{verbatim}
\end{small}

%%%%%%%%%%%%%%%%%%%%%%%%%%%%%%%%%%%%%%%%%%%%%%%%%%%%%%%%%%%%%%%%%%%%%%%%%%%%%%
\head{widesection}{TrailingArray(($U_1$,...,$U_n$),($V_1$,...,$V_m$))}{asugLangTypesTrailingArray}%
\Export{TrailingArray}{(Tuple Type,Tuple Type) -> Type}{}%
\index{TrailingArray}%

Trailing arrays are an aggregate data type consisting of two parts; a header,
and an array of objects immediately following.  The representation for this is a
single block of memory,  the array immediately following the header.  It
is up to the user to ensure that accesses to the trailing part of the structure
are correct.

For example, a polynomial could be represented as:

\verb+TrailingArray( sz: Integer , (coef: R, deg: NNI))+

This would create a data structure looking like:

\begin{tabular}{|c|c|c|c|c|c|c|c}\hline
sz & coef & deg & coef & deg & coef & deg & \dots\\ \hline
\end{tabular}

The advantage of this representation is that the object is created by a single
allocation, and that there is no overhead for data pointers (as there would be
in a representation using a list of records.  For example, the domain

\verb+Record(sz: Integer, tail: PrimArray Record(r: R, deg: NNI))+

contains exactly the same information, but looks like:

\begin{tabular}{cccccccccccccccccc}
sz & tail \\ 
   & $\downarrow$  & \\ 
& 0 & & 1 & & 2 & & $\ldots$ \\ 
& $\downarrow$ & & $\downarrow$ & & $\downarrow$  \\
& coef & deg &coef &deg &coef &deg & $\ldots$ \\ 
\end{tabular}

%-------------
%| sz | tail |
%-------------
%            |
%     -------------------------------------------------------
%     |      0      |        1       |       2      |  ...
%     -------------------------------------------------------
%            |               |               |
%     --------------  --------------  --------------
%     | coef | deg |  | coef | deg |  | coef | deg |   ...
%     --------------  --------------  --------------


\subsubsection{Usage}


\verb+TrailingArray( U, V )+

U and V are tuples of types, and so take the form `($T_1$, $T_2$, ...)', or simply
`$T_1$' if there is only a single type in the tuple.  Assume that U is `($u_1$: $U_1$, $u_2$: $U_2$,
$\ldots$)'

The domain exports the following functions:

The following are equivalent to the exports of \verb+Record(U)+ :

\Export{apply}{(\%, '$u_n$') -> $U_n$}{}
\ExportToo{set!}{(\%, '$u_n$', $U_n$) -> $U_n$}{}

Trailing part access:

\Export{apply}{(\%, MachineInteger, '$v_n$') -> $V_n$}{}
\ExportToo{set!}{(\%, MachineInteger, '$v_n$', $V_n$) -> $V_n$}{}

Allocation and disposal:

\Export{bracket}{(MachineInteger, Cross U, Cross V) -> \%}{}
\ExportToo{trailing}{(MachineInteger, Cross U, Cross V) -> \%}{}
\ExportToo{dispose!}{\% -> ()}{}

The allocation functions take 3 arguments: a size, and an initial value for both
the header and trailing parts.  The trailing part argument is ignored at the
moment.  These have to be filled in by the user, and the initial value of the
trailing parts is undefined.

Currently the datatype does not support dependent types at all.


%%%%%%%%%%%%%%%%%%%%%%%%%%%%%%%%%%%%%%%%%%%%%%%%%%%%%%%%%%%%%%%%%%%%%%%%%%%%%%
\head{widesection}{Union($T_1$,...,$T_n$)}{asugLangTypesUnion}%
\Export{Union}{Tuple Type -> Type}{}%
\index{Union}%
The \verb+Union+ constructor provides types which can be used
to represent values belonging to any one of several alternative types.

If a function were to return either an integer or a 
floating point number, then a type such as
\verb"Union(int: Integer, flo: DoubleFloat)"
could be used.
This type would then provide the following operations:

\Export{bracket}{(int: Integer) -> \%}{}
\ExportToo{bracket}{(flo: DoubleFloat)   -> \%}{}
\ExportToo{union}{(int: Integer) -> \%}{}
\ExportToo{union}{(flo: DoubleFloat)   -> \%}{}
\ExportToo{case}{(\%, 'int') -> Boolean}{}
\ExportToo{case}{(\%, 'flo') -> Boolean}{}
\ExportToo{apply}{(\%, 'int') -> Integer}{}
\ExportToo{apply}{(\%, 'flo') -> DoubleFloat}{}
\ExportToo{set!}{(\%, 'int', Integer) -> Integer}{}
\ExportToo{set!}{(\%, 'flo', DoubleFloat)   -> DoubleFloat}{}
\ExportToo{dispose!}{\% -> ()}{}

That is, for each argument, $ id : T $, the union type exports
the following operations:

\Export{bracket}{(id: T) -> \%}{}
\ExportToo{union}{(id: T) -> \%}{}
\ExportToo{case}{(\%, 'id') -> Boolean}{}
\ExportToo{apply}{(\%, 'id') -> T}{}
\ExportToo{set!}{(\%, 'id', T) -> \%}{}

Just as with record types, constructors
are available both with a generic name (\ttin{bracket}) 
and a type-specific name (\ttin{union}).
These form union values from values belonging to the branch types.

The \ttin{case} operation tests whether the union value is in the given
branch. The \ttin{apply} operation extracts the value and the \ttin{set!}
operation modifies the value of an existing union.

Example:

\begin{small}
\begin{verbatim}
import from Union(num: Integer, rec: Record(c: Character, s: String));

-- Generic constructors
u := [3];
u := [[char "c", "hello"]];

-- Specially named constructors
u := union 3;
u := union record(char "c", "hello");
\end{verbatim}
\end{small}

\begin{small}
\begin{verbatim}
-- Construction using field names
u := [num == 3]
u := union(rec == record(char "c", "bye"));

-- Testing the case of a union.
u case num;   -- This returns false now.
u case rec;   -- This returns true now.

-- Access the union value as a record.
h := u.rec.s;
\end{verbatim}
\end{small}

Using the constructors with keyword arguments is particularly
useful when more than one branch of a union has the same type:

\begin{small}
\begin{verbatim}
MyType(R: Type, E: Type): with {
        fun: Union(coef: R, expon: E) -> %;
} == ...

-- This import causes both union branches to be `Integer'.
import from MyType(Integer, Integer);

fun [coef  == 7];
fun [expon == 7];
\end{verbatim}
\end{small}


%%%%%%%%%%%%%%%%%%%%%%%%%%%%%%%%%%%%%%%%%%%%%%%%%%%%%%%%%%%%%%%%%%%%%%%%%%%%%%
\index{Category}%
\head{section}{Category}{asugLangTypesCategory}%
\Export{Category}{Type}{}
Often it is desired to work with some specialised collection of types.
Each subtype of \verb"Type" in \asharp{} is a value which, itself,
belongs to the type \ttin{Category}.
%For example, \ttin{AbelianGroup} from the base library has as its members
%a restricted set of types --- those which are declared to have
%certain algebraic properties.
Categories allow parameterised constructions to specify the
requirements on type-valued parameters.  
A type satisfies a category if it provides all the required exports.

The keyword \ttin{with} is used to form basic \verb+Category+ values
and it is possible to test whether a type satisfies a category at evaluation
time, using \ttin{has}.
For more discussion on categories, see \secref{asugLangNTypeCats}.

%%%%%%%%%%%%%%%%%%%%%%%%%%%%%%%%%%%%%%%%%%%%%%%%%%%%%%%%%%%%%%%%%%%%%%%%%%%%%%
\index{Join}%
\head{widesection}{Join($C_1$,...,$C_n$)}{asugLangTypesJoin}
\Export{Join}{Tuple Category -> Category}{}
The type \ttin{Join($C_1$,...,$C_n$)} is a category which has
the union of all the exports of the argument categories $C_1$, ... , $C_n$.
Conditional exports have their conditions \verb"or"ed.

%%%%%%%%%%%%%%%%%%%%%%%%%%%%%%%%%%%%%%%%%%%%%%%%%%%%%%%%%%%%%%%%%%%%%%%%%%%%%%
%\index{Meet}%
%\head{section}{Meet($C_1$,...,$C_n$)}{asugLangTypesMeet}
%\Export{Meet}{Tuple Category -> Category}{}
%The type \ttin{Meet($C_1$,...,$C_n$)} is a category which has
%the intersection of all the exports of the argument
%categories $C_1$, ... , $C_n$.
%Conditional exports have their conditions \ttin{and}-ed.


%%%%%%%%%%%%%%%%%%%%%%%%%%%%%%%%%%%%%%%%%%%%%%%%%%%%%%%%%%%%%%%%%%%%%%%%%%%%%%
\head{section}{Boolean}{asugLangTypesBoolean}
\index{Boolean}
\Export{Boolean}{Type}{}
The type \ttin{Boolean} is used for the logical values \ttin{true} and
\ttin{false}.  
Values of this type are expected in the conditions of
\ttin{if} and \ttin{while}, and in other forms.
If a value in one of these contexts is not of type \ttin{Boolean}, then an
implicit call is made to a \verb+Boolean+-producing function called \ttin{test}.
See \chapref{asugLangTies}.
The base \asharp{} library extends the basic \verb+Boolean+ type.
%%%%%%%%%%%%%%%%%%%%%%%%%%%%%%%%%%%%%%%%%%%%%%%%%%%%%%%%%%%%%%%%%%%%%%%%%%%%%%

\head{section}{Literal}{asugLangTypesLiteral}
\index{Literal}
\Export{Literal}{Type}{}
The type \ttin{Literal} is used to pass the textual form of
literal constants as arguments to the constant forming operations
\ttin{integer}, \ttin{float} and \ttin{string}.
The result is in a form suitable for use with the exported conversion
operations from the \ttin{Machine} package.  
This allows constant forming functions to be evaluated at compile time.
See \chapref{asugLangTies}.

%%%%%%%%%%%%%%%%%%%%%%%%%%%%%%%%%%%%%%%%%%%%%%%%%%%%%%%%%%%%%%%%%%%%%%%%%%%%%%
\index{Generator}%
\head{section}{Generator~$T$}{asugLangTypesGenerator}%
\Export{Generator}{Type -> Type}{}
The type {\tt Generator $T$} is used to provide values of type $T$
serially to a \ttin{for} iterator.
The following program shows a function to consume the 
values of a generator:

\begin{small}%
\begin{verbatim}
consume(gg: Generator T, f: T -> ()): () ==
        for t in gg repeat
                f t;
\end{verbatim}
\end{small}

Values of type {\tt Generator $T$} 
are formed by \ttin{generate} expressions.
$T$ is the unified type of the arguments of the {\tt yield}s within the
{\tt generate}.
For more discussion on generators, see \chapref{asugLangGener}.
The base \asharp{} library provides an extension which
allows generators to be formed and manipulated using functions.

%%%%%%%%%%%%%%%%%%%%%%%%%%%%%%%%%%%%%%%%%%%%%%%%%%%%%%%%%%%%%%%%%%%%%%%%%%%%%%
\index{Exit}
\head{section}{Exit}{asugLangTypesExit}
\Export{Exit}{Type}{}
An expression of type \verb+Exit+ does not return to the invoking
context, and hence does not produce a local result.
Expressions formed with 
\ttin{break},    \keywordIndex{break}
\ttin{goto},     \keywordIndex{goto}
\ttin{iterate},  \keywordIndex{iterate}
\ttin{never},    \keywordIndex{never}
\ttin{return}    \keywordIndex{return}
and \ttin{yield} \keywordIndex{yield}
all have type \verb+Exit+ since they do not return directly to the 
expression containing them.

The unification of type \verb+Exit+ with any other type $T$ is $T$.
This allows a function to end with a \verb+return+, \eg{}:

\begin{small}%
\begin{verbatim}
#include "aldor"
f(x: Integer): Integer == { if x < 0 then x := -x; return x }
\end{verbatim}
\end{small}

Variables and defined constants may not
have type \verb+Exit+, but functions may have \verb+Exit+ as their
return type.  
A function with return type \verb+Exit+ promises
to never return to the caller.  

% "\index" after "\Export" causes an indent on the next line, so:
\index{error}\Export{error}{String -> Exit}{}
Having \verb"Exit" as the return type allows this function
to be used in writing programmer-defined error functions.  In the
standard \asharp{} library it is provided by {\tt String}.  Note that a
call to \ttin{error} will terminate the program.

Example:

\begin{small}
\begin{verbatim}
#include "aldor"

import from String;

errDenom(): Exit ==
     error "Wrong denominator";

f(n: Integer, d: Integer): Integer == {
        d > 0 => n quo d;
        d < 0 => (-n) quo (-d);
        errDenom();
}
\end{verbatim}
\end{small}
%% FIXME: Explain formatted output
%(See \secref{asugAxllibFormattedOutput} for the use of ``\verb"~a"'' in formatted
%output.)

Notice again that an expression of type \verb"Exit" 
may appear in a value context.
A consequence of this is that expressions such as the following are
completely legitimate, but odd:

\begin{small}%
\begin{verbatim}
#include "aldor"

l: List Integer := [2, 3, 4, if n < 10 then 5 else error "Too big"];

f(): Integer == return { return { return 7 } }
\end{verbatim}
\end{small}

%%%%%%%%%%%%%%%%%%%%%%%%%%%%%%%%%%%%%%%%%%%%%%%%%%%%%%%%%%%%%%%%%%%%%%%%%%%%%%
\head{section}{Foreign $I$}{asugLangTypesForeign}%
\index{Foreign}
\Export{Foreign}{Type}{}
\ExportToo{Foreign}{Type -> Type}{}
The type \ttin{Foreign} allows programs to receive values from
or provide values to programs which are not written in \asharp{}.
The result of the function \ttin{Foreign} is similar, 
but uses an interface for a particular language or environment.  

To refer to values which are not produced by an \asharp{} program,
a qualified \ttin{import} statement such as the following is used:

\begin{small}
\begin{verbatim}
import { DSQRT: DoubleFloat -> DoubleFloat } from Foreign Fortran
\end{verbatim}
\end{small}

To provide values to another environment, 
a qualified \ttin{export} statement is used:

\begin{small}
\begin{verbatim}
export { J0: DoubleFloat -> DoubleFloat } to Foreign C;
\end{verbatim}
\end{small}\index{export to}

The environments which are understood, are:

\Export{Builtin}{Type}{}%
\index{Builtin}%
The type \ttin{Builtin} can be used as an interface to abstract machine
level operations.  

The \asharp{} compiler, for instance, generates calls to functions for
some of the run-time support.  These functions are, themselves,
written in \asharp{} and made available to the abstract machine with
export declarations such as these:

\begin{small}
\begin{verbatim}
export {
        rtCacheMake:  () -> PtrCache;
        rtCacheCheck: (PtrCache, Tuple Ptr) -> (Ptr, Boolean);
        rtCacheAdd:   (PtrCache, Tuple Ptr, Ptr) -> ();
} to Foreign Builtin;
\end{verbatim}
\end{small}

%\pagebreak
\Export{C}{Type}{}
\ExportToo{C}{Literal -> Type}{}%
\index{C, interface with}%
The type \ttin{C} is used as an interface to functions written in
(or call-compatible with) the C programming language.
The type \ttin{C {\em filename}}  is used as an interface to
C functions or macros provided by a particular header file.
For details, see \chapref{asugUsingWithC}.

\Export{Lisp}{Type}{}%
\index{Lisp}%
The type \ttin{Lisp} is used as an interface to functions written in Lisp.  
This is most useful when \asharp{} programs are compiled to Lisp code, and
allows access to all the operations of the underlying Lisp system (see
\ttin{import} example above).

\Export{Fortran}{Type}{}%
\index{Fortran, interface with}%
The type \ttin{Fortran} is used as an interface to functions written in Fortran.
For details, see \chapref{asugUsingWithFortran}.
%%%%%%%%%%%%%%%%%%%%%%%%%%%%%%%%%%%%%%%%%%%%%%%%%%%%%%%%%%%%%%%%%%%%%%%%%%%%%%
\head{section}{Machine}{asugLangTypesMachine}%
\index{Machine}
\Export{Machine}{with ...}{}
A number of types and operations upon them are provided by \ttin{Machine}.
This is the basic vocabulary of hardware-oriented data types out of which
all data values in \asharp{} are composed.  

The types exported by \verb+Machine+ do not themselves export any operations.
Instead, the operations are exported by \verb"Machine" at the same level
as the types.  This is an example of what is known as a
{\em multi-sorted algebra} in the literature on type systems%
\index{multi-sorted algebra}.

Most programs do not make any reference to \ttin{Machine} or its exports.
In practice, these types and operations are used within low-level libraries
to build a set of richer basic types,
which themselves provide relevant operations.

The types exported by \verb+Machine+ are described below.
The exported operations are listed in \secref{asugLangMachine}.

\Export{XByte, HInt, SInt, BInt}{Type}{}%
\index{XByte}\index{HInt}\index{SInt}\index{BInt}%
These are integer types of various sizes.  
The types \verb"XByte", \verb"HInt", \verb"SInt" are
unsigned byte, half-precision and single-precision integer types,
capable of representing values in at least the ranges
0 to 255, -32767 to 32767 and -2147483647 to 2147483647, respectively.
The type \verb"BInt" provides a ``big'' integer type, which, in principle,
may be of any size.  In practice, the size will be limited by considerations
such as the amount of memory available to a program.

\Export{SFlo, DFlo}{Type}{}\index{SFlo}\index{DFlo}%
These are single-precision and double-precision floating point types.

\Export{Bool}{Type}{}\index{Bool}%
This type represents the logical values true and false.

\Export{Char}{Type}{}\index{Char}%
This type provides characters for natural language text.  
These should not be used to store what are logically numbers,
as the values may be translated from one character set to another
in moving data across platform.


\Export{Nil, Arr, Rec, Ptr}{Type}{}%
\index{Nil}\index{Arr}\index{Rec}\index{Ptr}%
These provide the basis for composite data structures.
Values of type \verb"Nil", \verb"Arr", and \verb"Rec" may
be converted to type \verb"Ptr" as desired.

\Export{Word}{Type -> Type}{}\index{Word}%
Values of types
\verb"SInt", \verb"SFlo" and \verb"Ptr" may be converted to this type.

\head{section}{Ref $T$}{asugLangTypesRef}%
\index{Ref}
\Export{Ref}{Type}{}

This type is mainly used when interfacing Aldor with Fortran, as described
in \chapref{asugUsingWithFortran}. 
In effect, it allows a reference to an object
to be passed as a parameter, rather than a copy of the object itself.  This
is particularly useful when dealing with Fortran routines since they often
update their arguments to return results.

In reality, when calling a Fortran routine with a {\tt Ref(T)} argument, 
the compiler passes a copy of the initial value to the Fortran code and
on exit copies the final value back into the appropriate location
(so-called {\em copy in, copy out} semantics).  Not declaring a parameter
to a Fortran routine to be a {\tt Ref(T)} is legal even if that routine
updates it, it simply means that the new value is not visible to Aldor.

When passing an Aldor routine to a Fortran program, it is necessary to
be a bit more careful and use the correct operations to update the values
of those parameters declared to be of type {\tt Ref(T)}.

This type has the following exports:

\Export{ref}{T -> \%}{}
\ExportToo{deref}{\% -> T}{}
\ExportToo{update!}{(\%,T) -> T}{}

\head{section}{Magic Types}{asugLangTypesMagic}%
\index{FileName}
\index{Pointer}
\index{MachineInteger}
\index{TextWriter}

In addition to the types defined by the language, there are a small number
of domains which need to be present for the interpreter and runtime to
function properly.  Thus if a user wishes to replace all the standard
libraries, they need to provide compatible versions.

The domains are:
\begin{itemize}
\item FileName
\item Pointer
\item MachineInteger
\item TextWriter
\end{itemize}



