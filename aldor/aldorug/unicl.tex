% *********************************************************************
\head{chapter}{The \uniclcmd{} driver}{asugUnicl}
%
%*********************************************************************

\index{unicl!{\em see \chapref{asugUnicl}}}
\newcommand{\uoption}[1]{{\tt #1}}


The \uniclcmd{} program, which is supplied in the distribution and
can be found in {\tt \$ALDORROOT/bin}, is a configurable driver for 
the C compiler and linker. It is used whenever the \asharpcmd{} options \uoption{-Fo} or 
\uoption{-Fx} are specified.

The \asharpcmd{} option \uoption{-v} can be used to show how \uniclcmd{}
is invoked and how it calls the underlying C compiler. Invoking \uniclcmd{}
on its own will display some useful information about the options it understands.
These include some options commonly found in C compilers such as 
\begin{itemize}
\item [\uoption{-O} :] Optimize.
\item [\uoption{-c} :] Compile only, do not link.
\item [\uoption{-g} :] Include debug information in generated objects.
\item [\uoption{-o \meta{name}} :] Name the executable.
\item [\uoption{-p} :] Include profiling code.
\item [\uoption{-D \meta{def}} :] Define a macro.
\item [\uoption{-U \meta{undef}} :] Undefine a macro.
\item [\uoption{-I \meta{dir}} :] Specify directory for included files.
\item [\uoption{-L \meta{dir}} :] Specify directory for linked libraries.
\item [\uoption{-l \meta{lib}} :] Specify libraries to link.
\end{itemize}

Additionally, \uniclcmd{} understands the following options

\begin{itemize}
\item [\uoption{-Wh} :] Show help information.
\item [\uoption{-Wv} :] Be verbose.
\item [\uoption{-Wv=\meta{n}} :] Set verbosity level (currently 0-4).
\item [\uoption{-Wn} :] Do not execute generated commands.
%NOT SUPPORTED \item \uoption{-R \meta{dir}} Specify directory for generated files.
\end{itemize}

An extra set of options act as flags for certain non-default \uniclcmd{} behaviour. These are
\begin{itemize}
\item [\uoption{-Wstdc} :] Use an ANSI C compiler. \\ This is passed  from \asharpcmd{}
unless \asharpcmd{} \uoption{-Cold} is specified.
\item [\uoption{-Wshared} :] Produce a shared object.
\item [\uoption{-Wfortran} :] Link in FORTRAN code. \\ This is passed from \asharpcmd{}
when \asharpcmd{} option \uoption{-Cfortran} is specified.
\item [\uoption{-Wfnonstd} :] Enable non-standard (non-IEEE) floating point. \\ This is passed
from \asharpcmd{} when \asharpcmd{} \uoption{-Q4} or \uoption{-Qcc-fnonstd} is specified.
\end{itemize}

The \uniclcmd{} program will also look for options in the {\tt UNICL} environment variable
and will prepend them to the command line options if the variable exists.

The \uniclcmd{} program itself does not have any knowledge about the underlying
C compiler for each platform. All the necessary information is read by \uniclcmd{} from 
an ASCII configuration file. By default, this file is searched for in 
{\tt \$ALDORROOT/include/aldor.conf} but the 
\begin{itemize}
\item [\uoption{-Wconfig=\meta{file}} :] Specify location of configuration file.
\end{itemize}

option can be used to select an alternative. The configuration file consists of named sections
introduced by an identifier enclosed in square brackets. Each section consists of assignments of the 
form key=value. The options
\begin{itemize}
\item [\uoption{-Wsys=\meta{name}} :] Specify which section of the configuration file to use.
\item [\uoption{-Wv=2} :] Show which section is being used.
\end{itemize}

set and show the section selected. When \asharpcmd{} calls \uniclcmd{} it automatically
\footnote {Currently, it does this
via the {\tt UNICL} environment variable so you need to give the \asharpcmd{} option
\uoption{-Cargs=-Wv=3} to see that. The \uniclcmd{} option \uoption{-Wstdc} is also passed that way
unless you specify \asharpcmd{} option \uoption{-Cold}.} 
specifies the appropriate section of the configuration file. 
The advantage of this method is that 
all the details for every implementation of \asharpName{} can be kept in one place.
It also means that the compiler user has complete control over the back end C compiler
and linker. It is a simple matter, for instance to introduce a new section with a modified name, 
populate it with some  variations on the original values and give the \asharpcmd{} option 
\uoption{-Csys=newname}.

%\newpage
These are the meaningful keys in the configuration file.
\renewcommand{\descriptionlabel}[1]%
{\texttt{#1}}


\begin{itemize}

%# name
\item The name of the program to call for compiling and linking
\begin{description}
\item    [cc-name :]  compiling only, non-ANSI C
\item    [link-name :] linking, non-ANSI C 
\item    [std-cc-name :] compiling only, ANSI C
\item    [std-link-name :] linking, ANSI C
\end{description}

%# -g
\item The flag to use when \uoption{-g} (debugging) is given 
\begin{description}
\item    [cc-debug :]  compiling only, non-ANSI C 
\item    [link-debug :]  linking, non-ANSI C
\item    [std-cc-debug :]  compiling only, ANSI C
\item    [std-link-debug :] linking, ANSI C
\end{description}

%# -p 
\item The flag to use when \uoption{-p} (profiling) is given 
\begin{description}
\item    [cc-profile :] compiling only, non-ANSI C
\item    [link-profile :] linking, non-ANSI C
\item    [std-cc-profile :] compiling only, ANSI C
\item    [std-link-profile :] linking, ANSI C
\end{description}

%# -O 
\item The flags to use when \uoption{-O} (optimise) is given (without the
option \uoption{-Wfnonstd})
\begin{description}
\item    [cc-optimize :] compiling only, non-ANSI C
\item    [link-optimize :] linking, non-ANSI C
\item    [std-cc-optimize :] compiling only, ANSI C
\item    [std-link-optimize :] linking, ANSI C
\end{description}

%# -Wfnonstd 
\item The flags to use when \uoption{-O} and \uoption{-Wfnonstd}
(non-IEEE optimisation) are given
\begin{description}
\item    [cc-non-std-float :] compiling only, non-ANSI C
\item    [link-non-std-float :] linking, non-ANSI C
\item    [std-cc-non-std-float :] compiling only, ANSI C
\item    [std-link-non-std-float :] linking, ANSI C
\end{description}

%# -Wopts
\item The flags to use in any case. These will appear before any specified files.
\begin{description}
\item    [cc-opts :] compiling only, non-ANSI C 
\item    [link-opts :] linking, non-ANSI C
\item    [std-cc-opts :] compiling only, ANSI C
\item    [std-link-opts :] linking, ANSI C
\end{description}

%# -Wpost
\item The flags to use in any case. These will appear at the end of the command.
\begin{description}
\item    [cc-post :] compiling only, non-ANSI C 
\item    [link-post :] linking, non-ANSI C
\item    [std-cc-post :] compiling only, ANSI C
\item    [std-link-post :] linking, ANSI C
\end{description}

%# -Wtwixt
\item The flags to use in any case. These will appear after filenames but before 
any library specification only at the link step.
\begin{description}
\item    [link-twixt :]  linking, non-ANSI C
\item    [std-link-twixt :] linking, ANSI C
\end{description}

%# -l 
\item The flags to use when linking libraries.
\begin{description}

\item    [library :] The flag for linking libraries (commonly \uoption{-l}).
\item    [library-sep :] Do we need a space between the flag and the name of the library (true/false).
\item    [lib-extra :] A space delimited list of libraries to be linked. \\ Example: {\tt nsl socket}. 

%# -L 
\item    [libpath :] The flag for specifying library directories (commonly \uoption{-L}).
\item    [libpath-sep :] Do we need a space between the flag and the name of the library directory (true/false).
\item    [expand-libs :] Do we turn {\tt -llib} options into complete file arguments (true/false).
\item    [lib-ext :] The file extension for libraries (commonly {\tt a}).
\item    [lib-default-path :] A space delimited list of the directories to be searched for libraries. \\ 
          Example: {\tt /usr/local/lib /usr/X11/lib}.
\end{description}


%# -I 
\item The flags to use when specifying include file directories.
\begin{description}
\item    [include :] The flag for specifying include file directories (commonly \uoption{-I}).
\item    [include-sep :] Do we need a space between the flag and the name of the include file directories (true/false).
\item    [include-default-path :]  A space delimited list of directories to be searched for include files. \\
         Examples: {\tt /usr/local/include /usr/X11/include}.
\end{description}


%# -D 
\item The flags to use when defining or undefining macros.
\begin{description}
\item    [define :] The flag for defining a macro (commonly \uoption{-D}).
\item    [define-sep :] Do we need a space between the flag and the name of the macro (true/false).

%# -U 
\item    [undefine :] The flag for undefining a macro (commonly \uoption{-U}).
\item    [undefine-sep :] Do we need a space between the flag and the name of the macro (true/false).
\end{description}

%# -o 
\item The flag to use when specifying name of linked executable.
\begin{description}
\item    [output-name :] The flag for specifying the name of the executable (commonly \uoption{-o}).
\item    [output-name-sep :] Do we need a space between the flag and the name (true/false).
\end{description}

\item Miscellaneous keys
\begin{description}
%# -c 
\item    [compile-only :] The flag that specifies compilation only (commonly \uoption{-c}).
%# General things 
\item    [debug-profile-ok :] Can we specify debugging and profiling at the same time (true/false).
\item    [debug-optimize-ok :] Can we specify debugging and optimizing at the same time (true/false).
\item    [fortran-libraries :] A string which will be passed verbatim to
the linker  when \uoption{-Wfortran} is specified (see
\secref{asugFortPlatform}).  For example: {\tt -lfort -lf77}
\end{description}

%# Aldor looks for these things
\item Keys that \asharpcmd{} looks for
\begin{description}
\item    [generate-stdc :] Do we generate ANSI C (true/false).
\item    [fortran-cmplx-fns :] The scheme to use when dealing with 
         FORTRAN functions which return complex numbers. 
         The possible values are {\tt string}, {\tt return-void}, 
         {\tt return-struct}, and {\tt disallowed}.
         See \secref{asugFortPlatform}.
\item    [fortran-naming-scheme :] Which scheme to use when resolving external FORTRAN names.
         The possible values are {\tt underscore} {\tt no-underscore} {\tt underscore-bug}.
         See \secref{asugFortPlatform}.
\end{description}
\end{itemize}

The special key {\tt inherit} takes as value a section name and directs \uniclcmd{}
to look in the named section if a key is not found in the current one. The special character
{\tt \$} prepended to a key name stands for the value of that key. You can introduce your own
keys for convenience and refer to them using {\tt \$}. The following conditions are provided 
\begin{itemize}
\item  [{\tt compile}]     the program is compiling C  code (one of `link' and `compile' will be true)
\item  [{\tt link}]        the program is linking its arguments to an executable
\item  [{\tt stdc}]       \uoption{-Wstdc} was specified
\item  [{\tt shared}]      \uoption{-Wshared} was specified
\item  [{\tt optimize}]    \uoption{-O} was specified
\item  [{\tt nonstdfloat}] \uoption{-Wnonstdfloat} was specified
\item  [{\tt stdfloat}]    \uoption{-Wnonstdfloat} was not specified
\item  [{\tt profile}]     \uoption{-p} was specified.
\end{itemize}
You can use the conditions when specifying values with this form:
\begin{small}
\begin{verbatim}
{\tt \$?condition w1 w2 w3 : w4 ;}
\end{verbatim}
\end{small}
meaning 'If condition is true then {\tt w1 w2 w3} else {\tt w4}'. Conditional 
forms can be nested.




















