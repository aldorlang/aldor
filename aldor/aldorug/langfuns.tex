%%%%%%%%%%%%%%%%%%%%%%%%%%%%%%%%%%%%%%%%%%%%%%%%%%%%%%%%%%%%%%%%%%%%%%%%%%%%%
%%%%%%%%%%%%%%%%%%%%%%%%%%%%%%%%%%%%%%%%%%%%%%%%%%%%%%%%%%%%%%%%%%%%%%%%%%%%%
%%%                                                                       %%%
%%% :: Functions
%%%                                                                       %%%
%%%%%%%%%%%%%%%%%%%%%%%%%%%%%%%%%%%%%%%%%%%%%%%%%%%%%%%%%%%%%%%%%%%%%%%%%%%%%
%%%%%%%%%%%%%%%%%%%%%%%%%%%%%%%%%%%%%%%%%%%%%%%%%%%%%%%%%%%%%%%%%%%%%%%%%%%%%

\head{chapter}{Functions}{asugLangFuncs}

Functions lie at the heart of \asharp{}:
typical expressions consist mostly of function calls.

Much of what is done by {\em ad hoc} means in other languages
is done in \asharp{} through normal functions.
It is the job of the compiler to ensure that relying on functions
in this way does not adversely affect performance.

This chapter describes how to define and use functions, beginning with
typical examples of function definition and application, then describing
more specialized features, including keyword arguments, default
arguments, function expressions (also called ``anonymous functions''),
and curried functions.

%%%%%%%%%%%%%%%%%%%%%%%%%%%%%%%%%%%%%%%%%%%%%%%%%%%%%%%%%%%%%%%%%%%%%%%%%%%%%
\head{section}{Function definition}{asugLangFunDefine}
%%%%%%%%%%%%%%%%%%%%%%%%%%%%%%%%%%%%%%%%%%%%%%%%%%%%%%%%%%%%%%%%%%%%%%%%%%%%%

A typical function definition has the following form:
\index{function definition}

\begin{small}
\begin{verbatim}
    f (s1: S1, ..., sn: Sn) : T == E
\end{verbatim}
\end{small}

This definition has a number of parts:
\begin{itemize}
\item the {\em function name\/}, {\tt f}
\item the {\em formal parameter names\/}, {\tt s1,...,sn}
\item the {\em formal parameter types\/}, {\tt S1,...,Sn}
\item the {\em return type\/}, {\tt T}, and
\item the {\em function body\/}, {\tt E}.
\end{itemize}

The {\em function name\/} is an identifier which will be used
to denote the function.
In a given scope there may be more than one functions with the same
name; in this case the function name is said to be {\em overloaded\/}.
\index{function name}
\index{overloading}

%\pagebreak
The {\em formal parameter names\/} are identifiers which are used to
refer to the values passed to the function as arguments.  The formal
parameter names are visible in the body of the function, in the types
of the formal parameters, and in the return type (See
\secref{asugLangNamesApplication}).
\index{parameter}
\index{formal parameter}
\index{dependent function}
\index{dependent type}

The {\em formal parameter types\/} are type-valued expressions
(\eg{} \ttin{Integer} or \ttin{SquareMatrix(n+m, Complex Float)})
which specify what type of value is expected as the corresponding
actual argument to the function.

The {\em return type\/} is a type-valued expression which specifies
the type of the value computed by the function.
\index{return type}

The {\em function body\/} is an expression which, when evaluated,
produces the return value of the function.  The type of the
value returned by the function body must be compatible
with the given return type.
\index{function body}

More elaborate forms of function definitions are described in sections~%
\ref{asugLangFunDefault} and \ref{asugLangFunCurry}.

%\newpage
\subsubsection{Multiple return values}

Just as functions can take any number of parameters, they may also
return any number of results.  The typical function definition given above
is a special case of the more general form:

\begin{small}
\begin{verbatim}
    f (s1: S1, ..., sn: Sn) : (t1: T1, ..., tm: Tm) == E
\end{verbatim}
\end{small}

Now in place of a single return type we have:

\begin{itemize}
\item the {\em return names\/}, {\tt t1,...,tm}, and 
\item the {\em return types\/}, {\tt T1,...,Tm}.
\end{itemize}

This function definition takes $n$ arguments and returns $m$ results.

The {\em return names\/} are identifiers which can be used 
to refer to the values returned from the function.
The return names are visible in the types of the return values.
\index{dependent function}
\index{dependent type}

The {\em return types} are type-valued expressions which specify
the type of the corresponding value returned from the function.

Any or all of the return names may be omitted,
in which case the $i\/$th return declaration would read \ttin{Ti},
rather than \ttin{ti:~Ti}.

When a function has no formal parameter or no return value
an empty pair of parentheses is used as the formal parameter list
or return value list.
For example, the following function takes no parameter and
returns no result:

\begin{small}
\begin{verbatim}
    f () : () == E
\end{verbatim}
\end{small}

\subsubsection{Return expressions}

Inside the body of a function definition, a {\em return expression\/} is
used to explicitly pass control back to the calling environment and to
return values from the function.
The general form of the return expression is:
\keywordIndex{return}

\begin{small}
\begin{verbatim}
    return E
\end{verbatim}
\end{small}

The value of the expression \ttin{E} is returned as the value
of the function.  The type of \ttin{E} must be compatible with
the declared return type of the function.  The return expression
itself has type \ttin{Exit}.
\index{return value}

Inside a function which returns more than one value the return expression
may explicitly supply more than one value:

\begin{small}
\begin{verbatim}
    return (v1, ..., vn)
\end{verbatim}
\end{small}

\vbox{
Inside a function which returns no value, an empty return expression
may be used:

\begin{small}
\begin{verbatim}
    return
\end{verbatim}
\end{small}
}

Since the value of the function body is used as the value of the function,
in many cases an explicit return expression is unnecessary:

\begin{small}
\begin{verbatim}
    f (n: Integer) : Integer == if n < 1 then 1 else n * f(n-1)
\end{verbatim}
\end{small}

The value returned by the function \ttin{f} is the value returned by
the \ttin{if} expression.  An explicit \ttin{return} is not needed.

%%%%%%%%%%%%%%%%%%%%%%%%%%%%%%%%%%%%%%%%%%%%%%%%%%%%%%%%%%%%%%%%%%%%%%%%%%%%%
\head{section}{Function application}{asugLangFunCall}
%%%%%%%%%%%%%%%%%%%%%%%%%%%%%%%%%%%%%%%%%%%%%%%%%%%%%%%%%%%%%%%%%%%%%%%%%%%%%

A typical (prefix) function application has the following form:
\index{prefix operators}
\index{function application}

\begin{small}
\begin{verbatim}
    f (a1, ..., an)
\end{verbatim}
\end{small}

This form of function application has the following parts:
\begin{itemize}
\item the {\em function\/}, {\tt f}, and
\item the {\em actual parameters\/}, {\tt a1,...,an}.
\end{itemize}

The {\em function\/} is an expression which denotes the function to be called.
In general, \ttin{f} can be any expression whose value is a function.
\index{function}

The {\em actual parameters} are expressions which specify the values
to be passed as arguments to the function.  The types of the actual
parameters must be compatible with the type of the function \ttin{f}.
\index{actual parameter}

As discussed in \secref{asugLangExprMultis}, the order of evaluation
of the actual arguments in an application is not defined.  More elaborate
%% "to" replaced by "in" - it seems to me that "application" is a meta-
%% function, the arguments to which, in this context, are (i) the
%% function being applied and (ii) its set of arguments (viewed as a
%% single object).  "In" seems clearer to me as referring to the
%% function's arguments, rather than the application's.  Similar
%% changes made later in this section.    MGR
forms of function application are developed in sections~\ref{asugLangFunKey},
\ref{asugLangFunExpr} and \ref{asugLangFunCurry}.

When a function takes no parameter, an application of that
function must have as its actual parameters an expression which produces
no value.  Often, such an application takes the following form:

\begin{small}
\begin{verbatim}
    f ()
\end{verbatim}
\end{small}

\subsubsection{Other application notations}

In addition to the normal prefix application notation there are a
small number of special syntactic forms in \asharp{} denoting function
application.  The general reason behind these rules is to make
programs more readable.  For example there is a set of infix operators
(see \secref{asugLangElementsNames}), so that you can write \ttin{a +
b} instead of \ttin{+(a,b)}.  Furthermore, some language forms cause
implicit application of functions:
\begin{itemize}
\item the treatment of literal values,
\item the application of one object to another,
\item the updating of objects,
\item the interpretation of tests in conditional statements 
\item the mechanism for generating a set of values for iteration.
\end{itemize}
Refer to \chapref{asugLangTies} for a more detailed description of these forms.

%%%%%%%%%%%%%%%%%%%%%%%%%%%%%%%%%%%%%%%%%%%%%%%%%%%%%%%%%%%%%%%%%%%%%%%%%%%%%
\head{section}{Keyword arguments}{asugLangFunKey}
%%%%%%%%%%%%%%%%%%%%%%%%%%%%%%%%%%%%%%%%%%%%%%%%%%%%%%%%%%%%%%%%%%%%%%%%%%%%%

Consider the following function definition:

\begin{small}
\begin{verbatim}
    -- Compute a point on the line with slope `m' and intercept `b'.
    line(x: DoubleFloat, m: DoubleFloat, b: DoubleFloat): DoubleFloat ==
        m*x + b;
\end{verbatim}
\end{small}

Because all the parameters have the same type, it may be difficult
to remember which one is which.  As a result, the meaning of a call such as

\begin{small}
\begin{verbatim}
    line(3.2, 8.2, 1.0);
\end{verbatim}
\end{small}

might not be readily apparent.

One way to increase the readability of such a program is to place
the arguments in named variables before calling the function:

\begin{small}
\begin{verbatim}
    slope     := 8.2;
    intercept := 1.0;

    line(3.2, slope, intercept);
\end{verbatim}
\end{small}

But this approach needlessly increases the number of variables
used by the program.  In addition, now the values for the slope
and intercept are not explicitly visible at the call point.
So one sort of unreadability has been exchanged for another 
(and {\em remembering} the order of the parameters is no easier
than before).

\subsubsection{Keyword arguments}

An alternative in \asharp{} is to allow the actual arguments
in an application to be supplied by name using {\em keyword arguments\/}.
For example:
\index{keyword argument}

\begin{small}
\begin{verbatim}
    line(3.2, b == 1.0, m == 8.2);
\end{verbatim}
\end{small}

An actual argument in this form of application has the following parts:
\begin{itemize}
\item the {\em formal parameter name}, (\eg{} {\tt b}),
\item the double-equal symbol \ttin{==} and
\item the {\em actual parameter value}, (\eg{} {\tt 1.0}).
\end{itemize}

The {\em formal parameter name} is an identifier which must match
one of the formal parameter names given in the definition of the function.

The {\em actual parameter value} is then used as the value of the formal
parameter with the same name, regardless of its position in the actual
argument list in the function application.  The type of the actual parameter
value must match the type of the formal parameter with the same name.
\index{formal parameter}
\index{actual parameter}

Any parameters supplied as keyword arguments must appear after any arguments
supplied by position alone.  It is an error if any of the formal parameters
is not supplied with a value, either as a positional argument or by using
a keyword argument.

%%%%%%%%%%%%%%%%%%%%%%%%%%%%%%%%%%%%%%%%%%%%%%%%%%%%%%%%%%%%%%%%%%%%%%%%%%%%%
\head{section}{Default arguments}{asugLangFunDefault}
%%%%%%%%%%%%%%%%%%%%%%%%%%%%%%%%%%%%%%%%%%%%%%%%%%%%%%%%%%%%%%%%%%%%%%%%%%%%%

In a programming language where function names may not be overloaded,
such as Lisp or C, some functions are written to take a variable number
of arguments.  When these functions are called, they decide which arguments
they have been passed and what to do about the missing ones.
\index{overloading}

In a language such as \asharp{}, where function names may be overloaded,
there are often several functions visible with the same name.
Which function to use is decided on the basis of the number of actual arguments
supplied and their types, and possibly on the type of the return value
required by the context of the function application.

So, instead of writing functions which take a variable number of
arguments, in \asharp{} we are allowed to write several functions with
the same name, each with a fixed number of arguments.  One advantage
of doing this is that the decision on which function to call can be made
once, at compile time, whereas code in the body of the function, to
supply missing arguments, would be exercised in each time the function
was run.

%\pagebreak
So, continuing the example from \secref{asugLangFunKey}, we can write:

\begin{small}
\begin{verbatim}
    -- Compute a point on the line with slope `m' and intercept `b'.
    line(x:DoubleFloat, m:DoubleFloat, b:DoubleFloat):DoubleFloat ==
        m*x + b;

    -- Assume a default intercept of `0'.
    line(x:DoubleFloat, m:DoubleFloat):DoubleFloat == line(x, m, 0); 
\end{verbatim}
\end{small}

Soon afterward, however, we want to give other arguments default values.
Then the number of functions needed increases exponentially in the number
of arguments which are to have default values.  

Another problem arises when arguments have the same type:
it is not always possible to overload the function name enough to
provide each argument with a default value.

For these reasons, languages with name overloading sometimes provide
an explicit way to supply default values to named parameters.

\subsubsection{Default arguments}

In \asharp{}, default values for named parameters can be supplied in
the definition of a function using the following form:
\index{default argument}

\begin{small}
\begin{verbatim}
    f (s1: S1 == v1, ..., sn: Sn == vn) : T == E
\end{verbatim}
\end{small}

This form of function definition has the following additional part:
\begin{itemize}
\item the {\em default argument value specifications\/}, in which
\ttin{==} (the \linebreak double-equal symbol) introduces each of the
{\em default argument values\/}, {\tt v1,...,vn}.
\end{itemize}

The default argument values are expressions which, when evaluated,
produce values, each of which can be used as the corresponding argument
to the function.  The type of the default value for a parameter must be
compatible with the corresponding parameter type.

Any or all of the default argument values may be omitted, in which case the
form of the $i\/$th formal parameter would read \ttin{si:~Si} instead of
\ttin{si:~Si == vi}.  A function definition which supplies a default argument
value for one of its parameters must also supply a default argument
value for each of the following parameters. 

Once again continuing the example given above, we can now define a single
function which allows any combination of its final two arguments to be omitted:

\begin{small}
\begin{verbatim}
    -- Compute a point on the line with slope `m' and intercept `b'.
    -- The default slope is 1, and the default intercept is 0.

    DF ==> DoubleFloat;

    line(x: DF, m: DF == 1, b: DF == 0) : DF == m*x + b;
\end{verbatim}
\end{small}

This definition supplies a default value of \ttin{1} for \ttin{m},
and a default value of \ttin{0} for \ttin{b}.  Some example
applications of the function \ttin{line} are as follows:

\begin{small}
\begin{verbatim}
    x: DoubleFloat := 3.2;

    stdout << line(x, 8.0) << newline;       -- 8.0 * x + 0
    stdout << line(x) << newline;            -- 1   * x + 0
    stdout << line(x, b == 5.0) << newline;  -- 1   * x + 5.0
\end{verbatim}
\end{small}

In an application of a function whose definition supplies default
argument values, it is an error if any of the formal parameters is not
supplied with a value, either as a positional argument, or by using a
keyword argument, or by using the default argument value supplied (if
any) for that formal parameter.  The default argument values are
evaluated, if and when they are used in a function application, as the
other actual parameters are evaluated.  As discussed in
\secref{asugLangExprMultis} the order of evaluation of the actual
arguments to an application is not defined.
\index{formal parameter}
\index{actual parameter}

%%%%%%%%%%%%%%%%%%%%%%%%%%%%%%%%%%%%%%%%%%%%%%%%%%%%%%%%%%%%%%%%%%%%%%%%%%%%%
\head{section}{Function expressions}{asugLangFunExpr}
%%%%%%%%%%%%%%%%%%%%%%%%%%%%%%%%%%%%%%%%%%%%%%%%%%%%%%%%%%%%%%%%%%%%%%%%%%%%%

Function expressions are the primitive form for building functions
in \asharp{}.  The general form for a function expression is:
\index{function expression}

\begin{small}
\begin{verbatim}
    (s1: S1 == v1, ..., sn: Sn == vn) : (t1: T1, ..., tm: Tm) +-> E
\end{verbatim}
\end{small}

The syntax \ttin{(s:~S)~:~T +-> E(s)} for a function expression
is intended to resemble the mathematical notation $s \mapsto E(s)$
for a function specification.  The infix keyword \ttin{+->}
denotes the $\lambda$ operator from a typed lambda calculus.
A function expression has many of the same parts as a function definition,
including the formal parameter names/types, the return names/types,
the function body (see \secref{asugLangFunDefine}), and
default arguments (see \secref{asugLangFunDefault}).

\index{closure}
When a function expression is evaluated, it captures the lexical environment
in which it appears, creating a lexical closure.  The values of the variables
which are visible in the scope of the function expression are then available
when it is eventually applied to a set of arguments.

More typical cases of function expressions are formed in much the same
way as the corresponding cases of function definitions.  For example,
the expression:

\begin{small}
\begin{verbatim}
    (f: Integer -> Integer, n: Integer) : Integer +->
        if n < 1 then 1 else n * f(n-1)
\end{verbatim}
\end{small}

represents an integer function of two arguments which bears some superficial
resemblance to a factorial function.

Since a function expression has the same parts as a function
definition except for the name, function expressions are sometimes
known as {\em anonymous functions\/}.
\index{anonymous function}
Function expressions are also known as {\em lambda expressions\/}
in many programming languages.
\index{lambda expression}

\subsubsection{Function types}

Like any other expression, a function expression can be assigned a type.
The type assigned to a function expression is called a {\em function type\/}.
A function type is formed with the infix keyword \ttin{->}:
\index{function type}

\begin{small}
\begin{verbatim}
    (s1: S1 == v1, ..., sn: Sn == vn) -> (t1: T1, ..., tm: Tm)
\end{verbatim}
\end{small}

The syntax \ttin{S -> T} for a function type is intended to be reminiscent
of the mathematical notation $S \rightarrow T$ for a set of functions.
A function type has many of the same parts as a function definition,
including the formal parameter names/types, the return names/types
(see \secref{asugLangFunDefine}), and default arguments 
(see \secref{asugLangFunDefault}).

For function types, any or all of the formal parameter names may be omitted,
in which case the $i\/$th parameter declaration would read \ttin{Si}
rather than (in the most general case) \ttin{si:~Si == vi}.

The type of \ttin{f}, (\ttin{Integer -> Integer}), in the function
expression shown previously, is a typical example of a function type.
Any function expression which takes one integer argument and returns
one integer result is a member of this type.

The ability to include the formal parameter names and default arguments
in the specification of the type is useful when using
keyword arguments (see \secref{asugLangFunKey}),
default arguments (see \secref{asugLangFunDefault}), and
dependent types (see \secref{asugLangTypesFunction}).

\subsubsection{Function definition revisited}

The typical form for a function definition given in
\secref{asugLangFunDefine} is really just a shorthand
for the following equivalent definition:
\index{function definition}

\begin{small}
\begin{verbatim}
    f : (s1: S1, ..., sn: Sn) -> T == (s1: S1, ..., sn: Sn) : T +-> E
\end{verbatim}
\end{small}

This form makes it easier to see that function definitions are the same
as definitions of any other values.  Specifically, the expression:

\begin{small}
\begin{verbatim}
    n : Integer == 8
\end{verbatim}
\end{small}

defines a type (\ttin{Integer}) and a value (\ttin{8}) for the identifier
\ttin{n}.  In the same way the function definition:

\begin{small}
\begin{verbatim}
    f (n: Integer) : Integer == if n < 1 then 1 else n * f(n-1)
\end{verbatim}
\end{small}

%\pagebreak
which can also be written as:

\begin{small}
\begin{verbatim}
    f : (n: Integer) -> Integer ==
        (n: Integer) : Integer +-> if n < 1 then 1 else n * f(n-1)
\end{verbatim}
\end{small}

defines a function type (\ttin{(n:~Integer) -> Integer}) and
a function expression for the identifier \ttin{f}.

\subsubsection{Function application revisited}

Since function expressions evaluate to functions, they can be used
% "evaluate to functions" rather than "to a function value" -
% "squared" is a function (over the integers, say) - "4" is a function
% value (squared 2).  MGR
in the place of the function name in a function application:

\begin{small}
\begin{verbatim}
    ((n: Integer) : Integer +-> if n < 1 then 1 else n * (n-1))(5);
\end{verbatim}
\end{small}

This way of calling function expressions may be rather verbose. However,
function expressions can also be assigned to local variables:

\begin{small}
\begin{verbatim}
    g := (n: Integer) : Integer +-> if n < 1 then 1 else n * (n-1);
    g(5);
\end{verbatim}
\end{small}

So once again we see that a typical function definition is merely
a special case of a more general framework, based on the fact that
function expressions can be used just like any other expression.

\vbox{
%%%%%%%%%%%%%%%%%%%%%%%%%%%%%%%%%%%%%%%%%%%%%%%%%%%%%%%%%%%%%%%%%%%%%%%%%%%%%
\head{section}{Curried functions}{asugLangFunCurry}
%%%%%%%%%%%%%%%%%%%%%%%%%%%%%%%%%%%%%%%%%%%%%%%%%%%%%%%%%%%%%%%%%%%%%%%%%%%%%

Since function expressions can be used just like any other expression,
we can write a function which returns a function as its result:%
\index{function-valued function}\index{function - function-valued}

\begin{small}
\begin{verbatim}
    DF ==> DoubleFloat;
    line (m: DF, b: DF) : (x: DF) -> DF ==
            (x: DF) : DF +-> m*x + b
\end{verbatim}
\end{small}
}

The function \ttin{line} is a function which has two formal parameters,
and which returns a function of one formal parameter.  The type of the
function returned by \ttin{line} is \ttin{(x:~DoubleFloat) -> DoubleFloat}.

A function which returns a function as its result is called
a {\em curried function\/},
after Haskell Curry, a significant contributor to the theory behind
functional programming\footnote{See, for instance, {\em Combinatory
Logic,\/} Curry and Feys, North Holland, Amsterdam 1958.}.
\index{curried function}\index{function - curried}
\index{functional programming}

To simplify the definition of curried functions in \asharp{},
two shorthand notations are provided.

%\pagebreak
\subsubsection{Function expressions revisited}

First, we generalise the syntax of function expressions
(see \secref{asugLangFunExpr}) by inductively defining
the {\em curried function expression\/}
\index{function expression}

\begin{small}
\begin{verbatim}
    (s1: S1) ... (sn: Sn)(s0: S0) : T +-> E
\end{verbatim}
\end{small}

to be equivalent to the expression

\begin{small}
\begin{verbatim}
    (s1: S1) ... (sn: Sn) : (s0: S0) -> T +-> (s0: S0) : T +-> E
\end{verbatim}
\end{small}

In the definition above, we have assumed a single formal parameter within
each set of parentheses and a single return type, to simplify the exposition.
Note that \ttin{->} and \ttin{+->} associate from right to left
and that \ttin{->} groups more tightly than \ttin{+->}.

Using the equivalent method of writing function definitions, given in
\secref{asugLangFunExpr}, and applying this shorthand to the
right-hand side, the definition of the function \ttin{line} given above
can be written as:

\begin{small}
\begin{verbatim}
    DF ==> DoubleFloat;
    line : (m: DF, b: DF) -> (x: DF) -> DF ==
            (m: DF, b: DF)(x: DF) : DF +-> m*x + b;
\end{verbatim}
\end{small}

Note that the function \ttin{line} is still a function which has
two formal parameters, and which returns a function of one formal parameter.
The only additional notation used here is the curried function expression
used to define the value of \ttin{line}.  An expression for \ttin{line} can
be written without using a curried function expression, but the shorter form
is more convenient.

\subsubsection{Function definition revisited}

As a second shorthand for defining curried functions, we define
the {\em curried function definition\/}
\index{function definition}

\begin{small}
\begin{verbatim}
    f (s1: S1) ... (sn: Sn) : T == E
\end{verbatim}
\end{small}

to be equivalent to the expression

\begin{small}
\begin{verbatim}
    f : (s1: S1) -> ... -> (sn: Sn) -> T ==
        (s1: S1) ... (sn: Sn) : T +-> E
\end{verbatim}
\end{small}

Continuing the example from the previous paragraphs, we can now express
the definition of the function \ttin{line} in its most convenient form:

\begin{small}
\begin{verbatim}
    DF ==> DoubleFloat;
    line (m: DF, b: DF)(x: DF) : DF == m*x + b;
\end{verbatim}
\end{small}

As a further example, if we define exponentiation%
\index{exponentiation of functions} for {\tt MachineInteger} functions 
of a {\tt MachineInteger} as follows:

\begin{small}
\begin{verbatim}
    MI ==> MachineInteger;
    (f:(MI->MI))^(n:MI) : (MI->MI) == {
       n = 0 => (x:MI):MI +-> x;
       n = 1 => f;
       (x:MI):MI +-> f((f^(n-1))x);
    }
\end{verbatim}
\end{small}

an alternative notation could be defined, using the curried function
conventions, as:

\begin{small}
\begin{verbatim}
    multApply(n:MI,f:(MI->MI)):MI->MI == f^n
\end{verbatim}
\end{small}

\subsubsection{Function application revisited}

The application of curried functions to their arguments needs no additional
machinery:  a curried function is just a function which returns a function,
and any expression (including the result of a function application) can be
used as a function as long as the actual parameters to the application are
compatible with the type of the function.  For example:

\begin{small}
\begin{verbatim}
    #include "aldor"
    #include "aldorio"
    DF ==> DoubleFloat;
    import from DF;
    
    line(m: DF, b: DF)(x: DF): DF == m*x + b;

    stdout << "f(x) is  x - 1" << newline;

    stdout << "f(1.0) = " << line(1,-1)(1.0) << newline;
    stdout << "f(2.0) = " << line(1,-1)(2.0) << newline;
    stdout << "f(3.0) = " << line(1,-1)(3.0) << newline;
\end{verbatim}
\end{small}

So in the application \ttin{line(1,-1)(1.0)}, the curried function \ttin{line}
is applied to the arguments \ttin{1} and \ttin{-1}, which returns a function
of one argument, which is applied to the argument \ttin{1.0}.

As a result, we can use the \ttin{line} function in this example
to create other functions:

\begin{small}
\begin{verbatim}
    #include "aldor"
    #include "aldorio"
    DF ==> DoubleFloat;
    import from DF;
    
    line(m: DF, b: DF)(x: DF): DF == m*x + b;

    f := line(1, -1);
    stdout << "f(x) is  x - 1" << newline;

    stdout << "f(1.0) = " << f(1.0) << newline;
    stdout << "f(2.0) = " << f(2.0) << newline;
    stdout << "f(3.0) = " << f(3.0) << newline;
\end{verbatim}
\end{small}

When we supply in this way only some of the arguments to a function,
the result is a new function related to the original.
This technique is a basic feature of functional programming,
and is known as {\em currying\/} a function.

Some languages provide an automatic conversion of functions of
type \verb"(A, B) -> C" to functions of type \verb"A -> B -> C".
This automatic conversion is {\em not} done in \asharp{}.
When desired, this conversion can be made explicitly with a 
function expression:

\begin{small}
\begin{verbatim}
    #include "aldor"
    #include "aldorio"
    import from Integer;

    I ==> Integer;

    -- Curry the function `*'
    ctimes == (a: I)(b: I) : I +-> a * b;
    times3 == ctimes 3;
    
    stdout << "3 * 2 = " << times3 2 << newline;

    -- Convert general functions:
    curry(f: (I, I) -> I)(a: I)(b: I) : I == f(a,b);

    stdout << "3 + 2 = " << curry(+)(3)(2) << newline;
\end{verbatim}
\end{small}
