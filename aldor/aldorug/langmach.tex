%%%%%%%%%%%%%%%%%%%%%%%%%%%%%%%%%%%%%%%%%%%%%%%%%%%%%%%%%%%%%%%%%%%%%%%%%%%%%%
%%%%%%%%%%%%%%%%%%%%%%%%%%%%%%%%%%%%%%%%%%%%%%%%%%%%%%%%%%%%%%%%%%%%%%%%%%%%%%
%%%                                                                        %%%
%%% :: The Machine Interface
%%%                                                                        %%%
%%%%%%%%%%%%%%%%%%%%%%%%%%%%%%%%%%%%%%%%%%%%%%%%%%%%%%%%%%%%%%%%%%%%%%%%%%%%%%
%%%%%%%%%%%%%%%%%%%%%%%%%%%%%%%%%%%%%%%%%%%%%%%%%%%%%%%%%%%%%%%%%%%%%%%%%%%%%%
\head{chapter}{Standard interfaces}{asugLangStdints}
\head{section}{The machine interface}{asugLangMachine}
\index{Machine}
We provide here a detailed list of the exports
from the \ttin{Machine} package.
This package provides the basic machine-level types and operations
on them to \asharp{}.
%\footnote{In the first implementation, only a
%subset of these operations is available for Lisp generation;
%they are all available when C code is being generated.}.

\Export{Bool, Char, Nil, Ptr, Arr, Rec, Word}{Type}{}
\ExportToo{XByte, HInt, SInt, BInt, SFlo, DFlo}{Type}{}
These are the types provided by \ttin{Machine}.
They are described in \secref{asugLangTypesMachine}.

\Export{true, false}{Bool}{}
\ExportToo{space, tab, newline}{Char}{}
\ExportToo{nil}{Ptr}{}
\Export{0, 1}{XByte}{}
\ExportToo{0, 1}{SInt}{}
\ExportToo{0, 1}{HInt}{}
\ExportToo{0, 1}{BInt}{}
\ExportToo{0, 1}{SFlo}{}
\ExportToo{0, 1}{DFlo}{}
\Export{min, max}{Char}{}
\ExportToo{min, max}{XByte}{}
\ExportToo{min, max}{HInt}{}
\ExportToo{min, max}{SInt}{}
\ExportToo{min, max}{SFlo}{}
\ExportToo{min, max}{DFlo}{}
\ExportToo{epsilon}{SFlo}{}
\ExportToo{epsilon}{DFlo}{}
These are special values of the various types:
\ttin{min} and \ttin{max} are the smallest (most negative) and
largest (most positive) values belonging to the finite arithmetic types;
\ttin{epsilon} is the smallest number $\varepsilon$ such that 
$1 + \varepsilon$ is distinguishable from $1$ in floating point
arithmetic.

\Export{nil?}{Ptr -> Bool}{}
\ExportToo{digit?, letter?}{Char -> Bool}{}
\ExportToo{zero?, positive?, negative?}{SInt -> Bool}{}
\ExportToo{zero?, positive?, negative?}{BInt -> Bool}{}
\ExportToo{zero?, positive?, negative?}{SFlo -> Bool}{}
\ExportToo{zero?, positive?, negative?}{DFlo -> Bool}{}
\ExportToo{even?, odd?}{BInt -> Bool}{}
\ExportToo{even?, odd?}{SInt -> Bool}{}
\ExportToo{single?}{BInt -> Bool}{}
These operations provide various tests to inquire about values.
The \ttin{single?} function tests whether a \verb"BInt" value
can be represented as a member of \verb"SInt".

\Export{=, \~{}=, <, <=}{(Bool, Bool) -> Bool}{}
\ExportToo{=, \~{}=, <, <=}{(Char, Char) -> Bool}{}
\ExportToo{=, \~{}=, <, <=}{(SInt, SInt) -> Bool}{}
\ExportToo{=, \~{}=, <, <=}{(BInt, BInt) -> Bool}{}
\ExportToo{=, \~{}=, <, <=}{(SFlo, SFlo) -> Bool}{}
\ExportToo{=, \~{}=, <, <=}{(DFlo, DFlo) -> Bool}{}
\ExportToo{=, \~{}=}{(Ptr, Ptr) -> Bool}{}
These functions provide tests for equality, inequality and order.

\Export{upper, lower}{Char -> Char}{}
\ExportToo{char}{SInt -> Char}{}
\ExportToo{ord}{Char -> SInt}{}
These operations manipulate character data.  
The \ttin{upper} and \ttin{lower} operations change the case of letters,
and leave characters which are not letters unchanged.
The \ttin{ord} and \ttin{char} are inverse operations
mapping between character and integer values.

\Export{+, -, *}{(SInt, SInt) -> SInt}{}
\ExportToo{+, -, *}{(BInt, BInt) -> BInt}{}
\ExportToo{+, -, *}{(SFlo, SFlo) -> SFlo}{}
\ExportToo{+, -, *}{(DFlo, DFlo) -> DFlo}{}
\ExportToo{/}{(SFlo, SFlo) -> SFlo}{}
\ExportToo{/}{(DFlo, DFlo) -> DFlo}{}
\ExportToo{\^{}}{(BInt, SInt) -> BInt}{}
\ExportToo{\^{}}{(BInt, BInt) -> BInt}{}
These are the arithmetic operations of addition,
subtraction, multiplication, division and exponentiation.

%\pagebreak
\Export{\_*\_+}{(SInt, SInt, SInt) -> SInt}{}
\ExportToo{\_*\_+}{(BInt, BInt, BInt) -> BInt}{}
\ExportToo{\_*\_+}{(SFlo, SFlo, SFlo) -> SFlo}{}
\ExportToo{\_*\_+}{(DFlo, DFlo, DFlo) -> DFlo}{}
These are ``multiply-add'' operations: 
\verb"_*_+(a,b,c)" is mathematically equivalent to \verb"a*b + c"
but may be faster or have less rounding error.

\Export{quo, rem, mod, gcd}{(SInt, SInt) -> SInt}{}
\ExportToo{quo, rem, mod, gcd}{(BInt, BInt) -> BInt}{}
\ExportToo{divide}{(SInt, SInt) -> (SInt, SInt)}{}
\ExportToo{divide}{(BInt, BInt) -> (BInt, BInt)}{}
These are operations related to integer division.  
The \ttin{divide} operations return a quotient-remainder pair.

\Export{mod\_+, mod\_-, mod\_*}{(SInt, SInt, SInt) -> SInt}{}
\ExportToo{mod\_*inv}{(SInt, SInt, SInt, DFlo) -> SInt}{}
These operations perform arithmetic modulo their third arguments.
The \ttin{mod\_*inv} operation is equivalent to \ttin{mod\_*}
but expects an accurate floating point approximation
to the inverse of the modulus as its last argument.

\Export{\~{}}{Bool -> Bool}{}
\ExportToo{\~{}}{SInt -> SInt}{}
\ExportToo{\slope\back}{(Bool, Bool) -> Bool}{}
\ExportToo{\slope\back}{(SInt, SInt) -> SInt}{}
\ExportToo{\back\slope}{(Bool, Bool) -> Bool}{}
\ExportToo{\back\slope}{(SInt, SInt) -> SInt}{}
% Better to have "/" and "\" in the same typeface.
% (Both in \tt style would be nicer, but how?)
These are bit-wise logical operations.

\Export{length}{SInt -> SInt}{}
\ExportToo{length}{BInt -> SInt}{}
\ExportToo{bit}{(SInt, SInt) -> Bool}{}
\ExportToo{bit}{(BInt, SInt) -> Bool}{}
\ExportToo{shiftUp, shiftDown}{(SInt, SInt) -> SInt}{}
\ExportToo{shiftUp, shiftDown}{(BInt, SInt) -> BInt}{}
These functions provide access to the bits in the base-2 representation
of integers.

\Export{next, prev}{SInt -> SInt}{}
\ExportToo{next, prev}{BInt -> BInt}{}
\ExportToo{next, prev}{SFlo -> SFlo}{}
\ExportToo{next, prev}{DFlo -> DFlo}{}
These functions provide next and previous values of the given type.
For example, \ttin{next 1} is 2 for \verb"SInt" and \verb"BInt" values,
but might be approximately
\verb"1.0000000000000002220446"
for \verb"DFlo", depending on the platform.

\Export{double\_*}{(Word, Word) -> (Word, Word)}{}
\ExportToo{doubleDivide}{(Word, Word, Word) -> (Word, Word)}{}
\ExportToo{plusStep}{(Word, Word, Word) -> (Word, Word)}{}
\ExportToo{timesStep}{(Word, Word, Word, Word) -> (Word, Word)}{}
These are the basic operations underpinning multi-word arithmetic.%
The first two functions
provide double word integer multiply and divide operations.
The double word multiply instruction returns its results as a pair.
The double word division accepts the dividend as a pair,
passed as the first two arguments,
the divisor as the third argument, and returns the quotient and remainder.
The \ttin{plusStep} operation adds two values and a carry to produce
a result and a carry.
The \ttin{timesStep} operation computes the product of its first two
arguments, adds in the other two, and represents the result as a word pair.
This may be summarised as:

\begin{small}
\begin{verbatim}
double_*:  (rH, rL)    = divide(a * b,            wordsize)
plusStep:  (kout,  r)  = divide(a + b + kin,      wordsize)
timesStep: (rHout, rL) = divide(a * b + c + rHin, wordsize)
\end{verbatim}
\end{small}


\Export{assemble}{(SInt, SInt, Word) -> SFlo}{}
\ExportToo{assemble}{(SInt, SInt, Word, Word) -> DFlo}{}
\ExportToo{dissemble}{SFlo -> (SInt, SInt, Word)}{}
\ExportToo{dissemble}{DFlo -> (SInt, SInt, Word, Word)}{}
These operations allow manipulation of the
sign $s = \pm 1$, exponent $e$, 
and mantissa $x$ of floating point numbers, $f = s \times x \times 2^e$.


\ExportToo{nearest, zero, up, down, any}{() -> SInt}{}
\ExportToo{round}{SFlo -> BInt}{}
\ExportToo{round}{DFlo -> BInt}{}
\ExportToo{round\_+, round\_-, round\_*, round\_/}{(SFlo, SFlo, SInt) -> SFlo}{}
\ExportToo{round\_+, round\_-, round\_*, round\_/}{(DFlo, DFlo, SInt) -> DFlo}{}
These provide rounded floating point arithmetic.%
The third argument
of the \verb"round_"{\it op\/} functions is a rounding mode, as supplied
by \verb"Bnearest",
\verb"Bzero" (toward zero),
\verb"Bup" (toward plus infinity),
\verb"Bdown" (toward minus infinity),
or \verb"Bany" (doesn't matter).

\Export{convert}{SFlo -> DFlo}{}
\ExportToo{convert}{DFlo -> SFlo}{}
\ExportToo{convert}{XByte -> SInt}{}
\ExportToo{convert}{SInt -> XByte}{}
\ExportToo{convert}{HInt -> SInt}{}
\ExportToo{convert}{SInt -> HInt}{}
\ExportToo{convert}{SInt -> BInt}{}
\ExportToo{convert}{BInt -> SInt}{}
\ExportToo{convert}{SInt -> SFlo}{}
\ExportToo{convert}{SInt -> DFlo}{}
\ExportToo{convert}{BInt -> SFlo}{}
\ExportToo{convert}{BInt -> DFlo}{}
\ExportToo{convert}{Ptr -> SInt}{}
\ExportToo{convert}{SInt -> Ptr}{}
\ExportToo{convert}{Arr -> SFlo}{}
\ExportToo{convert}{Arr -> DFlo}{}
\ExportToo{convert}{Arr -> SInt}{}
\ExportToo{convert}{Arr -> BInt}{}
These functions convert values between the various \verb"Machine" types.
Most of the operations may be accomplished by simply padding or narrowing
the internal representation of values.
Others such as the integer to floating point conversion are more complex.

\Export{format}{(SInt, Arr, SInt) -> SInt}{}
\ExportToo{format}{(BInt, Arr, SInt) -> SInt}{}
\ExportToo{format}{(SFlo, Arr, SInt) -> SInt}{}
\ExportToo{format}{(DFlo, Arr, SInt) -> SInt}{}
\ExportToo{scan}{(Arr, SInt) -> (SFlo, SInt)}{}
\ExportToo{scan}{(Arr, SInt) -> (DFlo, SInt)}{}
\ExportToo{scan}{(Arr, SInt) -> (SInt, SInt)}{}
\ExportToo{scan}{(Arr, SInt) -> (BInt, SInt)}{}
These operations allow the conversion of numeric data to and from
a textual form.

\Export{array}{(E: Type) -> (E, SInt) -> Arr}{}
\ExportToo{get}{(E: Type) -> (Arr, SInt) -> E}{}
\ExportToo{set!}{(E: Type) -> (Arr, SInt, E) -> E}{}
These operations allow the implementation of array types.

\Export{dispose!}{Arr -> ()}{}
\ExportToo{dispose!}{BInt -> ()}{}
Calling one of these operations indicates that the storage for
the argument will no longer be used.  On some platforms this makes the
storage available for immediate re-use, while on others it does not
become available until after the next garbage collection.
If a program does not make calls to \ttin{dispose!}, then the storage
is reclaimed by garbage collection.

\Export{RTE}{() -> SInt}{}
\ExportToo{OS}{() -> SInt}{}
These operations allow programs
to inquire about the platform on which they are running.

The \ttin{RTE} operation indicates the run time environment.
This is sometimes useful in selecting between alternative foreign imports.
The following values may be returned:

\begin{tabular}{rl}
0 & The environment cannot be determined. \\
1 & The program is linked as a C-based application. \\
2 & The program is running on top of a Common Lisp system.
\end{tabular}

The \ttin{OS} operation specifies the operating system.
This is sometimes useful in understanding how to parse file names,
for example.
The values below correspond to the given operating systems:

\begin{tabular}{cl}
OS() quo 1000 & Operating system \\
\hline
0 & Cannot be  determined. \\
1 & A Unix derivative. \\
2 & IBM VM/CMS. \\
3 & OS/2. \\
4 & DOS for IBM PC compatibles. \\
5 & Microsoft Windows. \\
6 & DEC VMS. \\
7 & Macintosh System 7
\end{tabular}

\Export{halt}{SInt -> Exit}{}
Finally, this operation terminates the execution of a program.
On platforms which support it, the integer argument is returned
to the calling environment.
