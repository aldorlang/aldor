%%%%%%%%%%%%%%%%%%%%%%%%%%%%%%%%%%%%%%%%%%%%%%%%%%%%%%%%%%%%%%%%%%%%%%%%%%%%%%
%%%%%%%%%%%%%%%%%%%%%%%%%%%%%%%%%%%%%%%%%%%%%%%%%%%%%%%%%%%%%%%%%%%%%%%%%%%%%%
%%%                                                                        %%%
%%% :: Standard Libraries
%%%                                                                        %%%
%%%%%%%%%%%%%%%%%%%%%%%%%%%%%%%%%%%%%%%%%%%%%%%%%%%%%%%%%%%%%%%%%%%%%%%%%%%%%%
%%%%%%%%%%%%%%%%%%%%%%%%%%%%%%%%%%%%%%%%%%%%%%%%%%%%%%%%%%%%%%%%%%%%%%%%%%%%%%
\head{section}{Standard libraries}{asugLangStdlibs}

Programs will normally use more than just the language-defined types.  
The example programs in this guide make use of 
the standard \asharp{} library, also called \libaldor{}.
\index{libraries}

%%%%%%%%%%%%%%%%%%%%%%%%%%%%%%%%%%%%%%%%%%%%%%%%%%%%%%%%%%%%%%%%%%%%%%%%%%%%%%
{\bf The standard \asharp{} library}
\index{libaldor}
provides a set of basic types for numbers, data structures, objects, 
input, output and so on.
%and is described in \chapref{asugBaselibCats} and \chapref{asugBaselibDoms}.
This library should be used when developing
stand-alone programs, or programs to link with C- or Fortran-based applications.

The simplest way to make the base \asharp{} library available within
an \asharp{} is to use a line of the form
\begin{verbatim}
#include "aldor"
\end{verbatim}
This incorporates the text of the standard header file \fname{aldor.as}
into the program being compiled.

The standard \fname{aldorio.as} can also be included:
\begin{verbatim}
#include "aldor"
#include "aldorio"
\end{verbatim}

While \fname{aldor.as} makes the library visible, this extra include file also
defines a few standard macros and imports a few basic operations from types in
the library.  So, for example, after including these files, \ttin{stdout} and
\ttin{<<} have meanings.

This header file uses a \verb"#library" command to 
make the library archive \fname{libaldor.al} available
as a package called \ttin{aldorlib} within the program.
Normally it is not necessary to use the name \ttin{aldorlib} at all, but
if you are using other libraries which introduce colliding exports,
you can disambiguate the collisions by referring to the
base library entities with a \verb"$"-qualification,
such as ``\verb"Integer$aldorlib"'' or ``\verb"+$Integer$aldorlib"''.

%%%%%%%%%%%%%%%%%%%%%%%%%%%%%%%%%%%%%%%%%%%%%%%%%%%%%%%%%%%%%%%%%%%%%%%%%%%%%%
%AXIOM {\bf The \axiom{} library} is most useful when writing programs for use with
%AXIOM the \axiom{} system.  This library is described in the book
%AXIOM {\em \axiom{}: the scientific computation system} by Jenks and Sutor,
%AXIOM published by Springer Verlag.
%AXIOM Examples of using the \axiom{} library with \asharp{} are given in
%AXIOM \chapref{asugUsingAxiom} and \secref{AxiomSample}.
%AXIOM \index{AXIOM}
%AXIOM 
%AXIOM The simplest way to make the \axiom{} library available within an \asharp{}
%AXIOM program is to use the following line
%AXIOM \begin{verbatim}
%AXIOM #include "axiom"
%AXIOM \end{verbatim}
%AXIOM This associates the \axiom{} library archive with the name \ttin{AxiomLib}
%AXIOM in the program.  Similar comments apply to disambiguating using
%AXIOM \linebreak \verb+$+-qualification as applied for \ttin{AxlLib}.
%AXIOM 
%AXIOM The \axiom{} library is quite large, and the types it provides are very
%AXIOM interdependent (the base clique has 125 mutually recursive interfaces)
%AXIOM so compiling files against this library is quite a bit slower and takes
%AXIOM much more space than compiling files against the base \asharp{} library.
%AXIOM 
