%%%%%%%%%%%%%%%%%%%%%%%%%%%%%%%%%%%%%%%%%%%%%%%%%%%%%%%%%%%%%%%%%%%%%%%%%%%%%%
%%%%%%%%%%%%%%%%%%%%%%%%%%%%%%%%%%%%%%%%%%%%%%%%%%%%%%%%%%%%%%%%%%%%%%%%%%%%%%
%%%                                                                        %%%
%%% :: Language Orientation
%%%                                                                        %%%
%%%%%%%%%%%%%%%%%%%%%%%%%%%%%%%%%%%%%%%%%%%%%%%%%%%%%%%%%%%%%%%%%%%%%%%%%%%%%%
%%%%%%%%%%%%%%%%%%%%%%%%%%%%%%%%%%%%%%%%%%%%%%%%%%%%%%%%%%%%%%%%%%%%%%%%%%%%%%
\head{chapter}{Language orientation}{asugLangSummary}

\head{section}{Traditional and non-traditional aspects}{asugSummTrad}

In many regards, \asharp{} is a very conventional programming language:
\begin{itemize}
\item
  The language uses {\em explicit evaluation}, with the usual control flow.
  Functions play an important role,
  and most of a program's text consists of function definitions.  
\item
  Computed values may be saved in named {\em variables}
  or named {\em constants}. 
\item
  The language is {\em statically typed} so the nature of each variable
  is understood before evaluation begins.
  It is possible to write programs with all names explicitly
  declared, but the types and scopes of names are inferred if desired.
\item
  {\em Overloading} of names is possible: there
  may be several different meanings for a given name visible simultaneously.  
  The meaning of each occurrence is determined by the type required in context.
  Ambiguities may be explicitly resolved by stating the type or origin of
  the given name.
\item
  Functions may be called with {\em named parameters}, which may have
  {\em default values}.
\end{itemize}
It is possible to transliterate most programs written
in languages such as C, C++, Fortran or Lisp almost
directly into \asharp{}.

In other regards, \asharp{} is somewhat unconventional:
\begin{itemize}
\item 
  {\em Functions are first class values}
  and the language provides
  useful ways to create and manipulate them dynamically. 
  (Many languages allow functions to be used as values,
  but have no way to create or combine them during execution.)
\item
  {\em Types are first class values} and the language 
  provides useful ways to create and manipulate them dynamically.
\item
  {\em Dependent types} provide the freedom to defer decisions until
  program execution, but provide enough discipline to allow
  safe, efficient programs.
\item
  Programs may make independent, {\em post facto extensions} to libraries.
  This allows existing types to belong to newly defined families
  without modifying the original code.
\item
  Generators and type-specific tests provide {\em control abstraction}, 
  \ie{}~a way for looping and branching expressions to use new types.
  While the language is statically typed,
  types themselves are dynamic values.
\item
  The efficiency of {\em symbolic} and of {\em numeric} computation
  received equal consideration in the design of the language.
\end{itemize}

Some of these language aspects, when taken together and handled uniformly,
have particularly powerful interactions.  

We give one example:
Since types are first class values, functions may 
take types as parameters and return types.  
Let us call such a type-to-type function a ``functor.''
Since functions are first class,  so are these functors.
Therefore it is possible to write programs which manipulate functors.
A functor-manipulating program is discussed in \secref{swapSample}.

Programs such as these may be used to reorganise levels in type towers ---
for instance, to make the dynamic choice to convert values in the type
\begin{verbatim}
   Array List Integer
\end{verbatim}
to values in the type
\begin{verbatim}
   List Array Integer.
\end{verbatim}

\head{section}{Expressions and evaluation}{asugSummExprEval}

An \asharp{} program consists of a set of expressions
representing a computation to be performed.
Performing the computation is called {\em evaluation} or {\em execution}
and produces a set of {\em values}.

Each value has an associated {\em type}, which dictates how the value
is to be represented in computer memory.
%% PI: in computer memory -> in variables
Expression evaluation may cause other actions beyond producing values.
These additional actions are called {\em side-effects} and include
things such as output and modifying the values stored in computer memory.
\index{type}

The syntax of the language allows larger expressions to be built up from
smaller ones.  The evaluation of an expression
may involve the evaluation of some or all of its subexpressions,
and its value can be formed using the values of the subexpressions.

Elements such as {\em if\/}s, loops, and function forms
%% PI: {\em if\/}s -> {\em if\/} statements
are all expressions in \asharp{}.
This contrasts with some other languages, such as C or Fortran,
where they are special ``statements'' 
which can only appear in limited contexts.

The rules of the \asharp{} programming language ensure that, in a
well-formed program, all the values potentially produced by any given
expression have the same type.
%% PI, the previous sentence may generate confusion. How doesn't know
%% anything about A# could interpreter that all the values produced
%% have the same type. I would say:
%% ..., in a well-formed program, any expression produces always
%% values of the same type.
This has two consequences:
it allows programs to be transformed to faster, equivalent code,
by avoiding type tests during execution, and, more importantly,
it allows a common class of programming errors to be detected immediately.

We say the \asharp{} programming language is {\em expression based}
and {\em statically typed}.

\head{section}{Functions}{asugSummFunctions}

Functions  are treated as values in the same way as integers,
lists or floating point numbers.  
The language provides mechanisms for composing and 
manipulating functions in useful ways, 
incorporating them in other data structures, or returning them as results.  

Functions may depend on values defined externally to them so the act 
of dynamically creating a new function value captures the creation environment,
forming what is normally called a {\em closure}.
\index{environment}
\index{closure}

\head{section}{Domains}{asugSummDomains}

Often, several functions operate within a common environment.  
For this reason, in \asharp{}, an environment is called a
{\em domain of computation}, or {\em domain} for short.  
Environments are essentially dictionaries which tie names to values.
The language allows environments to be created and manipulated dynamically,
and these form the basis for abstract types,
packages, and objects as first-class values.
\index{domain}
\index{package}
\index{object}
\index{environment}

The language provides general mechanisms to allow new types to 
be used in all the same ways as built-in types:
they may provide the same sort of literal constants, 
participate in the same control structures,
admit the same optimisations, \etc{}
To ensure this equal status, the built-in types make use of the
same general mechanisms to provide their function.

This has had two consequences:
first, the extension mechanisms are pervasive and powerful;
second, the language itself has very little built in.
The language provides a minimal set of primitive types
and operations.  
These are combined and extended in standard libraries to provide
a rich set of facilities.

\head{widesection}{Compilation}{asugSummCompiltn}

Normally, evaluation does not directly use the source form of the
expressions in a program.  Rather, the evaluation is usually effected by
first translating the source expressions to an equivalent set of 
lower-level instructions more suitable for the target environment.
This is the job of the \asharp{} compiler.  The result of the translation
is a program which produces the same values and side-effects as the
original program, but which might otherwise be represented very differently.
\index{side-effects}
%% PI, previous paragraph: The result of the translation is a program
%% in a language closer to the machine representation but with exactly the
%% same behaviour of the \asharp{} program. It's said that the two
%% programs are {\em semantically equivalent}. The motivation behind this
%% transformation is that the user will write code in a high-level
%% language, with a formalism not too different from the mathematical
%% one, but the transformed program has closer to the way in which the
%% machine works, so can be executed much more efficiently.

\head{section}{Libraries}{asugSummLibraries}

\asharp{} is geared to developing and using combinations of libraries.
Certain properties of the language have therefore been oriented
to controlling the dependencies and interactions among libraries of programs.

A {\em source program} can see only the declared public behaviour
of the files it uses.

A {\em compiled program} can depend on the private behaviour of the
files it uses, but only when given explicit permission to do so.
It is up to the client to decide
whether it is willing to become dependent on
the private behaviour of the provider.

In practice, this permission amounts to increased scope
for optimisation.
In no case does it allow a source-level dependency.
