%%%%%%%%%%%%%%%%%%%%%%%%%%%%%%%%%%%%%%%%%%%%%%%%%%%%%%%%%%%%%%%%%%%%%%%%%%%%%%
%%%%%%%%%%%%%%%%%%%%%%%%%%%%%%%%%%%%%%%%%%%%%%%%%%%%%%%%%%%%%%%%%%%%%%%%%%%%%%
%%%                                                                        %%%
%%% :: Expressions
%%%                                                                        %%%
%%%%%%%%%%%%%%%%%%%%%%%%%%%%%%%%%%%%%%%%%%%%%%%%%%%%%%%%%%%%%%%%%%%%%%%%%%%%%%
%%%%%%%%%%%%%%%%%%%%%%%%%%%%%%%%%%%%%%%%%%%%%%%%%%%%%%%%%%%%%%%%%%%%%%%%%%%%%%

\head{chapter}{Expressions}{asugLangExpr}

\asharp{} is an {\em expression-based\/} language:  every construct
in the language produces zero, one or more values.  This chapter describes
the structure of expressions in \asharp{} and the rules for expression
evaluation.

\def\Headline#1#2{\Secref{#2} & #1 & page \pageref{#2} \\ }
\def\headline#1#2{\secref{#2} & #1 & page \pageref{#2} \\ }

\begin{tabular}{lp{2.85in}l}

\Headline{Names}                {asugLangExprNames}             % Done
\Headline{Literals}             {asugLangExprLiterals}          % Done
\Headline{Definitions}          {asugLangExprDefine}            % Done
\Headline{Assignments}          {asugLangExprAssign}            % Done
\Headline{Functions}            {asugLangExprFunExpr}           % Done
\Headline{Function Calls}       {asugLangExprFunCall}           % Done
\Headline{Imperatives}          {asugLangExprDo}                % Done
\Headline{Multiple Values}      {asugLangExprMultis}            % Done
\Headline{Sequences}            {asugLangExprSequences}         % Done
\Headline{Exits}                {asugLangExprExit}              % Done
\Headline{If}                   {asugLangExprIf}                % Done
\Headline{Select}               {asugLangExprSelect}            % Done
\Headline{Logical Expressions}  {asugLangExprLogicals}          % Done
\Headline{Loops}                {asugLangExprLoop}              % Done
\Headline{Generate Expressions} {asugLangExprGenerators}        % Done
\Headline{Collections}          {asugLangExprCollections}       % Done
\Headline{General Branching}    {asugLangExprGotos}             % Done
\Headline{Never}                {asugLangExprNever}             % Done

\end{tabular}

%\pagebreak
%%%%%%%%%%%%%%%%%%%%%%%%%%%%%%%%%%%%%%%%%%%%%%%%%%%%%%%%%%%%%%%%%%%%%%%%%%%%%%
\head{section}{Names}{asugLangExprNames}
%%%%%%%%%%%%%%%%%%%%%%%%%%%%%%%%%%%%%%%%%%%%%%%%%%%%%%%%%%%%%%%%%%%%%%%%%%%%%%

\index{name}
The evaluation of a name in \asharp{} causes the value of a
variable or constant to be retrieved.  

If a name refers to a variable or constant defined in the same scope or in
an enclosing scope then retrieving its value is a very inexpensive operation.  
If a name refers to an imported constant, then there may be a cost associated
with the look up, depending on the degree of optimisation used to compile the
program.

Consider the following example:

\begin{small}
\begin{verbatim}
    #include "aldor"

    f(n: MachineInteger): MachineInteger ==
            if n < 2 then
                    1
            else
                    n * f(n-1);
\end{verbatim}
\end{small}

In the last line, the name \ttin{n} is defined in the current scope
and the name \ttin{f} is defined in an enclosing scope, so their use
is very inexpensive.  The names \ttin{*}, \ttin{-} and \ttin{1} are
imported from \verb"MachineInteger".  If this program is compiled with
optimisation, then there is no cost in using the values from
\verb"MachineInteger".  Without optimisation, these values
would have to be dynamically retrieved from \verb"MachineInteger"
at a modest cost.

A name may represent a {\em variable\/}, which may take on different values
at different points in a computation, or a {\em constant\/}, which
always refers to the same value every time it is used.

Names for constants may be {\em overloaded\/} in \asharp{}.  That is,
it is possible to have more than one constant with a given name,
visible at the same point in a program.  Names for variables cannot be
overloaded.  A name cannot represent both a variable and a constant in
the same scope. 

%%%%%%%%%%%%%%%%%%%%%%%%%%%%%%%%%%%%%%%%%%%%%%%%%%%%%%%%%%%%%%%%%%%%%%%%%%%%%%
\head{section}{Literals}{asugLangExprLiterals}
%%%%%%%%%%%%%%%%%%%%%%%%%%%%%%%%%%%%%%%%%%%%%%%%%%%%%%%%%%%%%%%%%%%%%%%%%%%%%%

\index{literal}
Literal constants are expressions which represent the explicit data values
which appear in a program.  There are three styles of literal constants
in \asharp{}:  quoted strings, integers and floating-point numbers:

\begin{small}
\begin{verbatim}
    "Aloha!"                    -- quoted string literal
    10203040  16rFFFFC010       -- integer literals
    1.234e56                    -- floating-point literal
\end{verbatim}
\end{small}
The meaning of a constant in \asharp{} is determined by the environment
in which it is used.  For example, the constant \ttin{1.234e56} might
be a value of type \verb"SingleFloat" or \verb"DoubleFloat",
depending on the context.

When a literal expression is encountered in a program, it is treated
as an application of a corresponding ``literal accepting''
operation:

\begin{small}
\begin{verbatim}
    string:  Literal -> T
    integer: Literal -> T
    float:   Literal -> T
\end{verbatim}
\end{small}
(where \ttin{T} represents the type of the value being formed).
Each of these operations takes a single argument of type \verb"Literal"
and constructs a value of the appropriate type.  When programs are compiled
against the base \asharp{} library, the constant-folding optimisation
will immediately convert constants of the following types to their machine
representations:  \verb"String", \verb"MachineInteger", \verb"Integer",
\verb"SingleFloat", \verb"DoubleFloat".

New types may provide their own interpretation of literal constants
by exporting a literal forming operation with the corresponding signature.
As a consequence, if operations for creating string literals, for example,
are available from several types, it may be necessary to provide a declaration
to indicate which kind of literal is intended.  When implementing
literal-forming operations for new types, it is often useful to use the
\verb"string" literal-forming operation from \verb"String".

Some types may accept some literal values and not others.
For example, a fixed-precision integer type might reject values
which lie outside the range of values representable by the type.

\subsubsection{String literals}

\index{string literal}
\index{escape character}
String-style literals are enclosed in quotation marks.
Inside a string-style literal, an underscore character \ttin{\_} is used
as an escape character to modify the meaning of the characters which follow:

\begin{itemize}
\item A quotation mark after an underscore includes a quotation mark
in the literal instead of ending the literal.
\item An underscore after an underscore includes a single underscore
in the literal.
\item White space (any number of blanks, tabs or newlines) following
an underscore is ignored, and not included in the literal.
\end{itemize}

Examples:

\begin{small}
\begin{verbatim}
    "This literal is a _"string-style_" literal."
    "This literal contains a single underscore: '__'."

    s := "This literal is moderately long, and is broken _
          over two lines, even though the result is a single line."
\end{verbatim}
\end{small}

When using \libaldor{}
the type \verb"String" provides string-style literals.

\subsubsection{Integer literals}

\index{integer literal}
Integer-style literals provide a syntax for whole numbers.
These are in base ten unless otherwise indicated.

An integer-style literal is either
\begin{itemize}
\item a sequence of one or more digits, [{\tt 0}-{\tt 9}]+, with the
% distinguishing symbols (0 9) from metasymbols ([ - ] =).
      exception of a single \ttin{0} or single \ttin{1}, or
\item a sequence of one or more digits giving a radix (base), followed
      by the letter \ttin{r}, followed by a number of radix-digits using
      digits and/or capital letters:
            [{\tt 0}-{\tt 9}]+{\tt r}[{\tt 0}-{\tt 9A}-{\tt Z}]+.
\end{itemize}

%%% Base-n integers not working - aldorbug report sent 19940923.  MGR

In \asharp{} the numerals \ttin{0} and \ttin{1} are {\em not\/}
literal constants -- they are treated as names so that
various mathematical structures which export 0 or 1 can do so,
without being required to support general integer constants.

\index{escape character}
An underscore appearing in the middle of an integer-style literal
is ignored together with any following white space.
An underscore which appears at the very beginning of a word
causes the whole word to be treated as an identifier, rather than
as a literal constant.

Examples:

\begin{small}
\begin{verbatim}
    22_394_547

    38319238471239487123948237_
      192387491234712398478188_
      139823712983712938712391

    _33    -- This word is an identifier, not an integer-style literal.

    2r01010101010101010101    -- Base 2
    16rDEADBEEF               -- Base 16
\end{verbatim}
\end{small}

When using \libaldor{}, the types \verb"MachineInteger"
and \verb"Integer" provide integer-style literals.

%%% What is the ``{\em NN\/{\tt r}...\/}'' radix notation?  If the radix not'n
%%% just described, then what does support it?  (If it isn't supported at all,
%%% why mention it?)   MGR

%\pagebreak
\subsubsection{Floating-point literals}

\index{floating-point literal}
Float-style literals are numbers with a decimal point, an exponent, or both.

Examples:

\begin{small}
\begin{verbatim}
    3.0        3.       3e1     6.022E+23 
    0.2        .2       2e-1    4.8481E-6
\end{verbatim}
\end{small}

\index{escape character}
An underscore appearing in the middle of a float-style literal
is ignored together with any following white space.

\begin{small}
\begin{verbatim}
    3.14159_26535_89793_23846_26433_83279_50288_41971_
      69399_37510_58209_74944_59230_78164_06286_20899_
      86280_34825_34211_70679_82148_08651_32823_06647
\end{verbatim}
\end{small}

When using \libaldor{}, the types \verb"SingleFloat" and 
\verb"DoubleFloat" provide float-style literals.

%%%%%%%%%%%%%%%%%%%%%%%%%%%%%%%%%%%%%%%%%%%%%%%%%%%%%%%%%%%%%%%%%%%%%%%%%%%%%%
\head{section}{Definitions}{asugLangExprDefine}
%%%%%%%%%%%%%%%%%%%%%%%%%%%%%%%%%%%%%%%%%%%%%%%%%%%%%%%%%%%%%%%%%%%%%%%%%%%%%%

\index{constant}
\index{defined constant}
\index{definition expression}
A {\em constant\/} in \asharp{} denotes a particular value which
cannot be changed.  The general syntax for a constant definition is:

\begin{small}
\begin{verbatim}
    x : T == E
\end{verbatim}
\end{small}

\verb"x" is an identifier giving the name of the constant.

\verb"T" is an expression giving the type of the constant.
The type declaration is optional.  If the type is declared, then the
type of \verb"E" must satisfy it.  Otherwise the type is inferred
from the type of the expression \verb"E".

\verb"E" is an expression which computes the value of the constant.

Examples:

\begin{small}
\begin{verbatim}
    a : Integer == 23;
    b == "Hello world!";
\end{verbatim}
\end{small}

A function definition is a special case of a constant definition.
Function definitions are described in more detail in
\secref{asugLangFunDefine}.

%\pagebreak
Once a value has been given to a named constant, it cannot be
changed to refer to another value.

\begin{small}
\begin{verbatim}
    -- A constant cannot be changed to refer to a different value.
    num: Integer == 3;
    num: Integer == 4;  -- invalid

    -- In fact, it cannot be changed to refer to the same value!
    hi:  String  == "'Ello!";
    hi:  String  == "'Ello!"; -- invalid
\end{verbatim}
\end{small}

%%%%%%%%%%%%%%%%%%%%%%%%%%%%%%%%%%%%%%%%%%%%%%%%%%%%%%%%%%%%%%%%%%%%%%%%%%%%%%
\head{widesection}{Assignments}{asugLangExprAssign}
%%%%%%%%%%%%%%%%%%%%%%%%%%%%%%%%%%%%%%%%%%%%%%%%%%%%%%%%%%%%%%%%%%%%%%%%%%%%%%

\index{variable}
\index{assignment expression}
A {\em variable\/} in \asharp{} denotes a value which may change
during the evaluation of a program.
A variable is given a value by an {\em assignment\/} of the form:

\begin{small}
\begin{verbatim}
    x : T := E
\end{verbatim}
\end{small}

\verb"x" is an identifier giving the name of the variable.

\verb"T" is an expression giving the type of the variable.
The type declaration is optional.  If the type is declared, then the
type of \verb"E" must satisfy it.  Otherwise the type is inferred
from the type of the expression \verb"E".

\verb"E" is an expression which computes the value of the variable.

Several variables may be assigned a value at the same time:

\begin{small}
\begin{verbatim}
    (x1 : T1, ..., xn: Tn) := E
\end{verbatim}
\end{small}

Any or all of the type declarations may be omitted,
in which case the $i$\/th variable would read \ttin{xi},
rather than \ttin{xi:~Ti}, and the type of \verb"xi" is
inferred from the type of the expression \verb"E".

Examples:

\begin{small}
\begin{verbatim}
    n: Integer := 3;
    k := 3*n + 1;
    n := k quo 2;

    (a, b) := (1, 3);
    (s: String, x) := ("Natascia", false);
\end{verbatim}
\end{small}

The value of an assignment expression is the same as the value of \verb"E".

A special form of assignment expression is used to provide a general kind of
updating operation:

\begin{small}
\begin{verbatim}
    x(a1, ..., an) := E
\end{verbatim}
\end{small}

Typically, \verb"x" is an expression which evaluates to a structured data
value, such as an array or a list, and the expressions \verb"ai" taken
together specify some component of \verb"x".  An assignment expression of
this form is treated as an application of the operation \ttin{set!} of the
form: 

\verb^    ^{\tt set!(x, a1, ..., an, E)} 

For example, for lists, the \ttin{set!} function
%(see \secref{asugAxllibListCategory})
takes as its second (component specifying)
parameter a \ttin{MachineInteger} specifying the position
of the element in the list, so we could have:
\begin{small}
\begin{verbatim}
    #include "aldor";
    import from MachineInteger, List MachineInteger;
    L := [1,2,3];
    ...
    L(1) := 4;
\end{verbatim}
\end{small}
which would result in \verb"L" having the value \verb"list(4, 2, 3)".

The value of this form is the return value of the function
\ttin{set!}.

%%%%%%%%%%%%%%%%%%%%%%%%%%%%%%%%%%%%%%%%%%%%%%%%%%%%%%%%%%%%%%%%%%%%%%%%%%%%%%
\head{section}{Functions}{asugLangExprFunExpr}
%%%%%%%%%%%%%%%%%%%%%%%%%%%%%%%%%%%%%%%%%%%%%%%%%%%%%%%%%%%%%%%%%%%%%%%%%%%%%%

\index{function expression}
Function expressions are the primitive form for building functions
in \asharp{}.  
An example of a function expression is:

\begin{small}
\begin{verbatim}
    (n: Integer, m: Integer): Integer +-> n * m + 1
\end{verbatim}
\end{small}

See \secref{asugLangFunExpr} for a complete description.

%%%%%%%%%%%%%%%%%%%%%%%%%%%%%%%%%%%%%%%%%%%%%%%%%%%%%%%%%%%%%%%%%%%%%%%%%%%%%%
\head{section}{Function calls}{asugLangExprFunCall}
%%%%%%%%%%%%%%%%%%%%%%%%%%%%%%%%%%%%%%%%%%%%%%%%%%%%%%%%%%%%%%%%%%%%%%%%%%%%%%

\index{function call}
Typical expressions consist mostly of function calls.
For example, consider the expression

\begin{small}
\begin{verbatim}
    l: List Integer := [2 + 3, 4 - 5, 6 * 7, 8 ^ 9]
\end{verbatim}
\end{small}

This has six explicit function calls (\ie{}~to the arithmetic functions
\ttin{+}, \ttin{-}, \ttin{*}, \quotedCaret, to the $n$-ary function \ttin{bracket}
(called using the syntax \ttin{[...]}), and to the type constructor function \ttin{List}.
This expression also has eight implicit function calls to the literal
forming operation \ttin{integer}.

See \secref{asugLangTiesAppSyntax} for a complete description
of the syntax of function calls.

%%%%%%%%%%%%%%%%%%%%%%%%%%%%%%%%%%%%%%%%%%%%%%%%%%%%%%%%%%%%%%%%%%%%%%%%%%%%%%
\head{section}{Imperatives}{asugLangExprDo}
%%%%%%%%%%%%%%%%%%%%%%%%%%%%%%%%%%%%%%%%%%%%%%%%%%%%%%%%%%%%%%%%%%%%%%%%%%%%%%

\index{imperative expression}
\keywordIndex{do}

The \verb"do" expression evaluates \verb"E" and discards the computed value,
so that the \verb"do" expression returns no value.

\begin{small}
\begin{verbatim}
    do E
\end{verbatim}
\end{small}

%%%%%%%%%%%%%%%%%%%%%%%%%%%%%%%%%%%%%%%%%%%%%%%%%%%%%%%%%%%%%%%%%%%%%%%%%%%%%%
\head{section}{Multiple values}{asugLangExprMultis}
%%%%%%%%%%%%%%%%%%%%%%%%%%%%%%%%%%%%%%%%%%%%%%%%%%%%%%%%%%%%%%%%%%%%%%%%%%%%%%

\index{multiple values}
\index{,@{\tt ,} (comma)}
A series of comma-separated expressions is used in \asharp{} to produce
{\em multiple values\/}.  The expressions are evaluated one by one, and the
results are taken together as the (multiple) value of the whole expression:

\begin{small}
\begin{verbatim}
    3, 4, 5
\end{verbatim}
\end{small}
This expression produces three values, all of type Integer.
In general, an expression in \asharp{} produces zero, one or more
values, each having its own type.  For convenience,
nevertheless, we often speak of {\em the\/} value and {\em the\/} type
of an expression, even if it produces multiple values, when
the intended meaning is clear from the context.

See \secref{asugLangElementaryGrouping} for a discussion of the use
of parentheses in comma expressions.

Functions may be declared to accept or return multiple values.
The example below shows how to declare, define and use a function which
involves multiple values.

\begin{small}
\begin{verbatim}
    #include "aldor"
    #include "aldorio"

    -- Declaring a function to accept and return multiple values.
    local f: (Integer, Integer) -> (Integer, Integer);

    -- Defining a function to accept and return multiple values.
    f(i: Integer, j: Integer): (Integer, Integer) == (i+j, i-j);

    -- Using a function which accepts and returns multiple values.
    (q: Integer, r: Integer) := divide f(100, 93);

    -- Printing the result
    stdout << "The quotient is  " << q << newline;
    stdout << "The remainder is " << r << newline;
\end{verbatim}
\end{small}

The call to \verb"f" returns \verb"(193, 7)",
which are passed as arguments to the function \ttin{divide}
from \verb"Integer".  This returns

\begin{small}
\begin{verbatim}
    (193 quo 7, 193 rem 7)
\end{verbatim}
\end{small}

which are assigned to \verb"q" and \verb"r" respectively.

Comma-separated expressions are not necessarily evaluated
in any particular order; furthermore, the evaluation of their
subexpression may be interleaved.  Thus, the program:

\begin{small}
\begin{verbatim}
    #include "aldor"
    #include "aldorio"

    pr2(a: MachineInteger, b: MachineInteger): () ==
        stdout << a << " " << b << newline;

    n: MachineInteger := 1;

    pr2({n := n + 1; n}, {n := n + 1; n})
\end{verbatim}
\end{small}

could print any of \ttin{2 3}, \ttin{3 2}, \ttin{2 2} or \ttin{3 3},
depending on the implementation and whether the code is running in a
multiprocessor environment.
Programs which depend on the order of evaluation of expressions to
be used as arguments to a function should use a sequence to make
the order explicit.  (See \secref{asugLangExprSequences}.)

%%%%%%%%%%%%%%%%%%%%%%%%%%%%%%%%%%%%%%%%%%%%%%%%%%%%%%%%%%%%%%%%%%%%%%%%%%%%%%
\head{section}{Sequences}{asugLangExprSequences}
%%%%%%%%%%%%%%%%%%%%%%%%%%%%%%%%%%%%%%%%%%%%%%%%%%%%%%%%%%%%%%%%%%%%%%%%%%%%%%

\index{sequence}
A series of expressions to be evaluated one after another is called a
\linebreak {\em sequence\/}.  A sequence is written as a
semicolon-separated series of subexpressions:

\begin{small}
\begin{verbatim}
    a := 1; b := a + a; c := 3*b
\end{verbatim}
\end{small}

The expressions (that is, the subexpressions of the sequence)
are evaluated one by one, in the order of their
occurrence, and the value of the last expression evaluated
is used as the value of the sequence.  A sequence may also
contain one or more exit expressions, as described in
\secref{asugLangExprExit}, which prevent the evaluation of any
expressions later in the sequence and so provide a way to return
a value other than that of the last expression in the sequence.

Because semicolon has a relatively low precedence,
it is usually necessary to enclose a sequence in braces (\ttin{\{ \}})
or parentheses (\ttin{( )}) to get the desired result.
(See \secref{asugLangElementaryGrouping} for details.)

Examples:

\begin{small}
\begin{verbatim}
    #include "aldor"

    import from MachineInteger;

    n1 := (a := 1; b := a + a;  3 * b);
    n2 := {a := 1; b := a + a;  3 * b}

    f(i0: MachineInteger): MachineInteger == {
            a := i0;
            b := a + a;
            3 * b
    }
\end{verbatim}
\end{small}
    
The meaning of a sequence is the same whether braces or parentheses are
used.  Braces are normally used, especially to enclose a longer
expression split over several lines.  Parentheses are occasionally used
to enclose shorter sequences as part of other expressions.  An implicit
semicolon is assumed after a closing brace but {\em not} after a closing
parenthesis.

%\pagebreak
It is also possible to use indentation to construct sequences
by enclosing lines between the directives \ttin{\#pile} and \ttin{\#endpile}.
In this context, a group of consecutive lines indented by the same amount
is called a {\em pile\/} and is treated as a sequence.
The precise rules for forming piles are described in \secref{asugFLangLayout}.

%%%%%%%%%%%%%%%%%%%%%%%%%%%%%%%%%%%%%%%%%%%%%%%%%%%%%%%%%%%%%%%%%%%%%%%%%%%%%%
\head{section}{Exits}{asugLangExprExit}
%%%%%%%%%%%%%%%%%%%%%%%%%%%%%%%%%%%%%%%%%%%%%%%%%%%%%%%%%%%%%%%%%%%%%%%%%%%%%%

\index{exit expression}
\index{=>@{\tt =>} (exit expression)}
An {\em exit expression} has the form:

\begin{small}
\begin{verbatim}
    condition => E
\end{verbatim}
\end{small}

When an exit expression appears as one of the elements of a sequence,
the condition is evaluated.  If the condition evaluates to \verb"true"
then the value of the sequence is the value of the expression \verb"E",
and no further components of the sequences are evaluated.
If the condition evaluates to \verb"false" then evaluation continues with
the next expression in the sequence.  So the expression:

\begin{small}
\begin{verbatim}
    { a; b; c => d; e; f }
\end{verbatim}
\end{small}

is equivalent to

\begin{small}
\begin{verbatim}
    { a; b; if c then d else { e; f } }
\end{verbatim}
\end{small}

In a sequence which contains an exit expression, the type of the expression
\verb"E" must be compatible with the type of the sequence, that is, with
the type of the final element of the sequence\footnote{In the current release of
\asharp{}, this means that the types must be the same.}%
\index{compatible}.  If a sequence contains several exit expressions,
the types of the possible exit values must
%be mutually compatible.
all be compatible.
%% "mutually" suggests pairwise, which may not be adequate.

If the condition is not of type \verb"Boolean", then an implicit
application of the function \ttin{test} is performed to convert the
condition to the type \verb"Boolean".  (See \secref{asugLangTiesTests}.)
%% \secref{asugLangFunCall} says nothing about "test".   MGR

\index{Exit}
%An exit expression either transfers control elsewhere, or produces zero
%values.
%
%% I found the above sentence difficult to follow - do you mean the
%% following?  (It seems as if the "either .. or" should be "and".) MGR
An exit expression transfers control; it does not, itself, produce a
value.
As a result, the type of the exit expression is \verb"()", and so
an exit expression can only be used in a context which does not require
a value.
Examples:

\begin{small}
\begin{verbatim}
    #include "aldor"

    import from Integer;

    b: Integer := 1;

    a := { n := b * b; n < 10 => 0; n > 100 => 100 ; n }

    -- `a' will be assigned the value `0', corresponding to the
    -- right hand side of the first `=>'
\end{verbatim}
\end{small}

Note that all of the exit values (\ie{}~\ttin{0} and \ttin{100})
have type \verb"Integer", and the final element of the sequence
also has type \verb"Integer".

A series of exit expressions is often a compact way to enumerate
a list of alternative cases:

\begin{small}
\begin{verbatim}
    #include "aldor"
    #include "aldorio"

    import from Integer;

    power(base: Integer, exp: Integer): Integer == {
        exp = 0 => 1;
        exp = 1 => base;
        exp = 2 => base * base;
        exp = 3 => base * base * base;

        val := 1;

        for i in 1..exp repeat val := val * base;

        val;
    }

    stdout << power(2, 0) << newline;  -- Print `1'
    stdout << power(2, 1) << newline;  -- Print `2'
    stdout << power(2, 2) << newline;  -- Print `4'
    stdout << power(2, 3) << newline;  -- Print `8'
    stdout << power(2, 9) << newline;  -- Print `512'
\end{verbatim}
\end{small}

Since any expression can appear after the \ttin{=>}, another sequence,
which may contain other exit expressions, can appear there:

\begin{small}
\begin{verbatim}
    #include "aldor"

    import from Integer;

    +++ If `a' or `b' is zero, then return 0;
    +++ Otherwise return 1 if (a * b > 0), -1 elsewhere.
    productSign(a: Integer, b: Integer): Integer == {
        a = 0 => 0;
        b = 0 => 0;
        a < 0 => { b < 0 => 1; -1 }
        b > 0 => 1;
        -1 
    }
\end{verbatim}
\end{small}

If an exit expression appears as a strict subexpression of an expression
% (Design question: why not allow an exit expression itself to be treated
%  as a sequence of length one?  MGR)
other than a sequence, the exit expression is treated as a sequence
of length one:

\begin{small}
\begin{verbatim}
    if b < 0 then a > 0 => flag := false;
\end{verbatim}
\end{small}
% Example modified, since difficult at this point 
% to see how original could be used.

This example is treated as equivalent to:

\begin{small}
\begin{verbatim}
    if b < 0 then { a > 0 => flag := false }
\end{verbatim}
\end{small}



%%%%%%%%%%%%%%%%%%%%%%%%%%%%%%%%%%%%%%%%%%%%%%%%%%%%%%%%%%%%%%%%%%%%%%%%%%%%%%
\head{section}{If}{asugLangExprIf}
%%%%%%%%%%%%%%%%%%%%%%%%%%%%%%%%%%%%%%%%%%%%%%%%%%%%%%%%%%%%%%%%%%%%%%%%%%%%%%

\keywordIndex{if}
\keywordIndex{then}
\keywordIndex{else}
Conditional branching in \asharp{} is provided by the {\em if expression\/}.

\begin{small}
\begin{verbatim}
    if condition then T
    if condition then T else E
\end{verbatim}
\end{small}

When an \verb"if" expression is evaluated, the \verb"condition" is evaluated.
If the condition evaluates to \verb"true" then the value of the \verb"if"
expression is the value of the expression \verb"T".
If the condition evaluates to \verb"false" and the \verb"else" clause
is present, then the value of the \verb"if" expression is the value
of the expression \verb"E".
If the \verb"else" clause is not present, then the \verb"if" expression
returns no value.

An {\tt if} expression used in a context which requires a value must
have both a {\tt then} and an {\tt else} branch.  The types of the two
branches must be compatible, and the type of the expression is the
type which is satisfied by both branches.

In contexts which do not require a value,
the types of the {\tt then} and {\tt else} branch
are independent.   In this case, the type of the {\tt if} expression
is taken to be the type of the empty expression, and it is possible
to omit the {\tt else} branch altogether.

% Type of the if expression in category context.

If \verb"condition" is not of type \verb"Boolean", then an implicit
application of the function \ttin{test} is performed to convert
\verb"condition" to type \verb"Boolean".  (See \secref{asugLangTiesTests})

Example:

\begin{small}
\begin{verbatim}
    #include "aldor"
    import from MachineInteger;

    foo(a: MachineInteger): MachineInteger ==
        if a > 0 then a * a else 0
\end{verbatim}
\end{small}

Note that if the \verb"else" clause is not present, then the \verb"if"
expression cannot be used in a context which requires a value:

\begin{small}
\begin{verbatim}
    -- This assignment is not type correct.
    a : MachineInteger := if true then 1;
\end{verbatim}
\end{small}

\clearpage
%%%%%%%%%%%%%%%%%%%%%%%%%%%%%%%%%%%%%%%%%%%%%%%%%%%%%%%%%%%%%%%%%%%%%%%%%%%%%%
\head{section}{Select}{asugLangExprSelect}
%%%%%%%%%%%%%%%%%%%%%%%%%%%%%%%%%%%%%%%%%%%%%%%%%%%%%%%%%%%%%%%%%%%%%%%%%%%%%%
\keywordIndex{select}
\keywordIndex{in}
\operatorIndex{case}

A select statement takes the form:
\begin{small}
\begin{verbatim}
select E in {
        V1 => E1;
        V2 => E2;
        ...
        En;
}
\end{verbatim}
\end{small}

where {\tt E}, {\tt E1}, {\tt E2} and {\tt En} are arbitrary expressions of
the same type. {\tt V1}, {\tt V2} are also arbitrary expressions any do
not have to be the same type.

The statement is interpreted as follows: for each option ({\tt V1}, {\tt V2},
...) in turn , the function {\tt case} is called with the value of {\tt E} as
its first argument, and the value of the option as the second argument.
If the call to {\tt case} is true, then the code on the right of the {\tt =>}
is executed.  If no option is satisfied, then the default code {\tt En} is
executed.

A {\tt select} statement is type-correct if there exists {\tt case}
functions with type
\begin{small}
{\tt ({\em typeof({\tt E})}, {\em typeof({\tt V})}) -> Boolean}
\end{small}
for each {\tt V} in {({\tt V1}, {\tt V2}, ...)}.

Given an appropriate {\tt case} function, one can write:

\begin{small}
\begin{verbatim}
select gcd(p, q) in {
        0 => stdout << "gcd is zero";
        1 => stdout << "gcd is one";
        stdout << "gcd is complicated"
}


select getOption() in {
           0 => option1();
           1 => option2();
        2..5 => optionX();
        illegalOption();
}
\end{verbatim}
\end{small}

Unions also export a case function, so one can do:

\begin{small}
\begin{verbatim}
U ==> Union(var: Symbol, poly: Poly);
u: U := ...
select u in {
        var => ...
        poly => ...
        never
}
\end{verbatim}
\end{small}

%%%%%%%%%%%%%%%%%%%%%%%%%%%%%%%%%%%%%%%%%%%%%%%%%%%%%%%%%%%%%%%%%%%%%%%%%%%%%%
\head{section}{Logical expressions}{asugLangExprLogicals}
%%%%%%%%%%%%%%%%%%%%%%%%%%%%%%%%%%%%%%%%%%%%%%%%%%%%%%%%%%%%%%%%%%%%%%%%%%%%%%

Logical expressions in \asharp{} are provided using the following forms:

\keywordIndex{and}
\keywordIndex{or}
\keywordIndex{not}

\begin{small}
\begin{verbatim}
    E1 and E2
    E1 or  E2
    not E
\end{verbatim}
\end{small}

An \ttin{and} expression is \verb"true" if both \verb"E1" and \verb"E2" are
\verb"true".  If the first expression evaluates to \verb"false", then
the second expression is not evaluated.

An \ttin{or} expression is \verb"true" if \verb"E1" or \verb"E2" or
both are \verb"true".  If the first expression evaluates to \verb"true",
then the second expression is not evaluated.

A \ttin{not} expression is \verb"true" if \verb"E" is \verb"false".

If any of \verb"E1", \verb"E2" or \verb"E" is not of type \verb"Boolean",
then an implicit application of the function \ttin{test} is performed
to convert the value to type \verb"Boolean".  (See \secref{asugLangTiesTests})
The type of a logical expression is \verb"Boolean".

Examples:

\begin{small}
\begin{verbatim}
    #include "aldor"
    #include "aldorio"

    if true and false then stdout << "This string will not be printed.";

    import from MachineInteger;

    -- Define `test' for MachineInteger:  a null value returns false.
    test(x: MachineInteger): Boolean == if x = 0 then false else true;

    if 1 or 1 then stdout << "I can do this." << newline;

    -- Define `test' for String:  an empty string returns false.
    -- Note that `#s' (`the length of s') uses
    -- the previous test for MachineInteger.
    test(s: String): Boolean == if #s then true else false;

    if true and 1 and "I can do this, too." then
        stdout << "This string will be printed." << newline;
\end{verbatim}
\end{small}
The logical connectives \ttin{and}, \ttin{or} and \ttin{not} are
syntactic elements of the language that should not be confused with
similar functions exported from the type \verb"Boolean"
(\verb"/\", \verb"\/" and \verb"~").  The functional versions from
\verb"Boolean" evaluate all of their arguments before computing their
result, and denote function values.  The logical connectives cannot be
used as functions:

\begin{small}
\begin{verbatim}
    #include "aldor"
    #include "aldorio"

    import from List Boolean;

    -- This expression will print:  `[T,T,F]'
    stdout << ((map ~) [false, false, true]) << newline;
\end{verbatim}
\end{small}

but you cannot say:  {\small \verb+((map not) [false, false, true])+},
since \ttin{not} is {\em not\/} a function. The \asharp{} function
\ttin{map}, defined for all {\em BoundedFiniteLinearStructureType} types.
In the case of \verb+List+s, \ttin{map} is defined as a function taking
a function (for example, \ttin{~}) as an argument and returning a function
taking a \ttin{List} as an argument and returning a \ttin{List}.
%(see \secref {asugAxllibAggregate}
%applies its first argument to each component of its second argument.)

%%%%%%%%%%%%%%%%%%%%%%%%%%%%%%%%%%%%%%%%%%%%%%%%%%%%%%%%%%%%%%%%%%%%%%
%%%%%%%%%%%%%%%%%%%%%%%%%%%%%%%%%%%%%%%%%%%%%%%%%%%%%%%%%%%%%%%%%%%%%%%%%%%%%%
\head{widesection}{Loops}{asugLangExprLoop}
%%%%%%%%%%%%%%%%%%%%%%%%%%%%%%%%%%%%%%%%%%%%%%%%%%%%%%%%%%%%%%%%%%%%%%%%%%%%%%

\subsubsection{Repeat}
\keywordIndex{repeat}

%% Loops allow expressions to be evaluated many times for the side-effects
%% they produce.  
%% These side-effects might include variable assignments, output and so on.  
%% The values of the repeated expression are discarded, and the loop, itself,
%% produces no value.

The \asharp{} language provides a set of constructs to handle loops
in a way which is both elegant and efficient. The concept lying at the
base of \asharp{} loops is that of {\em generator}, discussed
in depth in \chapref{asugLangGener}. Generators provide an abstract
way to traverse values belonging to certain domains without violating
the principle of {\em encapsulation} (see \secref{asugLangNTypeDomains},
page \pageref{asugLangNTypeDomEncap}). Generators can be considered as
autonomous entities producing values. This section shows how values
they produce can be used to control loops.

The general form of a loop expression is:

\verb^    ^{\em Iterators\/} {\tt repeat }{\em Body}

The iterators control how many times the body is evaluated.
Any number of iterators may be used on a loop, and each is
either a \ttin{while} or a \ttin{for} iterator.

A loop with no iterator repeats forever, unless terminated by an
error or some other event.
For example:

\begin{small}
\begin{verbatim}
    #include "aldor"
    #include "aldorio"

    -- This will repeat forever...
    repeat {
            stdout << "Row, row, row your boat," << newline;
            stdout << "Gently down the stream."  << newline;
            stdout << "Merrily, merrily, merrily, merrily," << newline;
            stdout << "Life is but a dream."     << newline;
            stdout << newline;
    }
\end{verbatim}
\end{small}

Two forms, \ttin{break} and \ttin{iterate,} may be used within the
loop body to control the loop evaluation, as described later
on pages \pageref{asugLangBreak} and \pageref{asugLangIterate}.

%%%%%%%%%%%%%%%%%%%%%%%%%%%%%%%%%%%%%%%%%%%%%%%%%%%%%%%%%%%%%%%%%%%%%%%%%%%%%%%
\head{subsubsection}{While-iterators}{asugLangWhile}
%%%%%%%%%%%%%%%%%%%%%%%%%%%%%%%%%%%%%%%%%%%%%%%%%%%%%%%%%%%%%%%%%%%%%%%%%%%%%%%
\keywordIndex{while}

A \ttin{while} iterator allows a loop to continue so long as a condition
remains true.  The syntax of a \ttin{while} iterator is

\egindent{\tt while }{\em condition}

If a loop has a \ttin{while} iterator, then at the beginning of each
iteration {\em condition} is evaluated.
The result is then tested:
\begin{itemize}
\item If it is {\em true}, the evaluation of the loop {\em proceeds}.
\item If it is {\em false}, then the loop is {\em terminated}.
\end{itemize}

%\pagebreak
Just as with \ttin{if} and other expressions having tests for control
flow, if the condition of a \ttin{while} is not a \verb"Boolean"
value, then an appropriate \ttin{test} function is applied to
determine the sense of the condition.
\keywordIndex{test}
\index{control abstraction}

The following example shows a {\tt repeat} loop using a {\tt while} iterator:

\label{asugLangWhileExample}

\begin{small}
\begin{verbatim}
    #include "aldor"
    #include "aldorio"

    import from Integer;

    n := 10000;
    k := 0;         -- `k' counts the number of iterations.

    while n ~= 1 repeat {
            k := k + 1;

            if odd? n then n := 3*n + 1 else n := n quo 2;
    }

    stdout << "Terminated after " << k << " iterations." << newline;
\end{verbatim}
\end{small}

This loop counts the number of times the body has to be evaluated in
order for \verb"n" to reach one.  When \verb"n" reaches 1, the
evaluation of the loop is terminated, and the count is printed.
(It is a well-known conjecture that this process will terminate for all
integers $n > 0$.  This is sometimes called the ``$3 n + 1$
problem''.)

In a case such as this it is important to give \verb"k" an initial value,
since the loop may be iterated zero times!
In fact, if the {\tt while} condition is false the first time that it is
evaluated, then the {\tt repeat} body will be not executed at all:

\begin{small}%
\begin{verbatim}
    #include "aldor"
    #include "aldorio"

    import from Integer;

    n := 0;

    while n > 0 repeat {
        -- This will never be executed.
        stdout << "Hello world!" << newline;
    }
\end{verbatim}
\end{small}

%%%%%%%%%%%%%%%%%%%%%%%%%%%%%%%%%%%%%%%%%%%%%%%%%%%%%%%%%%%%%%%%%%%%%%%%%%%%%%%
\head{subsubsection}{For-iterators}{asugLangFor}
%%%%%%%%%%%%%%%%%%%%%%%%%%%%%%%%%%%%%%%%%%%%%%%%%%%%%%%%%%%%%%%%%%%%%%%%%%%%%%%
\keywordIndex{for}
\keywordIndex{in}
\index{generators}

%% Very often loops are used to operate on a sequence of values,
%% say elements of an array or a list.  PI: `sequence' has a special
%% meaning in this chapter. I rewrote:

Very often, loops are used to traverse certain kind of data structures,
such as lists, arrays or tables, or to execute some operations for all
the numerical values in a defined range.

%\pagebreak
\ttin{for} iterators make this sort of loop convenient to write.
Take, for instance, the following examples:

\begin{small}
\begin{verbatim}
    #include "aldor"
    #include "aldorio"

    -- Add up the elements of a list, return the sum.
    sum(ls: List Integer): Integer == {
            n := 0;
            for i in ls repeat n := n + i;
            n
    }

    -- Add up the elements of an array, return the sum.
    sum(arr: Array Integer): Integer == {
            n := 0;
            for i in arr repeat n := n + i;
            n
    }

    -- Add up the odd numbers in a range, return the sum.
    sum(lo: Integer, hi: Integer): Integer == {
            n := 0;
            if even? lo then lo := lo + 1;

            for i in lo..hi by 2 repeat n := n + i;
            n
    }


    import from List Integer, Integer;

    -- Use `sum: List Integer -> Integer':
    stdout << sum([1,2,3,4])     -- Prints `10'.

\end{verbatim}
\end{small}

The \ttin{for} iterators in these examples all have the form

\egindent{\tt for {\em lhs\/} in {\em Expr}}


The {\em lhs\/} part specifies a variable which will take on the
successive values over the course of the iteration.
The {\em lhs\/} has the form

\egindent{[ {\tt free} ] {\em name} [: {\em Type\/} ]}

The type declaration is optional.  If it is missing, the type of the
variable is inferred from the source of the values, {\em Expr}.

The \ttin{free} part of the {\em lhs} determines the scope of the variable.
\keywordIndex{free}
Without it, a new variable of the given name is made local to the loop.
If the word \ttin{free} is present, then the loop uses an existing
variable from the context.  In this case, the value of the variable
is available after the loop terminates.
%% PAB: The ought to point to env. chapter

%\pagebreak
The example on page \pageref{asugLangWhileExample} can be rewritten using a
free loop variable:

\label{asugLangForExample} 

\begin{small}
\begin{verbatim}
    #include "aldor"
    #include "aldorio"

    import from Integer;

    n := 10000;
    k := 0;                      -- Make a top-level variable `k'.

    for free k in 1.. repeat {   -- Use above `k' freely in the loop.

            if odd? n then n := 3*n + 1 else n := n quo 2;

            if n = 1 then break;  -- exit the loop if n = 1
    }

    -- Here the last value of `k' is available.
    stdout << "Terminated after " << k << " iterations." << newline;
\end{verbatim}
\end{small}

The previous example uses the \ttin{break} construct explained later
in this section, on page \pageref{asugLangBreak}.
When a {\tt break} is evaluated it causes the
termination of the loop. In this case, the {\tt break} is the only
exit point of the loop, because we are iterating over an open segment,
producing infinitely many values: 1, 2, 3, ....

Now we turn our attention to the expression traversed by the
\ttin{for} iterator.
In the examples, we have seen
\begin{itemize}
\item a list, \ttin{ls},
\item an array, \ttin{arr}, and
\item integer segments, \ttin{lo..hi by 2} and \ttin{1..}.
\end{itemize}

In general, the \ttin{for} iterator expression must have type
{\tt Generator {\em T\/}}, where {\em T\/} is the type of the 
\ttin{for} variable.   If the expression is not of this type,
then an implicit call to \ttin{generator} is inserted.
Any visible \verb"generator" function of an appropriate type will be used.
\index{control abstraction}
\index{generator function}
See \chapref{asugLangGener} for a description of generators and how
to create new ones.

The examples we have seen work because the list, array
and segment types provided by \libaldor{} export \verb"generator"
functions.

%\subsubsection{For-suchthat}
\index{"|@\verb="|= (such that)}

There is a second form of \ttin{for} iterator which filters the
values used.  This has the form:

\egindent{\tt for {\em lhs\/} in {\em Expr\/} | {\em condition}}

\vbox{
This kind of \ttin{for} iterator skips those values which do not
satisfy the condition.  For example, the \verb"sum" example we saw
earlier could have been written as:

\begin{small}
\begin{verbatim}
    #include "aldor"

    -- Add up the odd numbers in a range
    sum(lo: Integer, hi: Integer): Integer == {
            n := 0;
            for i in lo..hi | odd? i repeat n := n + i;
            n
    }
\end{verbatim}
\end{small}

%%%%%%%%%%%%%%%%%%%%%%%%%%%%%%%%%%%%%%%%%%%%%%%%%%%%%%%%%%%%%%%%%%%%%%%%%%%%%%%
\subsubsection{Multiple iterators}
%%%%%%%%%%%%%%%%%%%%%%%%%%%%%%%%%%%%%%%%%%%%%%%%%%%%%%%%%%%%%%%%%%%%%%%%%%%%%%%

A \ttin{repeat} loop may have any number of iterators, with the
following syntax:

\verb^    ^{\em $iterator _{1}$ ... $iterator _{n}$} \verb^ repeat ^ {\em
Body}

The loop is repeated until one of the iterators terminates it:
the first \ttin{while} which has a false condition or the
first \ttin{for} which consumes all its values ends the loop.

This is convenient when the termination condition does not relate
directly to a \ttin{for} variable, or when structures are to be
traversed in parallel.

Continuing the example on page \pageref{asugLangForExample}, we may use
a \ttin{while} to decide if the end has been reached, and, at the same
time, use a \ttin{for} to count the number of times we have evaluated
the loop body:

\begin{small}
\begin{verbatim}
    #include "aldor"
    #include "aldorio"

    import from Integer;

    n := 10000;
    k := 0;

    while n ~= 1 for free k in 1.. repeat
            if odd? n then n := 3*n + 1 else n := n quo 2;

    stdout << "Terminated after " << k << " iterations." << newline;
\end{verbatim}
\end{small}

Multiple \ttin{for} iterators can allow lists to be combined in an
efficient way:

\begin{small}
\begin{verbatim}
    #include "aldor"
    #include "aldorio"

    import from List Integer, Integer;

    l1 := [1,2,3];
    l2 := [8,7,6,5];

    -- Add the pair-wise products in two lists.
    x := 0;
    for n1 in l1 for n2 in l2 repeat
            x := x + n1 * n2;

    stdout << "The result is " << x << newline
\end{verbatim}
\end{small}
}
%\pagebreak

These two \ttin{for} iterators are used in parallel, like a zipper
combining the two lists.  The loop stops at the end of the shortest
list, in this case giving a sum of three products.  

This is not a double loop -- to use {\em all} pairs with the number
from \verb"l1" and the second number from \verb"l2",
you would use two nested loops, each with its own \ttin{repeat}:

\begin{small}
\begin{verbatim}
    #include "aldor"
    #include "aldorio"

    import from List Integer, Integer;

    l1 := [1,2,3];
    l2 := [8,7,6,5];

    x := 0;
    for n1 in l1 repeat
            for n2 in l2 repeat
                    x := x + n1 * n2;

    stdout << "The result is " << x << newline
\end{verbatim}
\end{small}

Using more than one iterator is often the most efficient natural way to write
a loop.  A loop with two \ttin{for} iterators is more efficient than

\begin{small}
\begin{verbatim}
    ...
    m: MachineInteger := min(#l1, #l2);
    for i in 1..m repeat
            x := x + l1.i * l2.i
\end{verbatim}
\end{small}

because it does not need to traverse the lists each time to pick off
the desired elements.  And the code is more concise than

\begin{small}
\begin{verbatim}
    ...
    t1 := l1;
    t2 := l2;
    while (not empty? t1 and not empty? t2) repeat {
            x := x + first t1 * first t2;
            t1 := rest t1;
            t2 := rest t2;
    }
\end{verbatim}
\end{small}


\head{subsubsection}{Break}{asugLangBreak}
\keywordIndex{break}

Evaluating a \ttin{break} causes a loop to terminate.  For
example:

\begin{small}
\begin{verbatim}
    #include "aldor"
    #include "aldorio"

    import from Integer;

    n := 10000;
    k := 0;         -- `k' counts the number of iterations.

\end{verbatim}
\end{small}
\vbox{
\begin{small}
\begin{verbatim}
    repeat {
            if odd? n then n := 3*n + 1 else n := n quo 2;

            k := k + 1;

            if n = 1 then break;
    }

    stdout << "Terminated after " << k << " iterations." << newline;
\end{verbatim}
\end{small}

\ttin{break} is most useful when the condition which quits the loop
falls most naturally in the middle or at the end of the loop body.
Sometimes it is possible to change a test which appears at the end
to a test at the beginning.  This helps make programs more readable,
since one sees immediately what will cause the loop to end.
Also, it allows the exit to be controlled by a \ttin{while} iterator,
rather than an \ttin{if} and a \ttin{break}.

If a \ttin{break} occurs inside nested loops, it terminates the deepest
one.


\head{subsubsection}{Iterate}{asugLangIterate}
\keywordIndex{iterate}

Evaluating an \ttin{iterate} abandons the current evaluation of the loop
body and starts the next iteration. For example:

\begin{small}
\begin{verbatim}
    #include "aldor"
    #include "aldorio"

    import from Integer;

    n := 10000;
    k := 0;         -- `k' counts the number of iterations.

    repeat {
            k := k + 1;
            if odd? n then { n := 3*n + 1; iterate }

            n := n quo 2;
            if n = 1 then break;
    }

    stdout << "Terminated after " << k << " iterations." << newline;
\end{verbatim}
\end{small}

This example does the same thing as the previous one, but is organised
slightly differently.  The \ttin{iterate} on the second line of the
loop body causes the rest of the body to be skipped.

\ttin{iterate} can be used instead of placing the remainder
of a loop body inside an \ttin{if} expression. 
This can make programs easier to read, by emphasising that certain
conditions are not expected, and by avoiding extra levels
of indentation.
It is particularly useful when the decision to go on to the next iteration
of a loop is buried deep inside some other logic, rather than appearing
at the top level of the loop body.

An \ttin{iterate} is equivalent to a \ttin{goto} branching to the end of
the loop body.  Thus, the meaning of \ttin{iterate} is independent
of whether there are any \ttin{while} or \ttin{for} iterators
controlling the loop.

If an \ttin{iterate} occurs inside nested loops,
it steps the deepest one.
}

\subsubsection{Definition in low-level terms}

It is possible to express the loop behaviour in terms of
\verb"goto"s and labels (see section~\ref{asugLangExprGotos} for 
details).  A loop of the form

\egindent{\tt $it_1$  $it_2$ ... $it_n$ repeat $Body$}

is equivalent to 

\verb"    {" \\
\verb"        "$init_1$; $init_2$; ... $init_n$; \\
\verb"        @TOP" \\
\verb"        "$step_1$; $step_2$; ... $step_n$; \\
\verb"        "$Body$; \\
\verb"        "\verb"goto TOP;" \\
\verb"        @DONE" \\
\verb"    }"

Where,
%\parbox[t]{100mm}{
%\begin{description}\vspace{-6.9mm}
%\item[]
if $it_i$ is ``{\tt while $cond_i$}'', then $init_i$ is empty
and $step_i$ is 

\egindent{\tt if not $cond_i$ then goto DONE}

%\item[]
if $it_i$ is ``{\tt for $lhs_i$ in $expr_i \mid cond_i$}'',
then $init_i$ is

\egindent{\tt $g_i$ := generator $expr_i$;}

\hspace{4mm}and $step_i$ is

\hspace{8mm}\parbox{60mm}
{
{\tt step!} $\!g_i${\tt ;}\\
{\tt if empty?} $g_i$ {\tt then goto DONE}\\
{\tt $lhs_i$ := value $g_i$}\\
{\tt if not $cond_i$ then goto TOP}
}

\hspace{4mm}(a \ttin{for} without a condition is treated as if it had 
the condition \ttin{true}).
%\end{description}}


In this, \verb"TOP", \verb"DONE" and the $g_i$ are generated
names, and are not accessible to the other parts of the program.

%%%%%%%%%%%%%%%%%%%%%%%%%%%%%%%%%%%%%%%%%%%%%%%%%%%%%%%%%%%%%%%%%%%%%%


%%%%%%%%%%%%%%%%%%%%%%%%%%%%%%%%%%%%%%%%%%%%%%%%%%%%%%%%%%%%%%%%%%%%%%%%%%%%%%
\head{section}{Generate expressions}{asugLangExprGenerators}
%%%%%%%%%%%%%%%%%%%%%%%%%%%%%%%%%%%%%%%%%%%%%%%%%%%%%%%%%%%%%%%%%%%%%%%%%%%%%%

Generate expressions are used to create generators.
For a complete description of how to use {\tt generate} to create
a generator, see \chapref{asugLangGener}.  The general syntax for a
generate expression is:

\begin{small}
\begin{verbatim}
    generate E
    generate to n of E
\end{verbatim}
\end{small}

\keywordIndex{of}
\keywordIndex{yield}
\keywordIndex{to}

{\tt E} is an expression which represents code which will be evaluated
each time a value is extracted from the generator.  Evaluation begins
at the start of the expression {\tt E} and continues until an expression
of the form

\begin{small}
\begin{verbatim}
    yield v
\end{verbatim}
\end{small}

is processed, where {\tt v} is the value to be passed back to the
context which is collecting values from the generator.  Each time a
value is requested after the first value is yielded, control resumes
after the {\tt yield} expression which produced the previous value.

%%%%%%%%%%%%%%%%%%%%%%%%%%%%%%%%%%%%%%%%%%%%%%%%%%%%%%%%%%%%%%%%%%%%%%%%%%%%%%
\head{section}{Collections}{asugLangExprCollections}
%%%%%%%%%%%%%%%%%%%%%%%%%%%%%%%%%%%%%%%%%%%%%%%%%%%%%%%%%%%%%%%%%%%%%%%%%%%%%%

\index{collect expressions}

A {\em collect} expression provides a convenient shorthand for
creating generators and aggregate objects.  A collect expression
has the form:

\verb@   @{\tt $E$\ ${iter}_1\ \ldots\ {iter}_n$}

where $n \ge 1$. The ${iter}_i$ are iterators, as described in
\secref{asugLangExprLoop}.  Consider the following example:

\begin{small}
\begin{verbatim}
    #include "aldor"
    #include "aldorio"
    ...
    import from Integer, List Integer;
    stdout << [x*x for x in 1..10] << newline;
    stdout << [x*y for x in 1..10 for y in 10..1 by -1] << newline;
\end{verbatim}
\end{small}

This program creates a list of the squares of the integers from 1 to 10,
and then a list of products of integers.  Note that {\tt while} can form
an iterator, and can therefore be used in a collect expression.

\begin{sloppypar}
Note that the square brackets are not part of the collect expression,
but are simply a shorthand for a call to the function \ttin{bracket},
with the value of the collection as an argument.  The domain
\ttin{List Integer} from the \asharp{} base library exports a
function with the signature
\verb"bracket: Generator Integer -> %", which is called twice in the
above example.
\end{sloppypar}

As collect expressions produce generators, one would expect that
generate expressions and collect expressions are related.  The collect
expression:

\verb@   @{\tt $E$\ ${iter}_1\ \ldots\ {iter}_n$}

is equivalent to the generate expression:

\verb@   @{\tt generate ${iter}_1\ \ldots\ {iter}_n$\ repeat yield $E$}

Thus, {\tt x*y for x in 1..9 for y in 9..1 by -1} is the same as:

{\tt generate for x in 1..9 for y in 9..1 by -1 repeat yield x*y}

Collect expressions provide a convenient notation for creating new aggregates,
and require no additional functionality in the language.

%%%%%%%%%%%%%%%%%%%%%%%%%%%%%%%%%%%%%%%%%%%%%%%%%%%%%%%%%%%%%%%%%%%%%%%%%%%%%%
\head{section}{General branching}{asugLangExprGotos}
%%%%%%%%%%%%%%%%%%%%%%%%%%%%%%%%%%%%%%%%%%%%%%%%%%%%%%%%%%%%%%%%%%%%%%%%%%%%%%

\keywordIndex{label}
\asharp{} provides unconditional branching using {\em label expressions\/}
and {\em goto expressions\/}.  A label expression is of the following form:

\begin{small}
\begin{verbatim}
    @L E
\end{verbatim}
\end{small}

where \verb"L" is an identifier known as the {\em label name},
and \verb"E" is any expression.  The type of the label expression
is the same as the type of \verb"E".  Names which are used as labels
have no relationship with variables or constants of the same name,
and a label name may also be used as a variable or constant. A label
name obeys the same scope rules as constants or variables, but may not
appear in any expressions other than {\tt goto}s and other label
expressions. Since labels are constants, it is an error to bind the
same label twice in the same scope.

\keywordIndex{goto}
A goto expression has the following form:

\begin{small}
\begin{verbatim}
    goto L
\end{verbatim}
\end{small}

where \verb"L" is the name of a label.
After the evaluation of a goto expression, execution resumes
with the expression associated with the label \verb"L".
The type of the goto expression itself is the type \verb"Exit".
\index{Exit}

%The label must be in the same scope as the \verb"goto"
% (\ie{} in the same function or \verb"where" expression).
%Also, a \verb"goto" must not branch into a loop with \ttin{for}
%iterators. 

The label must appear in the same function body as the \verb"goto". In
addition, it is an error for a \verb"goto" to branch into an inner scope
of the scope in which it appears (that is, % or did "ie." mean "e.g."?
a local function or any repeat or
collection including a {\tt for}, {\tt where}, {\tt add} or {\tt with}
expression). A \verb"goto" may also not branch out of a function or a
{\tt with} or {\tt add} expression.

Example:

\begin{small}
\begin{verbatim}
    foo(a: MachineInteger): MachineInteger == {
        if a <= 0 then goto ERROR;

        return a * a;

        @ERROR
        stdout << "The argument must be a positive value!" << newline;
        0
    }
\end{verbatim}
\end{small}

If the first test is successful, then the \ttin{return} expression
is skipped and the execution proceeds on the line following \ttin{\@ERROR}.

Labels can also be defined at the top level of a file,
since the top level of a file is treated as a sequence:

\begin{small}
\begin{verbatim}
    #include "aldor"
    #include "aldorio"

    @LAB1
    stdout << "You will see this forever..." << newline;
    goto LAB1;
\end{verbatim}
\end{small}

%%%%%%%%%%%%%%%%%%%%%%%%%%%%%%%%%%%%%%%%%%%%%%%%%%%%%%%%%%%%%%%%%%%%%%%%%%%%%%
\head{section}{Never}{asugLangExprNever}
%%%%%%%%%%%%%%%%%%%%%%%%%%%%%%%%%%%%%%%%%%%%%%%%%%%%%%%%%%%%%%%%%%%%%%%%%%%%%%

\keywordIndex{never}
\index{Exit}
The expression \ttin{never} is a special value, of type \verb"Exit",
which acts as a programmer-supplied assertion that execution will never
reach that point.  An exit expression can be useful
as it may allow a program to be translated into more efficient code.

The following \asharp{} code is a possible use of {\tt never}

\begin{small}\begin{verbatim}
        s := {
                x = 0 => "zero";
                x > 0 => "positive";
                x < 0 => "negative";
                -- This expression is unreachable
                never
        }
\end{verbatim}
\end{small}

With luck, the \ttin{never} at the end of the sequence will not be
reached in any execution of the program.  (If it is reached, \asharp{}
will complain \linebreak \ttin{Aldor error: Reached a "never"}).
