% *********************************************************************
\head{chapter}{Installation}{asugInstall}
% *********************************************************************

In this chapter we tell you how to install the tester's subset
and the full internal development version of the compiler.
\index{installation}

% *********************************************************************
\head{section}{Directories and environment variables}{asugInstallDir}
% *********************************************************************

The installer of the compiler must decide on a main directory in
which to place the package.
This main directory will contain the files necessary to run the
compiler on a particular platform plus, if the internal
developer's version is also being installed, the source code for
one or more versions of the compiler.
If you are using the compiler under Unix, we recommend you
use ``{\tt /usr/\asharpcmd{}}'' for the main directory.

Each user must set the {\tt ALDORROOT}
\index{environment variables!ALDORROOT@\protect{\tt ALDORROOT}}
environment
\index{ALDORROOT@\protect{\tt ALDORROOT} environment variable}
variable\footnote{In a later version, we will recommend that the main
directory be {\tt /usr} and it will not be necessary for each user
to set ALDORROOT.}.
If you are running AIX 3.2 on an IBM RS/6000, we currently recommend
you use ``{\tt /usr/\asharpcmd{}/base/rs3.2}''.
Each user must also modify the {\tt PATH} environment variable to
include ``{\tt \$ALDORROOT/bin}''.
\index{environment variables!PATH@\protect{\tt PATH}}
\index{PATH@\protect{\tt PATH} environment variable}

For example, if you are using the Korn shell, you would place
the following commands in  your {\tt .profile} file.

{\tt export ALDORROOT=/usr/\asharpcmd{}/base/rs3.2}

{\tt export PATH=\$PATH:\$ALDORROOT/bin}

Each user can then access these subdirectories of
the {\tt \$ALDORROOT} directory:

\begin{tabular}{@{}lp{4.09in}@{}}
\tt bin & This directory contains the executables {\tt \asharpcmd{}}
(the \asharp{} compiler) and {\tt mailbug} (a tool for reporting
problems).
In the internal developer's version, it also contains additional
tools for building the compiler.
\index{directories!bin@\protect{\tt bin}}
\\
\tt lib & This directory contains compiled \asharp{} libraries
and the library needed for run-time support of stand-alone programs.
In the internal developer's version, it also contains additional
object files or libraries for building the compiler.
\index{directories!lib@\protect{\tt lib}}
\\
\tt include & This directory contains the C include file for
creating stand-alone programs and the basic \asharp{} include files.
\index{directories!include@\protect{\tt include}}
\\
\tt samples & This directory contains sample \asharp{} programs.
\index{directories!samples@\protect{\tt samples}}
\end{tabular}

If you are using a fileserver, there may be present
different versions of the compiler for different machines.
For example, the directories
\begin{verbatim}
/usr/aldor/base/rs3.1
/usr/aldor/base/rs3.2
/usr/aldor/base/sungcc
\end{verbatim}
may all exist.

The installer of the compiler must also decide on a version
\index{directories!compiler version}
directory if the compiler source code is also being installed.
If the main directory is ``{\tt /usr/\asharpcmd{}}'' and the version
level is ``\asharpver{}'', it is common to use
``{\tt /usr/\asharpcmd{}/version/\asharpver{}}'' as the version
directory.
Note, however, that the source directory may be placed anywhere in
the file system and is not required to be in any particular
position relative to {\tt \$ALDORROOT}.
Only installers and internal compiler developers need to be aware
of the exact version directory.
No environment variable needs to be set.

The following diagram shows the above suggested directory
structure, with {\tt \$ALDORROOT} set to ``{\tt /usr/\asharpcmd{}/base/rs3.2}''.

%		treedef.tex
%
%	These definitions for tree macros are taken from "Trees in TeX",
%	by David Eppstein, as published in TUGboat 6#1, March 1985.
%	David Eppstein's address (as of 15 June 1988) is
%		Computer Science Department
%		Columbia University
%		New York, NY 10027
%		Eppstein@cs.Columbia.edu

\newbox\treebox
\def\tree{\global\setbox\treebox=\boxtree}
\def\subtree{\ettext \boxtree}
\def\leaf#1{\subtree#1\endsubtree}

\def\endsubtree{\ettext \egroup}
\def\endtree{\endsubtree \settreesizes \typesettree}

\newif\iftreetext\treetextfalse         % Whether still aligning text
\def\boxtree{\hbox\bgroup               % Start outer box of tree or subtree
  \baselineskip 2.5ex                   % Narrow line spacing slightly
  \tabskip 0pt                          % No spurious glue in alignment
  \vbox\bgroup                          % Start inner text \vbox
  \treetexttrue                         % Remember for \ettext
  \let\par\crcr \obeylines              % New line breaks without explicit \cr
  \halign\bgroup##\hfil\cr}             % Start alignment with simple template
\def\ettext{\iftreetext                 % Are we still in inner text \vbox?
  \crcr\egroup \egroup \fi}             % Yes, end alignment and box

\def\cons#1#2{\edef#2{\xmark #1#2}}     % Add something to start of list.
\def\car#1{\expandafter\docar#1\docar}  % Take first element of list
\def\docar\xmark#1\xmark#2\docar{#1}    % ..by ignoring rest in expansion.
\def\cdr#1{\expandafter\docdr#1\docdr#1} % Similarly, drop first element.
\def\docdr\xmark#1\xmark#2\docdr#3{\def#3{\xmark #2}}
\def\xmark{\noexpand\xmark}             % List separator expands to self.
\def\nil{\xmark}                        % Empty list is just a separator.

\def\settreesizes{\setbox0=\copy\treebox \global\let\treesizes\nil \setsizes}
\newdimen\treewidth                     % Width of this part of the tree.
\def\setsizes{\setbox0=\hbox\bgroup     % Get a horiz list as a workspace.
  \unhbox0\unskip                       % Take tree, unpack it into horiz list.
  \inittreewidth                        % Get old width at this level.
  \sizesubtrees                         % Recurse through all subtrees.
  \sizelevel                            % Now set width from remaining \vbox.
  \egroup}                              % All done, finish our \hbox.

\def\inittreewidth{\ifx\treesizes\nil   % If this is the first at this level
    \treewidth=0pt                      % ..then we have no previous max width.
  \else \treewidth=\car\treesizes       % Otherwise take old max level width
    \global\cdr\treesizes               % ..and advance level width storage
    \fi}                                % ..in preparation for next level.

\def\sizesubtrees{\loop                 % For each box in horiz list (subtree)
  \setbox0=\lastbox \unskip             % ..pull it off list and flush glue.
  \ifhbox0 \setsizes                    % If hbox, it's a subtree - recurse
  \repeat}                              % ..and loop; end loop on tree text.

\def\sizelevel{\ifdim\treewidth<\wd0    % If greater than previous maximum
   \treewidth=\wd0 \fi                  % Then set max to new high
 \global\cons{\the\treewidth}\treesizes}% In either case, put back on list

\newdimen\treeheight                    % Height of this part of the tree.
\newif\ifleaf                           % Tree has no subtrees (is a leaf).
\newif\ifbotsub                         % Bottom subtree of parent.
\newif\iftopsub                         % Top subtree of parent.
\def\typesettree{\medskip \maketree \medskip}   % Make whole tree with spacing.
\def\maketree{\hbox{\treewidth=\car\treesizes   % Get width at this level.
  \cdr\treesizes                        % Set up width list for recursion.
  \makesubtreebox\unskip                % Set \treebox to text, make subtrees.
  \ifleaf \makeleaf                     % No subtrees, add glue.
  \else \makeparent \fi}}               % Have subtrees, stick them at right.

{\catcode`@=11                          % Be able to use \voidb@x.
\gdef\makesubtreebox{\unhbox\treebox    % Open up tree or subtree.
  \unskip\global\setbox\treebox\lastbox % Pick up very last box.
  \ifvbox\treebox                       % If we're already at the \vbox
    \global\leaftrue \let\next\relax    % ..then this is a leaf.
  \else \botsubtrue                     % Otherwise, we have subtrees.
    \setbox0\box\voidb@x                % Init stack of processed subs
    \botsubtrue \let\next\makesubtree   % ..and call \maketree on them.
  \fi \next}}                           % Finish up for whichever it was.

\def\makesubtree{\setbox1\maketree      % Call \maketree on this subtree.
  \unskip\global\setbox\treebox\lastbox % Pick up box before it.
  \treeheight=\ht1                      % Get height of subtree we made
  \advance\treeheight 2ex               % Add some room around the edges
  \ifhbox\treebox \topsubfalse          % If picked up box is a \vbox,
    \else \topsubtrue \fi               % ..this is the top, otherwise not.
  \addsubtreebox                        % Stack subtree with the rest.
  \iftopsub \global\leaffalse           % If top, remember not a leaf
    \let\next\relax \else               % ..(after recursion), set return.
    \botsubfalse \let\next\makesubtree  % Otherwise, we have more subtrees.
  \fi \next}                            % Do tail recursion or return.

\def\addsubtreebox{\setbox0=\vbox{\subtreebox\unvbox0}}
\def\subtreebox{\hbox\bgroup            % Start \hbox of tree and lines
  \vbox to \treeheight\bgroup           % Start \vbox for vertical rules.
    \ifbotsub \iftopsub \vfil           % If both bottom and top subtree
        \hrule width 0.4pt              % ..vertical rule is just a dot.
      \else \treehalfrule \fi \vfil     % Bottom gets half-height rule.
    \else \iftopsub \vfil \treehalfrule % Top gets half-height the other way.
      \else \hrule width 0.4pt height \treeheight \fi\fi % Middle, full height.
    \egroup                             % Finish vertical rule \vbox.
  \treectrbox{\hrule width 1em}\hskip 0.2em\treectrbox{\box1}\egroup}

\def\treectrbox#1{\vbox to \treeheight{\vfil #1\vfil}}
\def\treehalfrule{\dimen0=\treeheight   % Get total height.
  \divide\dimen0 2\advance\dimen0 0.2pt % Divide by two, add half horiz height.
  \hrule width 0.4pt height \dimen0}    % Make a vertical rule that high.

\def\makeleaf{\box\treebox}             % Add leaf box to horiz list.
\def\makeparent{\ifdim\ht\treebox>\ht0  % If text is higher than subtrees
    \treeheight=\ht\treebox             % ..use that height.
  \else \treeheight=\ht0 \fi            % Otherwise use height of subtrees.
  \advance\treewidth-\wd\treebox        % Take remainder of level width
  \advance\treewidth 1em                % ..after accounting for text and glue.
  \treectrbox{\box\treebox}\hskip 0.2em % Add text, space before connection.
  \treectrbox{\hrule width \treewidth}\treectrbox{\box0}} % Add \hrule, subs.

\endinput


\hfill\hbox{\tt%
\tree
  /usr/aldor

  \subtree
    base
    \subtree
      rs3.2
      \leaf{bin}
      \leaf{include}
      \leaf{lib}
      \leaf{samples}
    \endsubtree
  \endsubtree

  \subtree
    version
    \subtree
      \asharpver{}
      \leaf{src}
      \leaf{aldor}
      \leaf{test}
      \leaf{\ldots}
    \endsubtree
  \endsubtree
\endtree
}\hfill\break


% *********************************************************************
\head{section}{Installing from diskettes}{asugInstallDisk}
% *********************************************************************

This section describes installing the compiler package on an IBM RS/6000.
\index{installation!from diskettes}
The package has the following configuration when shipped
on diskettes for the IBM RS/6000 running AIX: there are four
diskettes, each containing one compressed tar file (with file
extension {\tt .tz}) and one file listing file (with file extension
{\tt .lst}).
\index{diskettes}
In addition, the first diskette contains a program ({\tt
aldor.ins}) that does the installation.
The diskettes are written using {\tt doswrite} and are read by {\tt dosread.}

\begin{tabular}{@{}l|l@{}}
\bf Disks &  \bf Contents \\ \hline \\
1, 2     &  {\it Installation:\/} executable programs for AIX, \\
         &  {\tt lib} and {\tt include} subdirectories of {\tt \$ALDORROOT} \\
         &  and sample programs. \\ \\
3        &  {\it Regression tests\/} and their output. \\
4        &  {\it Source} for compiler, tools and libraries. \\
\end{tabular}

Disks 1 and 2 are installed relative to the compiler main
directory.
Disks 3 and 4 are installed relative to the compiler version
directory.

Do the following to install the compiler.
\begin{enumerate}
\item Use {\tt su}, if necessary, to get the appropriate
permissions for creating the compiler directories.
%
\item Create the compiler main and version directories with the
names discussed in \spadref{asugInstallDir}.
We now assume

\begin{tabular}{@{\qquad}lcl}
main directory & =  & {\tt /usr/\asharpcmd{}} \\
version directory & = & {\tt /usr/\asharpcmd{}/version/\asharpver{}}.
\end{tabular}

Update your {\tt PATH} and {\tt ALDORROOT} environment variables in
your profile file (that is, {\tt .profile} for {\tt ksh} users,
and {\tt .cshrc} for {\tt csh} users.
%
\item Issue {\tt cd /tmp}
\item Insert diskette 1 into the diskette drive.
\item Issue {\tt dosread natural.ins > natural.ins}
\item Issue {\tt sh natural.ins}
\item Respond to the prompts.
If you are installing the tester's version of the compiler,
respond ``{\tt t}'' when you are asked about the version.
If you are installing the internal developer's version
respond ``{\tt d}''.
\item Issue {\tt rm natural.ins}
\item If you used {\tt su} in step 1,
enter the Ctrl-D key combination to the shell.
\end{enumerate}

Once your environment variable updates take effect, you can use
the compiler as we discuss in \chapref{asugUsing}.

