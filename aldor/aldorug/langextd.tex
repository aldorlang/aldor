%%%%%%%%%%%%%%%%%%%%%%%%%%%%%%%%%%%%%%%%%%%%%%%%%%%%%%%%%%%%%%%%%%%%%%%%%%%%%%
%%%%%%%%%%%%%%%%%%%%%%%%%%%%%%%%%%%%%%%%%%%%%%%%%%%%%%%%%%%%%%%%%%%%%%%%%%%%%%
%%%                                                                        %%%
%%% :: Post Facto Extensions
%%%                                                                        %%%
%%%%%%%%%%%%%%%%%%%%%%%%%%%%%%%%%%%%%%%%%%%%%%%%%%%%%%%%%%%%%%%%%%%%%%%%%%%%%%
%%%%%%%%%%%%%%%%%%%%%%%%%%%%%%%%%%%%%%%%%%%%%%%%%%%%%%%%%%%%%%%%%%%%%%%%%%%%%%
\keywordIndex{extend}
\head{chapter}{Post facto extensions}{asugLangExtension}

One of the real problems in reusing code is
how to use programs from independently developed libraries at the same time.

Every library has assumptions about the basic behaviour of the values it uses.
To use values created by functions of one library as input to 
functions of a second library, it is usually necessary
to convert the values to types derived from the second library.

As the number of libraries used by an application increases, the conversion
problem becomes more significant.  
%
For example, say a library defines a concept of
\ttin{FancyOutput}, of which the basic {\tt Integer} type could be an instance,
if only it supported a few extra operations.
The way this would be handled in other languages would be to derive a
new type from {\tt Integer}, say \ttin{FancyOutputInteger}, and to use
fancy integers in the program.

Now suppose we wish to use a library for differential operators,
which defines a concept of \ttin{DifferentialRing}.
Now the integers would be a suitable instance, if only they provided
a differentiation operator.  It is a trivial matter to supply a
differentiation operator for the set of integers:  it would just return 0.
At this point we end up with the new type
\ttin{DifferentiableFancyOutputInteger} which may now be used in place of 
\ttin{Integer} when using the differential operator library at the
same time as fancy output. 

Aside from being ugly,  this leads to a second problem:
Each library may refer to the original type Integer, or its own
derived version of it, in particular situations, and the 
{\tt DifferentiableFancyInteger}s will need to be converted to {\tt
FancyInteger}s or {\tt Integer}s and {\em vice versa}.

This problem gets worse when both libraries use the same base type,
{\tt String} or {\tt Window} for instance, with different sets of operations.

This problem is particularly acute in certain applications. 
In mathematics, for example, certain basic sets are simple instances
of very many abstract concepts.  It is impossible for the \asharp{}
library to anticipate all the mathematical contexts in which the
type {\tt Integer} may be expected to perform.

The way \asharp{} solves the problem just described is to 
allow programs to supply enhancements to already existing types via
{\em post facto extensions}.

%%%%%%%%%%%%%%%%%%%%%%%%%%%%%%%%%%%%%%%%%%%%%%%%%%%%%%%%%%%%%%%%%%%%%%%%%%%%%%
\head{section}{Extending types}{asugLangExtensionTypes}

If $T$ is a type-valued constant, then its meaning may be extended with 
a definition of the form:

\verb"    "{\tt extend {\em T\/}:~{\em C} == {\em D} }

Here $C$ is a new category to which $T$ will now belong and $D$ is
a package providing implementations for the newly required exports.
The new operations cannot see the internal representation of $T$.

Taking the earlier example, the type {\tt Integer} can be made to satisfy
the new categories {\tt FancyOutput} and {\tt DifferentialRing}
by providing appropriate extensions:

\begin{small}
\begin{verbatim}
extend Integer: FancyOutput == add {
        box(n: Integer): BoundingBox == [1, ndigits n, 0, 0]
}
\end{verbatim}
\end{small}

\begin{small}
\begin{verbatim}
extend Integer: DifferentialRing == add {
        differentiate(n: Integer): Integer == 0
}
\end{verbatim}
\end{small}

then these extensions may be used, independently or together,

Extensions are made visible in the same manner as other definitions:
by importing the package which provides them.  
This allows programs using extensions to be statically checked
and more heavily optimised.

Extensions may be private to a particular application, or 
be exported by a library.  
In the case where a library provides an
extension, the clients of the library see the extension applied to the type.

Within a given context,
a type-valued constant has as its type the \verb"Join" of the
categories of the original value all the extensions.  
Its value is obtained by \verb"add"-ing the extensions to the original value.

If two extensions have intersecting categories, then they may not be
imported in the same scope.
%%MCD: What does this mean?
%The extensions might, however, be combined explicitly.  
%The current implementation does not check for this error.

%%%%%%%%%%%%%%%%%%%%%%%%%%%%%%%%%%%%%%%%%%%%%%%%%%%%%%%%%%%%%%%%%%%%%%%%%%%%%%
%\newpage
\section{Extending functions}

In typical \asharp{} applications many of the types are provided
by functions, so an effective extension mechanism must provide some way
to extend these types as well.

One might imagine a solution which took a particular function result, say 
\ttin{List(Integer)} and modified it.  
The problem with this is that while that one list type would be extended,
the other list types would not be.  
It is really the meaning of \verb"List" which should be extended.

To extend the meaning of a function-valued constant in \asharp{}
one writes:

\verb"    "{\tt extend {\em fdef}}

where ``{\em fdef\/}'' has the form of a definition of the function 
being extended.  For example:

\begin{small}
\begin{verbatim}
   #include "aldor"
   #include "aldorio"

   -- Partial(S) is a type which includes values in `S' as well
   -- as a special `failed' value.
   --
   -- This extension does two things:
   -- 1. It allows `failed' to be treated as `false' in `if' statements.
   -- 2. It causes an import from Partial(S) to also import from S.

   extend Partial(S: Type): with {
           test: % -> Boolean;

           export from S;
   }
   == add {
           test(x: %): Boolean == not failed? x
   }

   PI ==> Partial Integer;
   import from PI;

   i := failed; j := [7];

   stdout << (if i then "oops" else "ok") << newline;
   stdout << (if j then "ok" else "oops") << newline;
\end{verbatim}
\end{small}

Any form of function definition is allowed, \eg{}:
\begin{small}
\begin{verbatim}

    extend F(S: Type): C(S) == ...;

    extend G: (S: Type) -> C(S)  ==  (S: Type): C(S) +-> ...;

    extend H(n: Integer, S: PrimitiveType) (R: ArithmeticType) == ....;
\end{verbatim}
\end{small}

%\newpage
Function extensions are meaningful when the argument and result types of
the extension are compatible with the argument and result types
of the original meaning.  

In the current implementation of the language, the extension
argument types are compatible if they are the same as
the argument types of the orginal function.

Corresponding result types are compatible if they are the same
or both are subtypes of the same type.
\index{subtype}
Additionally, the (base) type must have an appropriate method to combine values.
At present only functions and types provide a combination method.
%although a later version of the language may provide a generic interface.

In practice this means that types, or functions producing types, 
or functions producing functions producing types, \etc{} may be extended.

Type values are combined by \verb"add" and their types are combined
with \verb"Join".
Function values are combined by producing a new function which
runs all of them and combines the result (recursively, if necessary).

%%%%%%%%%%%%%%%%%%%%%%%%%%%%%%%%%%%%%%%%%%%%%%%%%%%%%%%%%%%%%%%%%%%%%%%%%%%%%%
%Corresponding arguments are compatible if their types are the same
%or both subtypes of the same type. For example,
%\begin{small}
%\begin{tt}
%    (Integer, Ring) -> ...
%    (Integer, Ordered) -> ...
%\end{tt}
%\end{small}
%Here the first arguments have the same type and the second arguments
%are both subtypes of {\tt Type}.
%
%If the original and extension types of a function are
%
%\def\nnS{s}
%\def\nnT{t}
%\def\nnX{x}
%\hspace{0in}
%\begin{displaymath}
%\begin{array}{c}
%(S_{0 1}, ... , S_{0 \nnS}) \rightarrow (T_{0 1}, ... , T_{0 {\nnT}}) \\
%(S_{1 1}, ... , S_{1 \nnS}) \rightarrow (T_{1 1}, ... , T_{1 {\nnT}}) \\
%\vdots \\
%(S_{{\nnX} 1}, ... , S_{{\nnX} \nnS}) \rightarrow (T_{{\nnX} 1}, ... , T_{{\nnX} {\nnT}})
%\end{array}
%\end{displaymath}
%
%then the type of the resulting extended function is
%
%\hspace{0in}
%\begin{displaymath}
%\begin{array}{c}
%(a_1 : SX_{1}, ... , a_n : SX_{\nnS}) \rightarrow
% (TX_{1} (a_1,...,a_{\nnS}), ... , TX_{{\nnT}} (a_1,...,a_{\nnS}))
%\end{array}
%\end{displaymath}
%
%where
%
%\hspace{0in}
%\begin{displaymath}
%\begin{array}{l}
%SX_j = 
%  {\tt Meet}_{i = 0}^{\nnX} \; S_{i j} \\
%\ \\
%TX_j (a_1,...,a_{\nnS}) = 
%  {\tt Join}_{i = 0}^{\nnX} \; \left\{
%    {\tt if} \; (\, {\tt and}_{k=1}^{\nnS} \, a_k \, {\tt has} \, S_{i k} \,) \;
%    {\tt then} \; T_{i j}
%  \right\}
%\end{array}
%
%Here, {\tt Meet} is the subtype consisting of all of the members of
%its arguments, which must have a common base type.  If there
%is no suitable {\tt Meet} operation for that type, then the 
%argument types in that position must be the same.
%%%%%%%%%%%%%%%%%%%%%%%%%%%%%%%%%%%%%%%%%%%%%%%%%%%%%%%%%%%%%%%%%%%%%%%%%%%%%%

\section{Extending the base \asharp{} library}

The base \asharp{} library builds many of its types in layers, 
{\hyphenation{using using} using} \ttin{extend}.
For instance, it extends
\verb"Boolean", \verb"Generator(S)" and \verb"Tuple(S)",
which are language-defined,
and creates \verb"Integer" and \verb"TextWriter" in stages.
%These examples may be found in the \fname{libaldor} subdirectory of the
%\asharp{} \fname{samples} directory.  (Please consult your systems
%administrator to determine the location of this directory.)
%% Previous text was Unix specific.

An application which introduces a new concept can provide a file
to extend an existing library to support the new concept.

Suppose, for instance, that an application needed to determine how much
dynamically allocated memory was used by data structures.
Then it could extend the \asharp{} library with a file which began
with code resembling that in Figure~\ref{FigureExtend}.

With this, it would now be possible to ask about the sizes of 
values belonging to the types \ttin{Integer}, \ttin{List(Integer)}, and
so on.

%%AXIOM \section{Extending the \axiom{} library}
%%AXIOM 
%%AXIOM The \axiom{} library bindings provided with \asharp{} also make
%%AXIOM \index{AXIOM}
%%AXIOM significant use of extensions.  
%%AXIOM The types, as provided by the \axiom{} library,
%%AXIOM do not provide the operations necessary
%%AXIOM to participate gracefully in \asharp{} programs.
%%AXIOM 
%%AXIOM \pagebreak
%%AXIOM For example, the \verb"Integer" type does not have an operation
%%AXIOM \ttin{integer} so it could not use integer-style literals; the
%%AXIOM \verb"List" constructor does not provide a \verb"generator" function,
%%AXIOM %so it would not be possible to iterate over lists.
%%AXIOM so iteration over lists would be impossible.
%%AXIOM % changed to avoid bad pagebreak.
%%AXIOM 
%%AXIOM So, in order to allow \asharp{} to be a suitable vehicle to extend the 
%%AXIOM \axiom{} library, we have provided a set of post facto extensions
%%AXIOM as part of the \axiom{} library interface.  
%%AXIOM When a program uses \verb+#include "axiom"+
%%AXIOM it sees the extended \axiom{} types, dressed for use with \asharp{}.
%%AXIOM 
%%AXIOM These extensions also serve a second purpose:  
%%AXIOM to allow the \asharp{} compiler to do better optimisation for 
%%AXIOM programs based on the \axiom{} library.
%%AXIOM By providing redundant \asharp{} implementations for efficiency-critical
%%AXIOM functions, we make these operations available for inter-file optimisation.
%%AXIOM 
%%AXIOM The part of the \fname{axextend.as} file which provides
%%AXIOM the extension for the \verb"Vector" constructor is shown on
%%AXIOM page~\pageref{asugLangExtendAxiomVector}.
%%AXIOM We present this, not as the most beautiful  program in the world,
%%AXIOM but rather as a real-life example of using extensions to solve a problem.
%%AXIOM 
\thegeneralfile{h}{FigureExtend}{Post facto extension of the Base \asharp{} library.}{examples/sizextd.as}

%%AXIOM \clearpage
%%AXIOM \newpage
%%AXIOM {\bf Excerpt of {\tt "axextend.as"} for Vector}
%%AXIOM \label{asugLangExtendAxiomVector}
%%AXIOM 
%%AXIOM \program{axexvect}
