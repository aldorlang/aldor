% *********************************************************************
\head{chapter}{Introduction}{asugIntro}
% *********************************************************************

% *********************************************************************
\head{section}{What is \asharpname{}?}{asugIntroWhatsit}
% *********************************************************************

The original motivation for \asharpfull{} came from the field of
computer algebra:
to provide an improved extension language for the \axiom{} system.
\index{AXIOM}

The desire to model the extremely rich relationships among
mathematical structures has driven the design of \asharp{} in a
somewhat different direction than that of other contemporary programming
languages.  \asharp{} places more emphasis on uniform handling of
functions and types, and less emphasis on a particular object model.
%!! ``object model'' is a buzz word relating to the language,
%!! and will be understood differently than ``model of the target objects''.
%This provides the flexibility necessary to solve the original problem,
%and has already also proved of significant use outside
%of this initial context.
%!! Actually the mathematical relationships are the ``original problem''
%!! We could have ``extended the axiom system'' in many a way.
\asharp{} is an acronym, standing for {\bf A L}anguage for {\bf D}escribing
{\bf O}bjects and {\bf R}elationships.

The primary considerations in the formulation of \asharp{} have been
generality, composibility and efficiency.
The \asharp{} language has been specifically designed to admit a number
of important optimizations, allowing compilation to machine code whose
efficiency is frequently comparable to that produced by a good C
or Fortran compiler.
\index{Fortran}

\asharp{} is unusual among compiled programming languages, in that
types and functions are {\em first class}:
that is, both types and functions may be
constructed dynamically and manipulated
in the same way as any other values.
This provides a natural foundation for both object-oriented and functional
programming styles, and leads to programs
in which independently developed components may be combined
in quite powerful ways.

Two novel features of \asharp{} are {\em dependent types\/},
which allow static checking of dynamic objects, 
and {\em post facto type extensions}, which allow complex libraries
to be separated into decoupled components.

%\newpage
The \asharp{} compiler described in this Guide can produce:
\begin{itemize}
\item stand-alone executable programs,
\item object libraries in native operating system formats,
\item portable byte code libraries,
%\item C and Lisp source.
\item C source.
\end{itemize}
The object libraries produced by the \asharp{} compiler can be linked
with one another, or with C or Fortran code, to form application
programs.
The byte code libraries can be interpreted, and are used by the compiler
for inter-file optimization.
%The Lisp code generated by \asharp{} may be embedded in Lisp-based
%applications, including \axiom{}.
%\index{Lisp}
%\index{AXIOM}


The \asharp{} distribution includes:
\begin{itemize}
\item an optimising compiler for the \asharp{} language,
%!! Want to say "for the same" here because normally interpreters
%!! are compromised to be different than the compiled version.
\item an interpreted, interactive environment for the {\em same}
      language,
%\item ``\xasharpcmd{}'' --- an \x11{} program for interactive debugging
%      (for Motif systems only),
\item libraries providing data structures and mathematical abstractions,
\item library bindings for standard tools including
%\x11{} and the NAG Fortran Library,
the NAG Fortran Library,
\item sample programs for symbolics, numerics and graphics.
\end{itemize}

%A detailed list is given in \chapref{asugManifest}.

The \asharp{} compiler has been designed for portability and runs
in many different environments.
%including DOS/Windows, OS/2, VMS, CMS, Macintosh System 7
%and several Unix-derivatives.
%\index{CMS}%
%\index{DOS}%
%\index{Windows}%
%\index{OS/2}%
%\index{VMS}%
%\index{Unix}%
%\index{Macintosh}%
Code generated by \asharp{} will run on 16, 32 and 64-bit architectures.
%
%A list of \asharp{} ports at the time of writing is given
%on page~\pageref{PortsTable}.
For an up-to-date list of available implementations, please visit the
official \asharp{} website: \url{http://www.aldor.org}.

%\newpage
% *********************************************************************
\head{section}{Compiling and running a single file}{asugUsingSingle}
% *********************************************************************

The first thing many people want to do is compile and run
a simple test file.  This section shows how to do this.

We start with an \asharp{} source file, \fname{sieve.as},
\index{files!source}
\index{files!as@\protect{\tt .as}}
\index{source file}
containing the simple program shown in \figref{FigureSieve}.

\thegeneralfile{h}{FigureSieve}{An \asharp{} program.}{examples/sieve.as}

To compile this file and run the resulting executable program, use
the following commands:

\shio{sieve}

In this example ``\osprompt{}'' is the operating system command line
prompt and should not be typed.
On most platforms the command to run the \asharp{} compiler is
``\asharpcmd{}''.

The ``\asharpcmd{}'' command takes the source file \fname{sieve.as}
and produces a file of machine code which can perform the computation.
The executable program is named according to the operating system's usual
conventions: for instance, \fname{sieve} on Unix, or \fname{sieve.exe} on Windows.
Once compiled, the new program can be used in the same way as
other executable programs for the given operating system.

Command line options control the behaviour of the compiler.
For example, the option \option{-Fx} in the previous example
directs the compiler to produce an executable file. Also,
the option \option{-laldor} directs the linker to compile
the file using \fname{libaldor.a}.
\index{compiler options!F@\protect{-Fx}}

There are many available command line options, regulating different
aspects of the compiler's actions.
They allow you to control the details of what the compiler
actually does.  Here we point out a few of the most important options
--- the rest are described in detail in \chapref{asugOptions}.

Keep in mind that you do not need to remember very much.
The only option you really need to know is \option{-help},
which gives help.  The command is:
\index{compiler options!v@\protect{\tt -help}}

\begin{small}
\osprompt\ \asharpcmd{ -help}
\end{small}

Another thing to keep in mind is that you can make your programs run
{\em much\/} faster by asking the compiler to optimize them.
The \option{-O} option tells the compiler to do this:

\begin{small}
\osprompt\ \asharpcmd{ -O -Fx -laldor sieve.as}
\end{small}
\index{compiler options!v@\protect{\tt -O} (optimize)}
\index{optimization}

Depending on the way in which \asharp{} has been installed on your
computer, you may need to set some system-specific variable or macro
so that the compiler can find its libraries. 
The value for this will depend on where \asharp{} is installed.  
For example, on one of our local Unix systems, this is achieved
by setting the ``environment variable''
\ttin{ALDORROOT}\index{ALDORROOT} to

\begin{small}
\begin{verbatim}
/usr/local/aldor
\end{verbatim}
\end{small}

To be able to use \asharp{} on this particular system,
one might put the following commands in a
(Bourne or Korn shell) initialization file:

\begin{small}%
\begin{verbatim}
ALDORROOT=/usr/local/aldor
PATH=$ALDORROOT/bin:$PATH
export ALDORROOT
export PATH
\end{verbatim}
\end{small}

Please refer to your system administrator for details of the
corresponding setup on your particular computer system.

%% % *********************************************************************
%% \head{section}{Quick answers}{asugIntroQuick}
%% % *********************************************************************
%% 
%% This section provides quick answers to certain common questions.
%% You shouldn't have to read this entire manual if you just want a quick
%% answer to one of these.
%% 
%% \begin{description}
%% \item[What is \asharp{}?] % \\
%% \asharp{} is a language for computer programming.  
%% The compiler for \asharp{} may be used to write programs extending
%% the \axiom{} system, or it may be used by itself interactively,
%% or it may be used to produce libraries for use with C or Fortran.
%% See \chapref{asugIntroWhatsit} and \spadref{asugUsingAxiom}.
%% 
%% \item[Who should use \asharp{}?] % \\
%% Fortran programmers who want an efficient, attractive high-level alternative.
%% Frustrated C++ programmers.
%% Users wanting to write their own programs for use within \axiom{}.
%% Developers wanting to write computer algebra software to embed in other
%% applications.
%% People wanting to access standard libraries from their computer algebra code.
%% People writing computer algebra programs who need exact semantics
%% or efficiency.  People doing symbolic-numeric computations.
%% 
%% \item[Where can I see some \asharp{} programs?] % \\
%% See \figref{FigureSieve} and \chapref{asugSample}.
%% 
%% \item[When will \asharp{} be available?] % \\
%% \asharp{} is presently available to beta testers.
%% We anticipate that it will initially become generally available
%% as part of Axiom version 2.0, scheduled for release in 1994.
%% 
%% \item[How do I install \asharp{} on my computer?] % \\
%% Answer 1:  Install Axiom version 2.0.
%% Answer 2:  See Appendix \asugInstallAppendix{}.
%% 
%% \item[How do I use \asharp{} with Axiom?] % \\
%% See \chapref{asugUsingAxiom}.
%% 
%% \item[How do I use C libraries from \asharp{} or {vice versa}?] % \\
%% See \spadref{asugUsingC}.
%% 
%% \item[How do I use \asharp{} interactively?] % \\
%% Use \ttin{aldor -i}.  See \chapref{asugUsingInteractive}.
%% 
%% \item[So what are the ``8 meanings for xyzzy''?] % \\
%% Compile with option \ttin{-M2} to get a more detailed message listing
%% all the possible interpretations.
%% 
%% \item[Can I distribute the C or Lisp code I generate?] % \\
%% Yes, you may distribute the C or Lisp code generated by \asharp{}
%% from programs you write. 
%% 
%% Note, though, that when using the cross-file inlining feature
%% of the \asharp{} optimizer,  the C or Lisp code is generated not
%% only from your program, but also possibly from the libraries you use.
%% Libraries get inlined only when you give permission
%% with the \ttin{inline} keyword or when you use the \ttin{-Qinline-all}
%% command-line option.  
%% 
%% Normally \asharp{} programs have permission to inline from 
%% basic libraries supplied with the compiler: \ttin{libfoam} and \ttin{libaldor}.
%% You may distribute the C or Lisp code generated from your programs which
%% use these libraries.
%% 
%% To distribute generated code which incorporates code from additional libraries,
%% you will have to get permission from the owners of those libraries.
%% 
%% \item[Is the \TBA{} language ``copyrighted'' ?] % \\
%% The {\em language} is not protected by copyrights ---
%% the {\em program} which implements the compiler for the language is.
%% If you want to write your own compiler, interpreter, program formatter
%% or other software tool, please go ahead.
%% 
%% We are, however, somewhat protective regarding the name.  Want to ensure
%% that a program which claims to handle the \TBA{} language actually does.
%% Therefore we are seeking trademark protection for the name ``\TBA{}''.
%% If you wish to use the name \TBA{} to describe your software
%% you must obtain our permission. 
%% \end{description}

%\newpage
% *********************************************************************
\head{section}{This guide}{asugIntroThis}
% *********************************************************************

This guide describes the \asharp{} programming language,
a compiler and an interpreter, and other related software.

{\bf Part I:}
The first two chapters provide a quick, informal introduction to
\asharp{}.
\begin{description}
\item[Chapter \ref{asugIntro}]
  provides an introduction and indicates how to compile and run
  simple programs.
  It gives a very brief description of what \asharp{} is and what the
  compiler can do.  \Spadref{asugIntroBugs} explains how to report
  problems.
\item[Chapter \ref{asugLangSimple}]
  discusses a number of (mainly) very simple programs.
\end{description}
{\bf Part II:}
The next chapters provide a guide to the \asharp{} programming language.
\begin{description}
\item[Chapters \ref{asugLangSummary} to \ref{asugLangStdints}]
present in detail the various aspects
of the language and provide a number of illustrative examples.
\end{description}

Other chapters in this Guide can also be useful in learning about the
language.
%A more leisurely introduction to some topics, including an
%extended example of building a new type from scratch, is given in
%\chapref{asugFNotes}.
Additional programming examples are discussed
in \chapref{asugSample}.  The formal language syntax is given in
\chapref{asugFLang}.

{\bf Part III:}
The next five chapters serve as a guide to the \asharp{} compiler and
related software.
\begin{description}
\item[Chapter \ref{asugMsgs}]
  explains how to interpret and control messages from the compiler.
\item[Chapter \ref{asugSeparate}] 
  describes how to build an \asharp{} program from several separately
  compiled files.
\item[Chapter \ref{asugUsingInteractive}]
  shows how to use the \asharp{} compiler interactively, to compile
  and evaluate a line of code at a time.  
%\item[Chapter 18]
%  explains how to use \asharp{} as a tool to extend the \axiom{} system.
%  Programs written in \asharp{} can have full access to the {\tt axiom}
%  library and are understood by Version 2.0 of the \axiom{} system.
\item[Chapter \ref{asugUsingWithC}]
  shows how to write \asharp{} programs which call C programs and
  {\it vice versa\/}.
\item[Chapter \ref{asugUsingWithFortran}]
  shows how to write \asharp{} programs which call Fortran programs and
  {\it vice versa\/}.
%\item[Chapter 20]
%  describes \xasharp, a graphical user interface environment which
%  communicates with the \asharp{} compiler for compiling
%  and debugging programs.
\end{description}

{\bf Part IV:}
The next chapter provides some examples to
help learn the language.
\begin{description}
%\item[Chapter \ref{asugSample}]
%  is a tutorial on \asharp{} by Tim Daly from the point of view of
%  an experienced \axiom{} programmer.  This  tutorial provides a leisurely
%  development showing how to do in \asharp{} the kinds of things
%  done in the \axiom{} library.
%  % Above changed for consistency with "Ch.Old-6" entry.  MGR
%% There is some feeling locally that Tim Daly's chapter makes less
%% demand on potential users and so should appear before Rob Corless's.
%% Done (provisionally) in the above.  MGR
%\item[Chapter 22]
%  is a narrative by Robert Corless, relating his experiences as
%  a complete novice to \asharp{}.  This chapter is organized as a
%  series of vignettes, describing what he learned from each of his
%  initial \asharp{} programs.
\item[Chapter \ref{asugSample}]
  provides a number of detailed sample programs.
  This includes examples which range from trivial half-page programs
  to complete applications.
  These provide concrete illustrations of how to use the various
  aspects of the programming language.
\end{description}

{\bf Part V:}
The remaining chapters provide reference material, and are not
intended to be read sequentially.
\begin{description}
\item[Chapter \ref{asugFLang}] 
  is a formal description of the language syntax.
%\item[Chapters 25 and 26] 
%  describe the base \asharp{} library.  
%  This library can be used by stand-alone \asharp{} programs,
%  or \asharp{} code linked into C-based applications.  
%  For this release, it is not possible to use this library together
%  with the {\tt axiom} library.
\item[Chapter \ref{asugOptions}] 
  provides a detailed description of the \ttin{aldor} command.
  It describes the types of files, all the options, and the environment
  variables understood by the compiler.
\item[Chapter \ref{asugUnicl}]
  discusses the use of the back-end compiler and linker driver \uniclcmd{}.
\item[Chapter \ref{asugMessages}]
  lists all the messages which the compiler can produce.  The names
  of the messages are also listed so you can turn off specific messages
  if you wish.
%\item[Chapter 30]
%%  gives a detailed list of the various components of the \asharp{} 
%%% Gives a bad linebreak.
%  gives a detailed list of the components of the \asharp{} 
%  distribution directory.
\end{description}

%{\bf Part VI:}
%The remaining chapters provide documentation for the features added
%at Version 1.1.12 of the compiler
%\begin{description}
%\item[Chapter 31] 
%  introduces the exception handling mechanism.
%\item[Chapter 32] 
%  discusses the facilities for linking FORTAN code with \asharp{}.
%\end{description}
%\newpage
% *********************************************************************
\head{section}{Reporting problems}{asugIntroBugs}
% *********************************************************************

If you discover an error in the \asharp{} compiler, libraries,
or companion software we want to know about it so we can fix it.

When reporting a problem, please supply
the precise compiler version and
have a file that demonstrates the problem.
To determine your compiler version, use the \option{-v} option
to cause the \asharp{} compiler to operate verbosely.
The first output line will contain the compiler version.  For example,

\shrec{verbose}

There are two ways to report a problem:
\begin{itemize}
\item (recommended) Use the {\tt aldorbug} tool supplied with any
\asharp{} distribution to send a description of the problem and all the
necessary files to reproduce it.
\end{itemize}
or
\begin{itemize}
\item Send an email with a description of the problem and all the
necessary files to reproduce it to {\tt bug-report@aldor.org}.
\end{itemize}

%On Unix-derivative systems, the \asharp{} distribution contains a
%program called \ttin{aldorbug} for electronically mailing bug reports.
%The {\tt aldorbug} program prompts the user for information;
%the responses in the example below are shown in italics:
%
%{\small \tt
%\% aldorbug \\
%Priority (1=urgent,..,9=detail): {\it 8}\\
%Subject (brief description): {\it Bad error msg compiling /etc/passwd}\\
%Name of file with detailed description (default none): {\it problem.rpt} \\
%Please give the exact command line \\
%(example: 'aldor -Q3 file.as', say 'none' if is irrelevant):  \\
%{\it aldor /etc/passwd}\\
%Source file demonstrating the bug: {\it /etc/passwd} \\
%Compiler version: {\it v0.35.0 for AIX ESA} \\
%}
%\index{aldorbug@\protect{\tt aldorbug}}
%
% For non-Unix systems, please send E-mail to infodesk@nag.co.uk.
%\input{ports.tbl}
