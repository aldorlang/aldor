% *********************************************************************
\head{chapter}{Separate compilation}{asugSeparate}
% *********************************************************************
\index{separate compilation}

% Redefinition for asugmacs.tex:
% \newcommand{\fname}[1]{``{\tt #1}''}
% I can't see any reason to have the `"' in monospace - it isn't part of
% the name (and isn't done in analogous commands).  MGR

This chapter describes how to run the \asharp{} compiler to produce
libraries or executable programs from multiple files.

% *********************************************************************
\head{section}{Multiple files}{asugSeparateMulti}
% *********************************************************************
\thegeneraltwofile{t}{SeparateCompFigure}%
{A program consisting of two files.}{examples/choose.as}{examples/poker.as}

Unless a program is very small, it is normal to develop it in stages
and to identify parts of it for potential re-use, compiling them 
separately.  \asharp{} supports separate compilation to platform-independent
\ttin{.ao} files.
\index{files!ano@\protect{\tt .ao}}
These files contain type information, intermediate code and other information
to allow type-safe separate compilation and cross-file optimisation.
These files have a portable format and it is possible to
move them between machines of different architectures, with different
character sets, byte orders or floating-point formats.
The \ttin{.ao} files can be thought of as platform-independent object files,
which may be imported into other \asharp{} programs or used to generate
C or Lisp or object code for a particular platform.

Let us give a a toy example which illustrates the steps one takes to
do separate compilation with the \asharp{} compiler.
Suppose we have two files: \fname{choose.as}, containing some functions,
and \fname{poker.as}, which uses them.
These files are shown in Figure \ref{SeparateCompFigure}.
The commands to compile these individual files together are:

\shin{separate}

The first step compiles the file \fname{choose.as}.
The \option{-Fao} option causes the compiler to create
a platform-independent file, \fname{choose.ao}, and the
\option{-Fo} option gives a platform-dependent object file,
such as \fname{choose.o} on Unix, containing machine code.
The \option{-O} option requests optimised code.

%\newpage
The second step compiles the file \fname{poker.as},
which uses \fname{choose.as}.
The \option{-lChooseLib=choose.ao} option tells the compiler that the
library \fname{choose.ao} is to be made visible within the program
as a package named \ttin{ChooseLib}.
This \option{-l} option could be avoided by using a \verb"#library"
command, as described in ~\secref{asugCmdlineLibSysCmd}.

In the third command, the \option{-Fx} option directs the compiler
%%% WHAT third command?  There are only two in "separate.in"!
to link an executable image from the object files on the command line.
The \option{-e} option causes the resulting program to begin execution
by evaluating the top-level statements of the file \fname{poker.as}.
The \option{-e} option also directs the compiler to give the executable
file the name \fname{poker} (possibly with some system-dependent extension,
such as \ttin{.exe} under DOS).
\index{compiler options!F@\protect{\tt -Fx}}
\index{compiler options!E@\protect{\tt -e}}

%AXIOM Note that if you were using \asharp{} to compile \axiom{} libraries
%AXIOM separately, then \option{-Fo} options and the link step would not be
%AXIOM necessary.
%AXIOM \index{compiler options!F@\protect{\tt -Fo}}
%AXIOM \index{AXIOM}

% *********************************************************************
\head{section}{Libraries}{asugCmdlineLib}
% *********************************************************************
%
In order to handle large numbers of separately compiled files,
the \asharp{} compiler allows \ttin{.ao} files to be combined
into aggregate \ttin{.al} files.  The aggregate library files
may be created and maintained on Unix using the \ttin{ar} command.
For the platforms for which \ttin{ar} is not available, the \asharp{}
distribution provides a program \ttin{uniar}.
\index{files!.ao}
\index{files!.al}

When specifying a library to the compiler with the {\tt -l{\it name}} option,
\index{libraries}
{\it name} is treated as a filename if it has a file extension or a
directory specification.  Otherwise it is treated as a shorthand reference
to either or both of the libraries \ttin{lib{\it name}.al} and
\ttin{lib{\it name}.a} in the current directory or on {\tt \$LIBPATH}. 
\index{LIBPATH}

Examples:
\begin{itemize}

\item The command-line argument {\tt -lnewmath} is treated as a reference
to the library {\tt libnewmath.al} in the current directory or on
{\tt \$LIBPATH}.%
\index{compiler options!l@\protect{\tt -l} (library)}%

\item The command-line argument {\tt -lmyfile.ao} is treated
as a reference to the file {\tt myfile.ao} in the current directory
or on {\tt \$LIBPATH}.
\index{files!ano@\protect{\tt .ao}!and \protect{\tt -l} option}

\item The command-line argument {\tt -lstuff/myfile.ao} is treated
as a relative pathname reference to the file {\tt myfile.ao} in
the directory {\tt stuff}.

\item The command-line argument {\tt -l/u/joe/myfile.ao} is treated
as an absolute pathname reference to the file {\tt myfile.ao} in the
directory {\tt /u/joe}.

\end{itemize}

When the \asharp{} compiler is asked to create an executable file,
references to libraries via {\tt -l} will be passed to the linker,
if appropriate.  This can be convenient on certain platforms.
For example, on Unix, it allows one to have  \ttin{lib{\it xxx}.al}
and \ttin{lib{\it xxx}.a} containing the platform-independent \ttin{.ao}
files and the platform-dependent \ttin{.o} files, respectively.

% *********************************************************************
\head{section}{Source code references to libraries}{asugCmdlineLibSysCmd}
% *********************************************************************

A separately compiled module can be referenced directly from the
source text of a program much the same way that it is referenced
from the command line.
In a source program, the system command {\tt \#library} plays the
same r\^ole as {\tt -l} does from the command-line.
\index{library@\protect{\tt \#library}}
The treatment of names in the {\tt \#library} system command is
exactly the same as in the {\tt -l} command-line argument
(see \secref{asugOptionsDirLib}.
When using the {\tt \#library} system command, however, remember
to put the name in quotation marks.

References to files using {\tt \#library} will be passed to the linker
under the same circumstances as the corresponding {\tt -l} command-line
argument.

% *********************************************************************
\head{section}{Importing from compiled libraries}{asugCmdlineLibTag}
% *********************************************************************

Compiled object files ({\tt .ao}) and compiled libraries ({\tt
.al}) have the same top-level semantics as \asharp{} domains:
before the symbols defined in a separately compiled file are
visible during the compilation of a given source file, the symbols
must be imported into the current scope.
Before a domain can be imported, it must be given a name.
The same is true of compiler object files: before symbols can be
imported from an object file or a library, it must be given a name.

Names are assigned to compiler libraries using an extension of the
{\tt -l} syntax and an extension of the {\tt \#library} syntax
described in the previous two sections.
If the argument to either of these forms is prefixed by a symbol
(which should {\it not} be enclosed in quotation marks), then the
compiler object or library which follows is taken as the
definition of a domain whose name is the given symbol.
%%% Why does the example in section 4.1 say "-lChooselib=choose.ao"
%%% rather than "-lChooselib choose.ao"? Are these optional
%%% alternatives or do the keyline and inline syntaxes differ?
The symbols exported from the newly named domain are exactly those
exported from the source code used to produce the compiled object.

Consider the following \asharp{} source file:

\begin{verbatim}

#library String "/tmp/mystring.ao"
import from String;

...

\end{verbatim}

%The first two lines specify that the domain {\tt String}, which is
%% Avoiding bad line break:
The first two lines specify that the domain {\tt String},
usually provided by the compiler libraries found in the
distribution, will, for the extent of the file, be provided by the
compiled object file \fname{/tmp/mystring.ao}.
The {\tt \#library} command specifies a binding for the symbol
{\tt String}, and then the {\tt import} command operates as it does
in any other context, extracting the exported symbols from the
domain {\tt String} and making them visible in the current scope.%
\index{importing from libraries}  Note that, unlike the command line
usage, the \ttin{\#library} command requires a space, rather than
\ttin{=}, between the domain and library names.

