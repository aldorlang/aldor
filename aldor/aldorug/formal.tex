\head{chapter}{Formal syntax}{asugFLang}

\head{section}{Source}{asugFLangSource}

\asharp{} source is a collection of lines containing
a textual representation of a program.

Lines beginning with the character ``\verb'#'''
are {\em system commands} and are not part of the program text.

\vskip .5cm

\subsection{Source inclusion}%
{\em Source inclusion} collects the source lines which make up a program.
This process is controlled by the following system commands:
\index{include@\protect{\tt \#include}}
%%\index{\#include@\protect{\tt \#include}}
\begin{itemize}
\item
\verb'#include   "'{\it file-name\/}\verb'"'
\\
\verb'#reinclude "'{\it file-name\/}\verb'"'
\\
These commands collect the lines from the named file.
If a given file has already been included, then subsequent
\verb'include' commands for that file have no effect.
A \verb'reinclude' command always includes the file, whether or not it
has already been included.

The includer tries to find the file relative to the directory of the current
source file and then in a sequence of user-specified and platform-specific
directories.
%
\item
\verb'#assert  '  {\it identifier\/}
\\
\verb'#unassert'  {\it identifier\/}
\\
These commands turn on or off named properties which may be tested
for conditional source inclusion.
\index{global assertions!inside a source file}
%%\index{\#assert@\protect{\tt \#assert}}
\index{assert@\protect{\tt \#assert}}
%%\index{\#unassert@\protect{\tt \#unassert}}
\index{unassert@\protect{\tt \#unassert}}
%
\item
\verb'#if    '    {\it identifier\/}
\\
\verb'#elseif'    {\it identifier\/}
\\
\verb'#else'
\\
\verb'#endif'
\\
These commands provide conditional source inclusion.
\index{conditional source inclusion}
%%\index{\#if@\protect{\tt \#if}}
\index{if@\protect{\tt \#if}}
%%\index{\#elseif@\protect{\tt \#elseif}}
\index{elseif@\protect{\tt \#elseif}}
%%\index{\#else@\protect{\tt \#else}}
\index{else@\protect{\tt \#else}}
%%\index{\#endif@\protect{\tt \#endif}}
\index{endif@\protect{\tt \#endif}}
%
\item 
\verb'#!' {\it text\/}
\\
\verb'#'
\\
Lines are ignored if they begin with ``\verb'#!''' or
begin with ``\verb'#''' and contain only white space.
\end{itemize}

\subsection{Prepared source}%
The following commands allow source to be prepared by another program.
\begin{itemize}
\item
\verb'#line'      {\it line-number\/} [\verb'"'{\it file-name\/}\verb'"']
\\
The next line is recorded as occurring at the given line number
in the named file.  The first line of a file is line number 1.
If the file name is missing, then the current file name is used.
%%\index{\#line@\protect{\tt \#line}}
\index{line@\protect{\tt \#line}}
%
\item
\verb'#error' {\it text\/}
\\
An error is considered to have occurred at the position of the system
command line and the given text is used as the message.
%%\index{\#error@\protect{\tt \#error}}
\index{error@\protect{\tt \#error}}
\end{itemize}
%
\subsection{Other system commands}%
Other system commands control the environment:
\begin{itemize}
\item
\verb'#pile'
\\
\verb'#endpile'
\\
These commands create a context in which indentation is significant.
Braces may be used to revert to a normal
state where indentation has no meaning, and
it is possible to nest
``\verb"#pile"''/``\verb"#endpile"''
and
``\verb"{"''/``\verb"}"''
pairs.
Closing ``\verb"#endpile"'' lines
at the end of the program may be omitted.
%%\index{\#pile@\protect{\tt \#pile}}
\index{pile@\protect{\tt \#pile}}
%%\index{\#endpile@\protect{\tt \#endpile}}
\index{endpile@\protect{\tt \#endpile}}
\item
\verb'#library' {\it identifier\/} \verb'"'{\it file-name\/}\verb'"'
\\
This links an entry in the file system to an identifier in the
source program.  
The identifier is treated as a defined constant with a domain value
in the file scope.
A specified sequence of directories is searched for the named library file.
See ``\verb"#libraryDir"''.
%%\index{\#library@\protect{\tt \#library}}
\index{library@\protect{\tt \#library}}
%
\item
\verb'#includeDir' \verb'"'{\it directory-name\/}\verb'"'
\\
\verb'#libraryDir' \verb'"'{\it directory-name\/}\verb'"'
\\
The given directory is prepended to a sequence of directories to be searched
for include files or libraries, respectively.
The sequences are initialised to platform-specific sets, which may be
augmented by environment variables or command line arguments.
%%\index{\#libraryDir@\protect{\tt \#libraryDir}}
\index{libraryDir@\protect{\tt \#libraryDir}}
%
\item
\verb'#int' {\it options}
\\
This command controls the behaviour of the compiler when used 
interactively. See \secref{asugInteractSyscmds} for a complete
description of available options.
%%\index{\#int@\protect{\tt \#int}}
\index{int@\protect{\tt \#int}}
\item
\verb'#quit'
\\
The \verb'quit' command causes the language processor to abandon the
program. If \asharp{} is running in interactive mode (\option{-g
loop}), then this command causes the termination of the interactive session.
%%\index{\#quit@\protect{\tt \#quit}}
\index{quit@\protect{\tt \#quit}}
\end{itemize}

\vbox{
\widesection{Lexical structure}
\vspace{-3mm}
\subsection{Characters}%
The \Term{standard \asharp{} character set} contains the
following 97 characters:
\begin{itemize}
\item the blank, tab and newline characters
\item the Roman letters: \verb'a-z A-Z'
\item the digits: \verb'0-9'
\item and the following special characters:
\\
\begin{tabular}{llll}
\verb'('  & left parenthesis   & \verb')' & right parenthesis  \\
\verb'['  & left bracket       & \verb']' & right bracket      \\
\verb'{'  & left brace         & \verb'}' & right brace        \\
\verb'<'  & less than          & \verb'>' & greater than       \\
\verb','  & comma              & \verb'.' & period             \\
\verb';'  & semicolon          & \verb':' & colon              \\
\verb'?'  & question mark      & \verb'!' & exclamation mark   \\
\verb'='  & equals             & \verb'_' & underscore         \\
\verb'+'  & plus               & \verb'-' & minus (hypen)      \\
\verb'&'  & ampersand          & \verb'*' & asterisk           \\
\verb'/'  & slash              & \verb'\' & back-slash         \\
\verb"'"  & apostrophe (quote) & \verb'`' & grave (back-quote) \\
\verb'"'  & double quote       & \verb'|' & vertical bar       \\
\verb'^'  & circumflex         & \verb'~' & tilde              \\
\verb'@'  & commercial at      & \verb'#' & sharp              \\
\verb'$'  & dollar             & \verb'%' & percent
\end{tabular}
\end{itemize}
Other characters may appear in source programs, but only in comments and
string-style literals.
Blank, tab and newline are called \Term{white space} characters.
All the special characters except quote, grave and ampersand are required for
use in tokens.  Grave and ampersand are reserved for future use.

\subsection{The escape character}%
Underscore is used as an ``escape'' character, which alters the meaning
of the following text.  The nature of the change depends on the context
in which the underscore appears.  

An escaped underscore is not an escape character.

An escape character followed by one or more white space characters
causes the white space to be ignored.
%\footnote{It is planned that it will also cause comments to be ignored.}
The remainder of this section assumes that escaped white space has been
removed from the source.
}

\head{subsection}{Tokens}{asugFLangLexTokens}%
The sequence of source characters is partitioned into \Term{tokens}.
The longest possible match is always used.

The tokens are classified as follows:
\begin{itemize}
%
\item the following language-defined \Term{keywords}:
\begin{verbatim}
add      and       always    assert    break
but      catch     default   define    delay
do       else      except    export    extend
fix      for       fluid     free      from
generate goto      has       if        import
in       inline    is        isnt      iterate
let      local     macro     never     not
of       or        pretend   ref       repeat
return   rule      select    then      throw
to       try       where     while     with
yield

.   ,    ;    :    ::   :*   $    @
|   =>   +->  :=   ==   ==>  '
[   ]    {    }    (    )
\end{verbatim}
The characters in a keyword cannot be escaped.
That is, if a character is escaped, the token is not 
treated as a keyword.
%
\item the following are not defined by the language
but are \Term{reserved words} for future use:
\begin{verbatim}
delay    fix       is        isnt       let       rule

(|  |)   [|  |]    {|  |}    `    &    ||
\end{verbatim}
%
\item the following set of definable \Term{operators}:
\begin{verbatim}
by      case     mod      quo      rem

#        +       -        +-       ~        ^
*        **      ..       =        ~=       ^=   
/  \     /\  \/  <   >    <=  >=   <<  >>   <-   ->   
\end{verbatim}
The characters in an operator cannot be escaped.
%
\item \Term{identifiers}:
\\
\verb'0'
\\
\verb'1'
\\
\verb'[%a-zA-Z][%?!a-zA-Z0-9]*'
\\
Any non-white space standard character may be included in an identifer
by escaping it.  
Thus ``\verb"a"'', ``\verb"_*"'', ``\verb"a_*"'' and ``\verb"_if"''
are all identifiers.
The escape character is not part of the identifier so ``\verb"ab"''
``\verb"_a_b"'' represent the same identifier.
Identifiers are the only tokens for which the leading character may be
escaped.
%\item \Term{pattern variables}:
%\\
%\verb'?[%?!a-zA-Z0-9]*'
%
\item \Term{string-style literals}:
\\
\verb=`"'[^"]*`"'=
\\
An underscore or double quote may be included in a string-style literal
by escaping it.
%
\item \Term{integer-style literals}:
\\
\verb=[2-9]=
\\
\verb=[0-9][0-9]+=
\\
\verb=[0-9]+`r'[0-9A-Z]+=
\\
Escape characters are ignored in integer-style literals and so may be used
to group digits.
\item \Term{floating point-style literals}:
\\
\verb=[0-9]*`.'[0-9]+{[eE]{[+-]}[0-9]+}=
\\
\verb=[0-9]+`.'[0-9]*{[eE]{[+-]}[0-9]+}=
\\
\verb=[0-9]+[eE]{[+-]}[0-9]+=
\\
\verb=[0-9]+`r'[0-9A-Z]*`.'[0-9A-Z]+{e{[+-]}[0-9]+}=
\\
\verb=[0-9]+`r'[0-9A-Z]+`.'[0-9A-Z]*{e{[+-]}[0-9]+}=
\\
\verb=[0-9]+`r'[0-9A-Z]+`e'{[+-]}[0-9]+=
\\
Escape characters are ignored in floating point-style literals
and so may be used to group digits.

Certain lexical contexts restrict the form of floats allowed.
This distinguishes cases such as \verb'sin 1.2' {\it vs} \verb'm.1.2'.
A floating point literal may not
\begin{itemize}
\item 
begin with ``\verb"."'', unless the preceding token is a keyword
other than ``\verb")"'', ``\verb"|)"'', ``\verb"]"'' or ``\verb"}"'',
\item
contain ``\verb"."'', if the preceding token is ``\verb"."'',
\item
end with ``\verb"."'', if the following character is ``\verb"."''.
\end{itemize}
%
\item \Term{comments}:
\\
The two characters ``\verb'--''' and all characters up to the end of the line.
Underscores are not treated as escape characters in comments.
%
\item \Term{documentation}: 
\\
The two characters ``\verb'++''' and all characters up to the end of the line.
Underscores are not treated as escape characters in documentation.
%
\item \Term{leading white space}:
\\
a sequence of blanks or tabs at the beginning of a line.
%
\item \Term{embedded white space}:
\\
a sequence of blanks or tabs not at the beginning of a line.
%
\item \Term{newline}:
\\
a newline character.
%
\item \Term{layout markers}:
\\
{\it SETTAB}\verb"       "{\it BACKSET}\verb"        "{\it BACKTAB}
\\
These do not appear in a source program but may be used
to represent a linearized from of the token sequence.
\end{itemize}
Comments and embedded white space are always ignored, except as used to 
separate tokens.  
For example, ``\verb"abc"'' is taken as one token but ``\verb"a b c"''
is taken as three.

\head{section}{Layout}{asugFLangLayout}
Normally page layout is not significant in an \asharp{} program.
Within a ``\verb"#pile"''/``\verb"#endpile"'' pair, however,
indendation and newlines have meaning,
and are used to group collections of lines.
Source within such a pair is in a \Term{piling context}.

Indentation sensitivity may be turned off by enclosing source
in a ``\verb"{"''/``\verb"}"'' pair.
Within braces all white space~--- leading, embedded, and newlines~---
is ignored.
This is a \Term{non-piling context}.
Piling and non-piling contexts may be nested.

The layout of a program in a piling context is understood by converting
leading white space and newlines to special markers
which are part of the language syntax.

This conversion follows the \Term{linearization rules}:
%\footnote{These rules are somewhat out of date and will be simplified.}
\begin{itemize}
\item Blank lines are ignored.
\item Consecutive lines indented the same ammount form a pile.
\item The pile is enclosed in a {\it SETTAB}-{\it BACKTAB} pair if
  \begin{itemize}
  \item the pile has more than one line, or
  \item the pile has only one line and the last token of the line before the
    pile is ``\verb"then"'', ``\verb"else"'', ``\verb"with"'' or ``\verb"add"''.
  \end{itemize}
\item The pile lines are joined to form a single line.
 A {\it BACKSET} inserted between adjacent pair of lines, unless
  \begin{itemize}
  \item the first contains only comments or documentation, or
  \item the first ends with ``\verb"("'', ``\verb"["'',
     ``\verb"{"'' or ``\verb","'', or
  \item the second begins with ``\verb"in"'', ``\verb"then"'',
     ``\verb"else"'', ``\verb")"'', ``\verb"]"'' or ``\verb"}"''.
  \end{itemize}
\item A line is joined to the previous line if it is indented with
  respect to it, forming a new single line.
\end{itemize}
These rules are applied from the most indented lines back out to the least
indented lines.

\clearpage
\section{Grammar}%

This section presents the grammar used by the \asharp{} compiler.
After expanding parameterized rules, this grammar is conflict-free and LALR(1).


\newcommand{\altline}{\\ \hspace*{4mm}}

\vskip 2cm
\subsubsection{Declarative expressions}

\Sn{Goal}
: \SnRule{CurlyContents}(\Sn{Labeled})

\Sn{Expression}
: \SnRule{enlist1a}(\Sn{Labeled}, ``\verb";"'')

\Sn{Labeled}
: \Sn{Comma}
\Alt{} \Sn{Declaration}
\Alt{} ``\verb"@"'' \Sn{Atom} \Opt{\Sn{Labeled}}

\Sn{Declaration}: \altline
\Kw{macro}   \Sn{Sig} \altline
\Alt{} \Kw{extend}  \Sn{Sig} \altline
\Alt{} \Kw{local}   \Sn{Sig} \altline
\Alt{} \Kw{free}    \Sn{Sig} \altline
\Alt{} \Kw{fluid}   \Sn{Sig} \altline
\Alt{} \Kw{default} \Sn{Sig} \altline
\Alt{} \Kw{define}  \Sn{Sig} \altline
\Alt{} \Kw{fix}     \Sn{Sig} \altline
\Alt{} \Kw{inline} \Opt{\Sn{Sig}} \Opt{\Sn{FromPart}} \altline
\Alt{} \Kw{import} \Opt{\Sn{Sig}} \Opt{\Sn{FromPart}} \altline
\Alt{} \Kw{export} \Opt{\Sn{Sig}} 
\Alt{} \Kw{export} \Opt{\Sn{Sig}} \Sn{ToPart}
\Alt{} \Kw{export} \Opt{\Sn{Sig}} \Sn{FromPart}

\Sn{ToPart}
: \Kw{to} \Sn{Infixed}

\Sn{FromPart}
: \Kw{from} \SnRule{enlist1}(\Sn{Infixed}, ``\verb","'')

\Sn{Sig}
: \Sn{DeclBinding}
\Alt{} \Sn{Block}

\Sn{DeclPart}
: ``\verb":"'' \Sn{Type}
\Alt{} ``\verb":*"'' \Sn{Type}

\Sn{Comma}
: \SnRule{enlist1}(\Sn{CommaItem}, ``\verb","'')

\Sn{CommaItem} : \altline
\SnRule{Binding}(\Sn{AnyStatement}) \altline
\Alt{} \SnRule{Binding}(\Sn{AnyStatement}) \Kw{where} \Sn{CommaItem}

\Sn{DeclBinding}
: \SnRule{BindingR}(\Sn{InfixedExprsDecl}, \Sn{AnyStatement})

\Sn{InfixedExprsDecl}
: \Sn{InfixedExprs}
\Alt{} \Sn{InfixedExprs} \Sn{DeclPart}

\Sn{InfixedExprs}
: \SnRule{enlist1}(\Sn{InfixedExpr}, ``\verb","'')


\SnRule{Binding}(E)
: \SnRule{BindingL}(\Sn{Infixed}, E)

\SnRule{BindingL}(R, L): \altline
L \altline
\Alt{} R ``\verb":="''  \SnRule{BindingL}(R, L)
\Alt{} R ``\verb"=="''  \SnRule{BindingL}(R, L) \altline
\Alt{} R ``\verb"==>"'' \SnRule{BindingL}(R, L)
\Alt{} R ``\verb"+->"'' \SnRule{BindingL}(R, L)

\SnRule{BindingR}(R, L): \altline
R \altline
\Alt{} R ``\verb":="''  \SnRule{Binding}(L)
\Alt{} R ``\verb"=="''  \SnRule{Binding}(L) \altline
\Alt{} R ``\verb"==>"'' \SnRule{Binding}(L)
\Alt{} R ``\verb"+->"'' \SnRule{Binding}(L)

\subsubsection{Control flow}

\Sn{AnyStatement}: \altline
\Kw{if} \Sn{CommaItem} \Kw{then} \SnRule{Binding}(\Sn{AnyStatement}) \altline
\Alt{} \SnRule{Flow}(\Sn{AnyStatement})

\Sn{BalStatement}
: \SnRule{Flow}(\Sn{BalStatement})

\SnRule{Flow}(X): \altline
\Sn{Collection}  \altline
\Alt{} \Kw{if} \Sn{CommaItem} \Kw{then} \SnRule{Binding}(\Sn{BalStatement}) \Kw{else} \SnRule{Binding}(X) \altline
\Alt{} \Sn{Collection} ``\verb"=>"'' \SnRule{Binding}(X) \altline
\Alt{} \Sn{Iterators} \Kw{repeat} \SnRule{Binding}(X) \altline
\Alt{} \Kw{repeat} \SnRule{Binding}(X) \altline
\Alt{} \Kw{try} \SnRule{Binding}(\Sn{AnyStatement}) \Kw{but} \Opt{\Sn{Cases}} \SnRule{AlwaysPart}(X)  \altline
\Alt{} \Kw{select} \SnRule{Binding}(\Sn{AnyStatement}) \Kw{in} \Sn{Cases} \altline
\Alt{} \Kw{do} \SnRule{Binding}(X) \altline
\Alt{} \Kw{delay} \SnRule{Binding}(X) \altline
\Alt{} \Kw{generate} \Sn{GenBound} \SnRule{Binding}(X) \altline
\Alt{} \Kw{assert} \SnRule{Binding}(X) \altline
\Alt{} \Kw{iterate} \Opt{\Sn{Name}} \altline
\Alt{} \Kw{break} \Opt{\Sn{Name}} \altline
\Alt{} \Kw{return} \Opt{\Sn{Collection}} \altline
\Alt{} \Kw{yield} \SnRule{Binding}(X) \altline
\Alt{} \Kw{except} \SnRule{Binding}(X) \altline
\Alt{} \Kw{goto} \Sn{Id} \altline
\Alt{} \Kw{never}

\Sn{GenBound}
: \Sn{Nothing}
\Alt{} \Kw{to} \Sn{CommaItem} \Kw{of}

\Sn{Cases}
: \SnRule{Binding}(\Sn{Collection})

\SnRule{AlwaysPart}(X)
: \Kw{always} \SnRule{Binding}(X)
\Alt{} \Sn{Nothing}

\Sn{Collection}
: \Sn{Infixed}
\Alt{} \Sn{Infixed} \Sn{Iterators}

\Sn{Iterators}
: \Sn{Iterators1}

\Sn{Iterators1}
: \Sn{Iterator}
\Alt{} \Sn{Iterators1} Iterator

\Sn{Iterator}
: \Kw{for} \Sn{ForLhs} \Kw{in} \Sn{Infixed} \Opt{\Sn{SuchthatPart}}
\Alt{} \Kw{while} \Sn{Infixed}

\Sn{ForLhs}
: \Sn{Infixed} 
\Alt{} \Kw{free} \Sn{Infixed}
\Alt{} \Kw{local} \Sn{Infixed}
\Alt{} \Kw{fluid} \Sn{Infixed}

\Sn{SuchthatPart}
: ``\verb"|"'' \Sn{Infixed}

\subsubsection{Infixed expressions}

\Sn{Infixed}
: \Sn{InfixedExpr}
\Alt{} \Sn{InfixedExpr} \Sn{DeclPart}
\Alt{} \Sn{Block}

\Sn{InfixedExpr}
: \SnRule{E11}(\Sn{Op})
\Alt{} \Sn{E3}


\Sn{E3}
: \Sn{E4}
\Alt{} \Sn{E3} \Kw{and} \Sn{E4}
\Alt{} \Sn{E3} \Kw{or} \Sn{E4}
\Alt{} \Sn{E3} \Sn{LatticeOp} \Sn{E4}

\Sn{E4}
: \Sn{E5}
\Alt{} \Sn{E4} \Kw{has} \Sn{E5}
\Alt{} \Sn{E4} \Sn{RelationOp} \Sn{E5}
\Alt{} \Sn{RelationOp} \Sn{E5}

\Sn{E5}
: \Sn{E6}
\Alt{} \Sn{E5} \Sn{SegOp}
\Alt{} \Sn{E5} \Sn{SegOp} \Sn{E6}

\Sn{E6}
: \Sn{E7}
\Alt{} \Sn{E6} \Sn{PlusOp} \Sn{E7}
\Alt{} \Sn{PlusOp} \Sn{E7}

\Sn{E7}
: \Sn{E8}
\Alt{} \Sn{E7} \Sn{QuotientOp} \Sn{E8}

\Sn{E8}
: \Sn{E9}
\Alt{} \Sn{E8} \Sn{TimesOp} \Sn{E9}

\Sn{E9}
: \SnRule{E11}(\Sn{E12})
\Alt{} \SnRule{E11}(\Sn{E12}) \Sn{PowerOp} \Sn{E9}

\SnRule{E11}(X): \altline
X 
\Alt{} \SnRule{E11}(X) ``\verb"::"'' \Sn{E12}
\Alt{} \SnRule{E11}(X) ``\verb"@"'' \Sn{E12}
\Alt{} \SnRule{E11}(X) \Kw{pretend} \Sn{E12}

\Sn{Type}
: \SnRule{E11}(\Sn{E12})

\Sn{E12}
: \Sn{E13}
\Alt{} \Sn{E13} \Sn{ArrowOp} \Sn{E12}

\Sn{E13}
: \Sn{E14}
\Alt{} \Sn{E14} ``\verb"$"'' \Sn{QualTail}

\Sn{QualTail}
: \Sn{LeftJuxtaposed}
\Alt{} \Sn{LeftJuxtaposed} ``\verb"$"'' \Sn{QualTail}

\Sn{OpQualTail}
: \Sn{Molecule}
\Alt{} \Sn{Molecule} ``\verb"$"'' \Sn{OpQualTail}

\Sn{E14}: \altline
\Sn{E15} 
\Alt{} \Sn{E14} \Kw{except} \Sn{E15} \altline
\Alt{} \Opt{\Sn{E14}} \Kw{with} \Sn{DeclMolecule}
\Alt{} \Opt{\Sn{E14}} \Kw{add} \Sn{DeclMolecule}

\Sn{E15}
: \Sn{Application}

\subsubsection{Infixed operators}

\Sn{Op}: \altline
\Sn{ArrowOp}
\Alt{} \Sn{LatticeOp}
\Alt{} \Sn{RelationOp}
\Alt{} \Sn{SegOp} \altline
\Alt{} \Sn{PlusOp}
\Alt{} \Sn{QuotientOp}
\Alt{} \Sn{TimesOp}
\Alt{} \Sn{PowerOp}

\Sn{NakedOp}: \altline
\Sn{ArrowTok}
\Alt{} \Sn{LatticeTok}
\Alt{} \Sn{RelationTok}
\Alt{} \Sn{SegTok} \altline
\Alt{} \Sn{PlusTok}
\Alt{} \Sn{QuotientTok}
\Alt{} \Sn{TimesTok}
\Alt{} \Sn{PowerTok}

\Sn{ArrowOp}:	 \SnRule{QualOp}(\Sn{ArrowTok}) \\
\Sn{LatticeOp}:	 \SnRule{QualOp}(\Sn{LatticeTok}) \\
\Sn{RelationOp}: \SnRule{QualOp}(\Sn{RelationTok}) \\
\Sn{SegOp}: 	 \SnRule{QualOp}(\Sn{SegTok}) \\
\Sn{PlusOp}: 	 \SnRule{QualOp}(\Sn{PlusTok}) \\
\Sn{QuotientOp}: \SnRule{QualOp}(\Sn{QuotientTok}) \\
\Sn{TimesOp}: 	 \SnRule{QualOp}(\Sn{TimesTok}) \\
\Sn{PowerOp}: 	 \SnRule{QualOp}(\Sn{PowerTok})

\Sn{ArrowTok}:	  ``\verb"->"''  \Alt{} ``\verb"<-"'' 

\Sn{LatticeTok}:  ``\verb"\/"''  \Alt{} ``\verb"/\"'' 

\Sn{RelationTok}: \altline
``\verb"="''   \Alt{} ``\verb"~="''\Alt{} ``\verb"^="'' \Alt{} %\altline
``\verb">="''  \Alt{} ``\verb">"'' \Alt{} ``\verb">>"'' \Alt{} %\altline
``\verb"<="''  \Alt{} ``\verb"<"'' \Alt{} ``\verb"<<"'' \Alt{} \altline
\Kw{is}  \Alt{} \Kw{isnt} \Alt{} \Kw{case} 

\Sn{SegTok}:	 ``\verb".."''  \Alt{} \Kw{by} 

\Sn{PlusTok}:	 ``\verb"+"''   \Alt{} ``\verb"-"''    \Alt{} ``\verb"+-"'' 

\Sn{QuotientTok}: \Kw{mod} \Alt{} \Kw{quo}  \Alt{} \Kw{rem} 

\Sn{TimesTok}:	 ``\verb"*"''   \Alt{} ``\verb"/"''    \Alt{} ``\verb"\"'' 

\Sn{PowerTok}:	 ``\verb"**"''  \Alt{} ``\verb"^"'' 

\subsubsection{Juxtaposed expressions}

The expressions \ttin{a b}, \ttin{a.b}, and \ttin{a(b)} have the
same semantics, but different grouping.  The following rules provide
desired precedence.  Juxtaposition \ttin{a b} is looser than
\ttin{.}  and \ttin{a(b)}, and associates the opposite way:
\begin{verbatim}
 A B C D                   as (.(.(.)))
 f(a).2(b)(c).x.y.(d).(e)  as (((.).).)
\end{verbatim}
This allows both the nested function application \ttin{sin cos x} and
the data access \ttin{T.x.first.tag} to be written naturally.

\Sn{Application}
: \Sn{RightJuxtaposed}

\Sn{RightJuxtaposed}
: \SnRule{Jright}(\Sn{Molecule})

\Sn{LeftJuxtaposed}
: \SnRule{Jleft}(\Sn{Molecule})

\SnRule{Jright}(H)
: \SnRule{Jleft}(H)
\Alt{} \SnRule{Jleft}(H) \SnRule{Jright}(\Sn{Atom})
\Alt{} \Kw{not} \SnRule{Jright}(\Sn{Atom})

\SnRule{Jleft}(H)
: \altline
H
\Alt{} \Kw{not} \Sn{BlockEnclosure}
\Alt{} \SnRule{Jleft}(H) \Sn{BlockEnclosure} \altline
\Alt{} \SnRule{Jleft}(H) ``\verb"."'' \Sn{BlockMolecule}

\subsubsection{Primary expressions}

\Sn{Molecule}
: \Sn{Atom}
\Alt{} \Sn{Enclosure}

\Sn{Enclosure}
: \Sn{Parened}
\Alt{} \Sn{Bracketed}
\Alt{} \Sn{QuotedIds}

\Sn{DeclMolecule}
: \Opt{\Sn{Application}}
\Alt{} \Sn{Block}

\Sn{BlockMolecule}
: \Sn{Atom}
\Alt{} \Sn{Enclosure}
\Alt{} \Sn{Block}

\Sn{BlockEnclosure}
: \Sn{Enclosure}
\Alt{} \Sn{Block}

\Sn{Block}
: \SnRule{Piled}(\Sn{Expression})
\Alt{} \SnRule{Curly}(\Sn{Labeled})

\Sn{Parened}
: ``\verb"{"'' ``\verb"}"'' 
\Alt{} ``\verb"{"'' \Sn{Expression} ``\verb"}"''

\Sn{Bracketed}
: ``\verb"["'' ``\verb"]"'' 
\Alt{} ``\verb"["'' \Sn{Expression} ``\verb"]"'' 

\Sn{QuotedIds}
: ``\verb"'"'' ``\verb"'"''
\Alt{} ``\verb"'"'' \Sn{Names} ``\verb"'"''

\Sn{Names}
: \SnRule{enlist1}(\Sn{Name}, ``\verb","'')

\subsubsection{Terminals}

\Sn{Atom}
: \Sn{Id}
\Alt{} \Sn{Literal}

\Sn{Name}
: \Sn{Id}
\Alt{} \Sn{NakedOp}

\Sn{Id}
: TK\_Id
\Alt{} TK\_Blank
\Alt{} ``\verb"#"''
\Alt{} ``\verb"~"''

\Sn{Literal}
: TK\_Int
\Alt{} TK\_Float
\Alt{} TK\_String

\subsubsection{Meta-rules}

\Sn{Nothing}
:

\SnRule{QualOp}(op)
: op
\Alt{} op ``\verb"$"'' \Sn{OpQualTail}

\Opt{E}
: \Sn{Nothing} 
\Alt{} E

\subsubsection{Documentation}

\SnRule{Doc}(E)
: \Sn{PreDocument} E \Sn{PostDocument}

\Sn{PreDocument}
: \Sn{PreDocumentList}

\Sn{PostDocument}
: \Sn{PostDocumentList}

\Sn{PreDocumentList}
: \Sn{Nothing}
\Alt{} TK\_PreDoc \Sn{PreDocumentList}

\Sn{PostDocumentList}
: \Sn{Nothing}
\Alt{} TK\_PostDoc \Sn{PostDocumentList}

\subsubsection{Separated lists}

The rule \SnRule{enlist1} provides lists with separators between elements,
\eg{}  ``\verb"x , y, z"''.

\SnRule{enlist1}(E, Sep)
: E
\Alt{} \SnRule{enlist1}(E, Sep) Sep E

The rule \SnRule{enlist1a} provides lists within which separators may
be repeated, \eg{}  ``\verb"x ; ; y ; z"''.
Any number of separators may follow the last element.

\SnRule{enlist1a}(E, Sep)
: E
\Alt{} \SnRule{enlist1a}(E, Sep) Sep E
\Alt{} \SnRule{enlist1a}(E, Sep) Sep

\subsubsection{Blocks}

\SnRule{Piled}(E)
: KW\_SetTab \SnRule{PileContents}(E) KW\_BackTab

\SnRule{Curly}(E)
: ``\verb"{"'' \SnRule{CurlyContents}(E) ``\verb"}"''

\SnRule{PileContents}(E): \altline
\SnRule{Doc}(E) \altline
\Alt{} \SnRule{PileContents}(E) KW\_BackSet \SnRule{Doc}(E) \altline
\Alt{} error KW\_BackSet \SnRule{Doc}(E)

\SnRule{CurlyContents}(E)
: \SnRule{CurlyContentsList}(E)

\SnRule{CurlyContentsList}(E)
: \SnRule{CurlyContent1}(E)
\altline\Alt{} \SnRule{CurlyContent1}(E) \SnRule{CurlyContentB}(E)

\SnRule{CurlyContent1}(E)
: \Sn{Nothing}
\Alt{} \SnRule{CurlyContent1}(E) \SnRule{CurlyContentA}(E)

\SnRule{CurlyContentA}(E): \altline
\SnRule{CurlyContentB}(E) ``\verb";"'' \Sn{PostDocument} \altline
\Alt{} error ``\verb";"'' \Sn{PostDocument}

\SnRule{CurlyContentB}(E)
: \Sn{PreDocument} E \Sn{PostDocument}

