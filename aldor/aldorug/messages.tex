% *********************************************************************
\head{chapter}{Compiler messages}{asugMessages}
% *********************************************************************
%
\vspace{1mm}
This chapter lists the messages produced by the
\index{error messages!list of}
\asharp{} compiler.
Each message has a {\em name} and an associated {\em text}.
For example, one message has the name ``{\tt ALDOR\_E\_TinOpMeans}''
and the associated text ``{\tt Operator has \%d possible types.}''

\vspace{1mm}
The name of this message has four parts:

\vspace{1mm}
%\begin{tabbing}
%{\tt OpMeans}~~\=\kill
%{\tt ALDOR}      \> indicates this message is for the \asharp{} compiler, \\
%{\tt E}       \> indicates this is an {\em error\/} message, \\
%{\tt Tin}     \> indicates the message arises in type inference, \\
%{\tt OpMeans} \> identifies the particular message.
%\end{tabbing}
\begin{tabular}{@{}ll@{}}   % @{} kills default whitespace
{\tt ALDOR}      & indicates this message is for the \asharp{} compiler, \\
{\tt E}       & indicates this is an {\em error\/} message, \\
{\tt Tin}     & indicates the message arises in type inference, \\
{\tt OpMeans} & identifies the particular message.
\end{tabular}

\vspace{1mm}
A number of letters may occur in the place of ``{\tt E}'' above.
The letters and their meanings are:
%\begin{tabbing}
%{\tt M}~~\=\kill
%{\tt F}\> fatal errors --- compiler stops \\
%{\tt E}\> soft errors  --- compiler keeps going \\
%{\tt W}\> warnings \\
%{\tt R}\> remarks \\
%\ \\
%{\tt H}\> help \\
%{\tt M}\> interactive loop messages \\
%{\tt N}\> notes --- cross references to other source lines \\
%{\tt D}\> details for another message\\
%{\tt P}\> punctuation \\
%{\tt X}\> continuation, when a message is given in parts \\
%\end{tabbing}

\vspace{1mm}
\begin{tabular}{@{}ll@{}}
{\tt F}& fatal errors --- compiler stops \\
{\tt E}& soft errors  --- compiler keeps going \\
{\tt W}& warnings \\
{\tt R}& remarks \\
\quad \\
{\tt H}& help \\
{\tt M}& interactive loop messages \\
{\tt N}& notes --- cross references to other source lines \\
{\tt D}& details for another message\\
{\tt P}& punctuation 
\end{tabular}

\vspace{1mm}
The name  can be used to enable or disable the display of a specific
warning or a remark.  
See \spadref{asugErrorsSelect} for details.

%\pagebreak
We show below the name and the text of most of the compiler messages.
The only messages we have omitted are those which give the command
line help:
\begin{center}
\begin{tabular}{@{}ll}
{\tt ALDOR\_H\_HelpCmd}            & {\tt ALDOR\_H\_HelpFileTypes} \\
{\tt ALDOR\_H\_HelpOptionSummary}  & {\tt ALDOR\_H\_HelpHelpOpt}   \\
{\tt ALDOR\_H\_HelpArgOpt}         & {\tt ALDOR\_H\_HelpDirOpt}	   \\
{\tt ALDOR\_H\_HelpFileOpt}        & {\tt ALDOR\_H\_HelpGoOpt}	   \\
{\tt ALDOR\_H\_HelpOptimOpt}       & {\tt ALDOR\_H\_HelpDebugOpt}  \\
{\tt ALDOR\_H\_HelpCOpt}       	   & {\tt ALDOR\_H\_HelpLispOpt}   \\
{\tt ALDOR\_H\_HelpMsgOpt}         & {\tt ALDOR\_H\_HelpDevOpt}	   \\
{\tt ALDOR\_H\_HelpMenuPointer}
\end{tabular}
\end{center}
This information is given in \chapref{asugOptions}.

Within the text of the messages,
items such as ``{\tt \%s}'' and ``{\tt \%d}'' have strings and
numbers substituted into them before the message is actually used.
For example, the message with the text
``{\tt Operator has \%d possible types.}''
might appear as
``{\tt Operator has 4 possible types.}''
when it is actually displayed in a given context.


\hbadness=10001 \vbadness=10001
\def\errmsg#1#2{\begin{tabular}{@{}p{48mm}p{68mm}@{}}\raggedright\small\tt
#1&\raggedright{}\small#2\end{tabular}\hfill\newline}

\errmsg{ALDOR\_E\_ExplicitMsg}{\%s}
\errmsg{ALDOR\_N\_ExplicitMsg}{\%s}
\errmsg{ALDOR\_N\_Here}{}
\errmsg{ALDOR\_F\_CmdBadOption}{Improper use of `\%s' option.  Type `\%s -help' for help.}
\errmsg{ALDOR\_F\_CmdNoOption}{`\%s' is not an option.  Type `\%s -help' for help.}
\errmsg{ALDOR\_F\_CmdCantUseEntry}{Could not use file `\%s' as the main file.}
\errmsg{ALDOR\_F\_CmdNoOutputDir}{The specified output directory does not exist.}
\errmsg{ALDOR\_W\_CmdFunnyEntry}{Bogus main file name `\%s' ignored.}
\errmsg{ALDOR\_W\_DisableNotKeyword}{Unable to enable or disable `\%s': not a keyword.}
\errmsg{ALDOR\_W\_FtnVarStringRet}{You cannot receive variable-length strings from Fortran functions. Use FixedString() instead.}
\errmsg{ALDOR\_W\_FtnNotFtnArg}{Unrecognised argument type for Fortran call.}
\errmsg{ALDOR\_F\_MsgTooManyErrors}{Too many errors (use `-M emax=n' or `-M no-emax' to change the limit).}
\errmsg{ALDOR\_R\_MsgCountMessages}{This file had \%d errors, \%d warnings and \%d remarks.}
\errmsg{ALDOR\_R\_MsgAdviseDetails}{Use `aldor -M$<$n$>$' to get more or less detail.}
\errmsg{ALDOR\_R\_MsgCondolences}{Sorry, your file did not compile.}
\errmsg{ALDOR\_R\_MsgCongratulations}{Congratulations, your file compiled!!}
\errmsg{ALDOR\_P\_MsgTagRemark}{(Remark) }
\errmsg{ALDOR\_P\_MsgTagWarning}{(Warning) }
\errmsg{ALDOR\_P\_MsgTagError}{(Error) }
\errmsg{ALDOR\_P\_MsgTagFatal}{(Fatal Error) }
\errmsg{ALDOR\_P\_MsgTagNote}{(Note \%d) }
\errmsg{ALDOR\_P\_MsgPreview}{(Message Preview)}
\errmsg{ALDOR\_P\_MsgAfterMacEx}{(After Macro Expansion) }
\errmsg{ALDOR\_P\_MsgExpandedExpr}{Expanded expression was: }
\errmsg{ALDOR\_P\_MsgSposFileLine}{"\%s", line \%d: }
\errmsg{ALDOR\_P\_MsgSposLineChar}{[L\%d C\%d] }
\errmsg{ALDOR\_P\_MsgNote}{Note \%d}
\errmsg{ALDOR\_P\_MsgSeeNote}{(see \%s)}
\errmsg{ALDOR\_P\_MsgCfNote}{(cf. L\%d C\%d)}
\errmsg{ALDOR\_P\_MsgCfFarNote}{(cf. "\%s" L\%d C\%d)}
\errmsg{ALDOR\_P\_MsgConjunction}{\hspace*{0.075in} and }
\errmsg{ALDOR\_E\_InclBadUnassert}{Unassert: property unknown: `\%s'.}
\errmsg{ALDOR\_E\_InclIfEof}{End of file encountered in `\#if'.}
\errmsg{ALDOR\_E\_InclInfinite}{Circular include files: \%s.}
\errmsg{ALDOR\_E\_InclUnbalElse}{Unbalanced `\#else'.}
\errmsg{ALDOR\_E\_InclUnbalElseif}{Unbalanced `\#elseif'.}
\errmsg{ALDOR\_E\_InclUnbalEndif}{Unbalanced `\#endif'.}
\errmsg{ALDOR\_E\_SysCmdBad}{Improper use of `\%s' system command.}
\errmsg{ALDOR\_W\_SysCmdUnknown}{Unknown system command.}
\errmsg{ALDOR\_E\_ScanBadAftRad}{improper number (after radix specification).}
\errmsg{ALDOR\_E\_ScanBadChar}{bad character in input.}
\errmsg{ALDOR\_E\_ScanBadExpon}{improper number (no digits in exponent).}
\errmsg{ALDOR\_E\_ScanBadRadix}{bad radix specification (2 $<$= radix $<$= 36).}
\errmsg{ALDOR\_E\_ScanNoDigits}{improper number (no digits).}
\errmsg{ALDOR\_E\_ScanOpenString}{string not closed.}
\errmsg{ALDOR\_E\_NormMacDecl}{Macros must not be given return types.}
\errmsg{ALDOR\_E\_NormMacBadBody}{Improper body in `macro' statement.}
\errmsg{ALDOR\_W\_NormNoId}{Couldn't find identifier for documentation}
\errmsg{ALDOR\_W\_NormFornForeign}{Foreign(Foreign) is deprecated: use Foreign instead.}
\errmsg{ALDOR\_W\_NormFornBuiltin}{Foreign(Builtin) is deprecated: use Builtin instead.}
\errmsg{ALDOR\_W\_NormNullForeign}{Foreign() is deprecated: use Foreign instead.}
\errmsg{ALDOR\_E\_MacBadDefn}{Improper form for macro definition.}
\errmsg{ALDOR\_E\_MacBadParam}{Improper macro parameter (should be an identifier).}
\errmsg{ALDOR\_E\_MacBadParamDecl}{Macro parameters must not have type declarations.}
\errmsg{ALDOR\_E\_MacBadArgc}{Macro used with incorrect number of arguments.}
\errmsg{ALDOR\_E\_MacBadArg}{Macro cannot match the given argument.}
\errmsg{ALDOR\_E\_MacInfinite}{Circular macro expansion: \%s.}
\errmsg{ALDOR\_W\_MacHides}{Definition of macro `\%s' hides an outer definition.}
\errmsg{ALDOR\_W\_MacRedefined}{Macro `\%s' redefined in the same scope.}
\errmsg{ALDOR\_F\_SyntaxOverflow}{Parser stack overflow (in state \%d).}
\errmsg{ALDOR\_E\_SyntaxError}{Syntax error.}
\errmsg{ALDOR\_E\_SyntaxErrorDebug}{Syntax error (in state \%d).}
\errmsg{ALDOR\_E\_SyntaxErrorHuh}{\%s.}
\errmsg{ALDOR\_E\_SyntaxNoRecovery}{Cannot recover from earlier syntax errors.}
\errmsg{ALDOR\_E\_SyntaxFullError}{Syntax error: \%s}
\errmsg{ALDOR\_E\_LinUnbalanced}{Unbalanced `\%s' -- missing `\%s'.}
\errmsg{ALDOR\_W\_LinUnbalanced}{Unbalanced `\%s' -- missing `\%s'.}
\errmsg{ALDOR\_F\_LoadNotAbSyn}{This is not an abstract syntax operator.}
\errmsg{ALDOR\_F\_LoadNotFoam}{This is not a Foam operator.}
\errmsg{ALDOR\_F\_LoadNotList}{Expecting a (parenthesized) list here.}
\errmsg{ALDOR\_F\_LoadNotUnary}{Expecting exactly one argument.}
\errmsg{ALDOR\_F\_LoadNotString}{Expecting a "quoted" string here.}
\errmsg{ALDOR\_F\_LoadNotSymbol}{Expecting a symbol here.}
\errmsg{ALDOR\_F\_LoadNotInteger}{Expecting an integer.}
\errmsg{ALDOR\_F\_LoadNotFloat}{Expecting a float.}
\errmsg{ALDOR\_W\_CantUnKeywordRef}{Unable to consider this `ref' as a non-keyword (compiler limitation: try a simpler definition).}
\errmsg{ALDOR\_E\_ChkBadAssign}{Incorrect left-hand side of an assignment.}
\errmsg{ALDOR\_E\_ChkBadDeclare}{Improper form appearing within a declaration.}
\errmsg{ALDOR\_E\_ChkBadDefine}{Incorrect left-hand side of a definition. Check indentation of succeeding definitions, if any.}
\errmsg{ALDOR\_W\_ChkBadDependent}{Bad dependent type detected (or compiler bug)}
\errmsg{ALDOR\_E\_ChkBadFor}{Expecting an identifier or single declaration after `for'.}
\errmsg{ALDOR\_E\_ChkBadForm}{Improper form appearing in `\%s' statement.}
\errmsg{ALDOR\_E\_ChkBadGoto}{A goto must have a label's identifier as its target.}
\errmsg{ALDOR\_E\_ChkBadLabel}{A label must consist of a single identifier.}
\errmsg{ALDOR\_E\_ChkBadMLambda}{Improper macro expansion.}
\errmsg{ALDOR\_E\_ChkBadMacro}{Improper macro definition.}
\errmsg{ALDOR\_E\_ChkBadParams}{Expecting a comma separated list of parameters.}
\errmsg{ALDOR\_E\_ChkBadParamsDups}{Improper duplicate use of parameter name.}
\errmsg{ALDOR\_E\_ChkBadQualification}{Improper LHS in \$-qualification.}
\errmsg{ALDOR\_E\_ChkBadRecordOrUnion}{Duplicate selector/type pair within Record, RawRecord or Union.}
\errmsg{ALDOR\_E\_ChkMissingRetType}{Function return type must be specified.}
\errmsg{ALDOR\_D\_ChkUseFromHint}{\hspace*{0.15in}  Maybe you want to use `import from ...'.}
\errmsg{ALDOR\_E\_ChkSelectSeq}{`select $<$E$>$ in' must be followed by a sequence.}
\errmsg{ALDOR\_E\_ChkSelectExits}{Unexpected `=$>$' in select}
\errmsg{ALDOR\_W\_FunnyJuxta}{Suspicious juxtaposition.  Check for missing `;'.\newline
Check indentation if you are using `\#pile'.}
\errmsg{ALDOR\_W\_FunnyColon}{Suspicious `{\tt :}'.  Do you mean `local' or `default'?}
\errmsg{ALDOR\_W\_FunnyEquals}{Suspicious `{\tt =}'.  Do you mean `{\tt ==}' or `{\tt :=}' ?}
\errmsg{ALDOR\_W\_FunnyEscape}{Escape character ignored.  Do you mean '\_\_'?}
\errmsg{ALDOR\_W\_OldSyntaxAlways}{Deprecated syntax: use `finally' instead of `always'}
\errmsg{ALDOR\_W\_OldSyntaxCatch}{Deprecated syntax: use `catch' instead of `but'}
\errmsg{ALDOR\_W\_OldSyntaxThrow}{Deprecated syntax: use `throw' instead of `except'}
\errmsg{ALDOR\_W\_OldSyntaxUnknown}{Deprecated syntax}
\errmsg{ALDOR\_E\_ScoAssAndDef}{Cannot both assign and define `\%s' in the same scope.\newline
Choose `{\tt ==}', `{\tt :=}', or use as a `for' variable.}
\errmsg{ALDOR\_E\_ScoAssAndRef}{Cannot both define and reference `\%s' in the same scope.\newline
Declare it as a variable instead.}
\errmsg{ALDOR\_E\_ScoAssTypeId}{`\%s' is used in a type, so must be constant, and so cannot be assigned to.}
\errmsg{ALDOR\_E\_ScoBadLexConst}{A local constant may not have the same name as an outer variable or parameter.}
\errmsg{ALDOR\_E\_ScoFluidShadow}{A fluid variable cannot shadow an outer non-fluid binding.}
\errmsg{ALDOR\_E\_ScoBadLoopAss}{Cannot explicitly assign a `for' variable.}
\errmsg{ALDOR\_E\_ScoBadParameter}{Improper form appearing in a parameter context.}
\errmsg{ALDOR\_E\_ScoBadTypeFree}{Free variable `\%s' is bound elsewhere with a different type.}
\errmsg{ALDOR\_E\_ScoDupDefine}{Constant `\%s' cannot be redefined.}
\errmsg{ALDOR\_E\_ScoFreeAndLoc}{Cannot declare `\%s' both free and local.}
\errmsg{ALDOR\_E\_ScoFreeConst}{A constant declared free in an inner scope (`\%s') cannot be defined in that scope.}
\errmsg{ALDOR\_E\_ScoLateFreeLocal}{It is illegal to declare an identifier free or local once it has already been used, defined or assigned.}
\errmsg{ALDOR\_E\_ScoLibrary}{Cannot assign to or redefine library or archive constant `\%s'.}
\errmsg{ALDOR\_E\_ScoNoFree}{A built-in or foreign function cannot be declared `free' or `local'.}
\errmsg{ALDOR\_E\_ScoNoParm}{A built-in or foreign function cannot have the same signature as a parameter.}
\errmsg{ALDOR\_E\_ScoNoSet}{A built-in or foreign function cannot be assigned to or defined.}
\errmsg{ALDOR\_E\_ScoNotBuiltin}{Unknown built-in.}
\errmsg{ALDOR\_E\_ScoParmLocFree}{Parameters cannot be declared local or free.}
\errmsg{ALDOR\_E\_ScoSameSig}{A built-in function cannot have the same signature as a foreign function.}
\errmsg{ALDOR\_E\_ScoParmType}{Parameter type (for \%s) must be specified explicitly or with default.}
\errmsg{ALDOR\_E\_ScoVarOverload}{Variables cannot have different types in the same scope.}
\errmsg{ALDOR\_E\_ScoUnknownFree}{Cannot find scope in which free variable `\%s' is bound.}
\errmsg{ALDOR\_W\_ScoNotProtocol}{Unknown foreign interface protocol.}
\errmsg{ALDOR\_W\_ScoBadLocal}{Implicit local `\%s' is a parameter, local or explicit free in an outer scope. Add a `local' declaration if this is what you intended.}
\errmsg{ALDOR\_W\_ScoFunnyUse}{Identifier `\%s' has different declarations in the same scope. Are all implicit and explicit declarations compatible?}
\errmsg{ALDOR\_W\_ScoBadUse}{Local `\%s' is used without being assigned or defined.}
\errmsg{ALDOR\_W\_ScoLocalNoUse}{Local `\%s' is not assigned, defined, or used.}
\errmsg{ALDOR\_W\_ScoVarDefault}{`\%s' has a default type and a different explicit type declaration.}
\errmsg{ALDOR\_R\_ScoMeaning}{Introducing \%s meaning for \%s with type \%s.}
\errmsg{ALDOR\_E\_ScoEarlyUse}{Implementation restriction: you cannot use a non-lazy constant outside an `add' before it has been defined. Perhaps you ought to define `\%s' sooner.}
\errmsg{ALDOR\_E\_StabDupLabels}{Cannot use label `\%s' more than once in a given scope.}
\errmsg{ALDOR\_R\_StabImporting}{Importing \%s.}
\errmsg{ALDOR\_R\_StabImportingQual}{The import was restricted to: \%s.}
\errmsg{ALDOR\_W\_StabNotImporting}{Ignoring explicit import from \%s.}
\errmsg{ALDOR\_F\_LibOutOfDate}{The file `\%s.ao' is newer than `\%s.ao'.}
\errmsg{ALDOR\_F\_LibBadVersion}{Library format (obsolete version) in file `\%s'.\newline
Current library format version \%d.\%d.\newline
Found library format version \%d.\%d.}
\errmsg{ALDOR\_F\_LibExportNotFound}{Looking for `\%s' with code `\%d'.  Export not found.}
\errmsg{ALDOR\_E\_LibBadMagic}{Library format (bad magic number) in file `\%s'.}
\errmsg{ALDOR\_E\_LibBadNumSect}{Library format (bad number of sections) in file `\%s'.}
\errmsg{ALDOR\_E\_LibBadSectHdr}{Library format (bad section header) in file `\%s'.}
\errmsg{ALDOR\_E\_LibBadSectName}{Library format (bad section name) in file `\%s'.}
\errmsg{ALDOR\_E\_LibSectDup}{Library format (duplicate section) in file `\%s'.}
\errmsg{ALDOR\_E\_LibSectLimit}{Library format (too many sections) in file `\%s'.}
\errmsg{ALDOR\_E\_LibSectOffset}{Library format (offset out of range) in file `\%s'.}
\errmsg{ALDOR\_W\_LibRedefined}{Redefinition of library symbol `\%s'.}
\errmsg{ALDOR\_E\_TinNoMeaningForId}{No meaning for identifier `\%s'.}
\errmsg{ALDOR\_E\_TinNoMeaningForLit}{No meaning for \%s-style literal `\%s'.}
\errmsg{ALDOR\_E\_TinBadDeclare}{Improper form appearing within a declaration.}
\errmsg{ALDOR\_E\_TinIfMeans}{The `if' expression has \%d possible types.}
\errmsg{ALDOR\_E\_TinAssMeans}{Assignment has \%d meanings.}
\errmsg{ALDOR\_E\_TinDefnMeans}{Definition has \%d meanings.}
\errmsg{ALDOR\_E\_TinCantSplitRHS}{This right hand side cannot be split for multiple assignment.}
\errmsg{ALDOR\_E\_TinAssignCreatesDepType}{The type of this variable includes a variable, '\%s'.\newline
Consider using '{\tt ==}' instead of '{\tt :=}'}
\errmsg{ALDOR\_E\_TinExprMeans}{Have determined \%d possible types for the expression.}
\errmsg{ALDOR\_E\_TinNMeanings}{There are \%d meanings for `\%s' in this context.}
\errmsg{ALDOR\_E\_TinBadGoto}{A goto must have a label's identifier as its target.}
\errmsg{ALDOR\_E\_TinOpMeans}{Operator has \%d possible types.}
\errmsg{ALDOR\_E\_TinWildExit}{The `=$>$' is not inside a sequence.}
\errmsg{ALDOR\_E\_TinWildReturn}{The `return' is not inside a function.}
\errmsg{ALDOR\_E\_TinWildYield}{The `yield' is not inside a `generate'.}
\errmsg{ALDOR\_E\_TinContext}{A value is needed but \%s does not produce one.}
\errmsg{ALDOR\_E\_TinContextAssert}{A value is needed but `assert' does not produce one.}
\errmsg{ALDOR\_E\_TinContextDo}{A value is needed but `do' does not produce one.}
\errmsg{ALDOR\_E\_TinContextExit}{A value is needed but `=$>$' does not produce one.}
\errmsg{ALDOR\_E\_TinContextIf}{A value is needed but `if' expression has no `else'.}
\errmsg{ALDOR\_E\_TinContextRepeat}{A value is needed but `repeat' does not produce one.}
\errmsg{ALDOR\_E\_TinContextSeq}{A value is needed but an empty sequence does not produce one.}
\errmsg{ALDOR\_E\_TinCantInferLhs}{The type of the assignment cannot be inferred.}
\errmsg{ALDOR\_E\_TinNoGoodOp}{There are no suitable meanings for the operator `\%s'.}
\errmsg{ALDOR\_E\_TinNoGoodInterp}{There is no suitable interpretation for the expression \%s}
\errmsg{ALDOR\_E\_TinFirstExitType}{The possible type for this \%s is \%s.}
\errmsg{ALDOR\_E\_TinFirstExitTypes}{The possible types for this \%s are:}
\errmsg{ALDOR\_E\_TinTypeConstIntro}{The interpretation of the type expression}
\errmsg{ALDOR\_X\_TinTypeConstFailed}{failed to satisfy the condition that}
\errmsg{ALDOR\_E\_TinCantBeAnalyzed}{Cannot determine the meaning of this expression because\newline
the type of one of its subexpressions cannot yet be completely analyzed.}
\errmsg{ALDOR\_E\_TinEmbeddedSet}{Implicit set within a multi-assign is not yet implemented.}
\errmsg{ALDOR\_E\_TinMultiTry}{try expressions must yield a single value (will be fixed later).}
\errmsg{ALDOR\_E\_TinPackedNotSat}{Raw record type does not satisfy \%s}
\errmsg{ALDOR\_D\_TinNoGoodInterp}{There is no suitable interpretation for the expression \%s}
\errmsg{ALDOR\_D\_TinNoMeaningForId}{No meaning for identifier `\%s'.}
\errmsg{ALDOR\_D\_TinSubexprMeans}{Subexpression `\%s':}
\errmsg{ALDOR\_D\_TinPossTypesLhs}{The possible types of the left hand side are:}
\errmsg{ALDOR\_D\_TinPossTypes}{The possible types were:}
\errmsg{ALDOR\_D\_TinPossInterps}{The possible interpretations of `\%s' are:}
\errmsg{ALDOR\_D\_TinPossTypesRhs}{The possible types of the right hand side (`\%s') are:}
\errmsg{ALDOR\_D\_TinAlternativeMeanings}{The following could be suitable if imported:}
\errmsg{ALDOR\_D\_TinOneMeaning}{Meaning \%d: \%s}
\errmsg{ALDOR\_D\_TinContextType}{The context requires an expression of type \%s.}
\errmsg{ALDOR\_D\_TinMissingExports}{The domain is missing some exports.}
\errmsg{ALDOR\_D\_TinMissingExport}{Missing \%s: \%s}
\errmsg{ALDOR\_D\_TinRejectedTypes}{These possible types were rejected:}
\errmsg{ALDOR\_X\_TinNoArgumentMatch}{rejected because argument \%d did not match `\%s'.}
\errmsg{ALDOR\_X\_TinParameterMissing}{rejected because parameter \%d (\%s) is missing.}
\errmsg{ALDOR\_X\_TinBadArgumentNumber}{rejected because it cannot take \%d arguments.}
\errmsg{ALDOR\_X\_TinBadFnType}{rejected because the context requires type `\%s'.}
\errmsg{ALDOR\_D\_TinRejectedType}{The rejected type is \%s.}
\errmsg{ALDOR\_D\_TinRejectedTypesForRhs}{These possible types for the right hand side were rejected:}
\errmsg{ALDOR\_D\_TinRejectedTypeForRhs}{The rejected type for the right hand side is \%s.}
\errmsg{ALDOR\_D\_TinShouldUseDoubleEq}{You should use \%s{\tt ==}\%s and not \%s.}
\errmsg{ALDOR\_D\_TinAvailableTypesForArg}{The available types for argument \%d were:}
\errmsg{ALDOR\_D\_TinArgNoMatchParTypes}{Argument \%d of `\%s' did not match any possible parameter type.}
\errmsg{ALDOR\_D\_TinOperatorNoMatch}{Operator (argument \%d of apply) did not match any possible parameter type.}
\errmsg{ALDOR\_D\_TinMoreMeanings}{There are \%d meanings for the operator `\%s'.}
\errmsg{ALDOR\_D\_TinRetTypesCantContext}{No one possible return type satisfies the context type.}
\errmsg{ALDOR\_D\_TinExpectedType}{Expected type \%s.}
\errmsg{ALDOR\_D\_TinExpectedTypes}{Expected one of:}
\errmsg{ALDOR\_D\_TinRejectedRetTypes}{These possible return types were rejected:}
\errmsg{ALDOR\_D\_TinOtherDiffArgNum}{There are other meanings rejected due to different number of arguments.}
\errmsg{ALDOR\_D\_TinPossSelectorTypes}{Possible types for the selector `\%s' were:}
\errmsg{ALDOR\_D\_TinPossRetTypeSetBang}{Possible return types for `set!' expression were:}
\errmsg{ALDOR\_D\_TinPossTypesForSetBang}{No `set!' found for any of the possible types for `\%s':}
\errmsg{ALDOR\_D\_TinSetBangBadArgNum}{There is no `set!' definition with this number of arguments.}
\errmsg{ALDOR\_D\_TinOneImpMeaning}{\%s: \%s from \%s}
\errmsg{ALDOR\_D\_TinOneLexMeaning}{\%s: \%s, a local}
\errmsg{ALDOR\_D\_TinOneLibMeaning}{\%s: \%s, a library}
\errmsg{ALDOR\_D\_TinOneMeaning0}{(...): \%s}
\errmsg{ALDOR\_D\_TinFirstExitCant}{This is not compatible with the types of the other \%ss.}
\errmsg{ALDOR\_N\_TinOtherExitType}{Here the \%s type is \%s.}
\errmsg{ALDOR\_N\_TinOtherExitTypes}{Here the \%s types are:}
\errmsg{ALDOR\_W\_TinNoValReturn}{The `return' gives a value but none is expected.}
\errmsg{ALDOR\_E\_TinReturnNoVal}{The `return' gives no value where one is expected.}
\errmsg{ALDOR\_R\_TinInferring}{Inferring \%s: \%s.}
\errmsg{ALDOR\_W\_TqNotBuiltin}{`\%s: \%s' is not exported by Builtin.}
\errmsg{ALDOR\_E\_GenImpNoRep}{Domains with implicit exports must define Rep (as a constant not a macro)}
\errmsg{ALDOR\_W\_GenDomFunNotConst}{Function returns a domain that might not be constant (which may cause problems if it is used in a dependent type).}
\errmsg{ALDOR\_W\_GenCatFunNotConst}{Function returns a category that might not be constant (which may cause problems if it is used in a dependent type).}
\errmsg{ALDOR\_W\_GenBadDefCycle}{Illegal recursive definition: \%s}
\errmsg{ALDOR\_W\_GenBadDefOrder}{Implementation restriction: the value of `\%s' depends on the value of \%s. Perhaps you ought to define `\%s' later?}
\errmsg{ALDOR\_M\_FintBreakHandler}{Execution terminated: use \#quit if you want to quit.}
\errmsg{ALDOR\_M\_FintBreakHandler0}{Use '\#quit' to quit Aldor.}
\errmsg{ALDOR\_M\_FintYesOrNo}{\ \newline
Please, answer y or n.}
\errmsg{ALDOR\_M\_FintRedefined}{\%s redefined.}
\errmsg{ALDOR\_M\_FintOptionState}{\%s is \%s.}
\errmsg{ALDOR\_M\_FintOptionValue}{\%s is \%d.}
\errmsg{ALDOR\_M\_FintUnknownOpt}{Unrecognized option. Type:\newline
\#int \%s\newline
for help.}
\errmsg{ALDOR\_M\_FintTimings}{\hspace*{3.225in}                                           Comp: \%d msec, Interp: \%d msec}
\errmsg{ALDOR\_M\_FintGbcStart}{Garbage collection...}
\errmsg{ALDOR\_M\_FintGbcEnd}{done.}
\errmsg{ALDOR\_M\_FintIntOptionsNoFile}{Cannot give files, such as `\%s', with `\#int options'}
\errmsg{ALDOR\_M\_ShellSyntax}{The correct syntax is: \#int \%s "$<$shell-command$>$"\newline
}
\errmsg{ALDOR\_M\_CdSyntax}{The correct syntax is: \#int \%s $<$directory$>$\newline
}
\errmsg{ALDOR\_M\_InvalidDir}{Invalid directory.\newline
}
\errmsg{ALDOR\_M\_FintOptions}{\ \newline
Available options:\newline
\ \newline
\#int \%s [on|off]  print the value of an evaluated expression.\newline
\#int \%s [on|off]  try to wrap an assignment around the current line.\newline
\#int \%s [on|off]  ask for confirmation before redefining something.\newline
\#int \%s [on|off]  display timings after every input.\newline
\#int \%s [num]  set the limit size of some messages; 0 for no-limit.\newline
\#int \%s ...  reset command line options.\newline
\#int \%s     perform garbage collection.\newline
\#int \%s "$<$command$>$"  execute a shell command.\newline
\#int \%s $<$directory$>$  change current directory.\newline
\#int \%s [0|1|2]  display backtrace when an exception occurs.\newline
\hspace*{0.375in}     0: never, 1: only when not caught, 2: always.\newline
\#int \%s     display this message.\newline
\ \newline
\#quit       quit the interactive loop.}
\errmsg{ALDOR\_F\_CdFailed}{Could not change working directory to `\%s'.}
\errmsg{ALDOR\_F\_CcFailed}{C compile failed.  Command was: \%s}
\errmsg{ALDOR\_F\_LinkFailed}{Linker failed.  Command was: \%s}
\errmsg{ALDOR\_F\_BadFType}{Cannot handle file `\%s' of type `\%s'.  Try using file type `\%s'.}
\errmsg{ALDOR\_F\_WdClobberIn}{Output would clobber input file `\%s'.}
\errmsg{ALDOR\_F\_WdClobberFile}{Output would clobber the source file `\%s'.}
\errmsg{ALDOR\_W\_WillObsolete}{The file `\%s' will now be out of date.}
\errmsg{ALDOR\_W\_RemovingFile}{Removing file `\%s'.}
\errmsg{ALDOR\_W\_NotCreatingFile}{Cannot create file `\%s' from input file.}
\errmsg{ALDOR\_W\_NoFiles}{No files!  Type `\%s -help' for help.}
\errmsg{ALDOR\_F\_NoConfig}{Could not find aldor.conf}
\errmsg{ALDOR\_W\_CfgError}{\%s}
\errmsg{ALDOR\_F\_NoFNameProperty}{Fortran naming scheme field (\%s) is not specified in aldor.conf}
\errmsg{ALDOR\_F\_BadFNameValue}{Unrecognised Fortran naming scheme (\%s) specified in aldor.conf}
\errmsg{ALDOR\_F\_NoFCmplxProperty}{Fortran complex functions field (\%s) is not specified in aldor.conf}
\errmsg{ALDOR\_F\_BadFCmplxValue}{Unrecognised Fortran complex functions field value (\%s) specified in aldor.conf}
\errmsg{ALDOR\_M\_BreakEnter}{Aldor compiler break ------------------------------------------------------}
\errmsg{ALDOR\_M\_BreakExit}{------------------------------------------------------------------------------}
\errmsg{ALDOR\_M\_BreakNoMsg}{No message.}
\errmsg{ALDOR\_M\_BreakNoCmd}{Unrecognized command: `\%s'.}
\errmsg{ALDOR\_M\_BreakMsgPrompt}{::: }
\errmsg{ALDOR\_M\_BreakMsgHelpAvail}{Help is available.}
\errmsg{ALDOR\_M\_BreakMsgBadNode}{Bad node.}
\errmsg{ALDOR\_M\_BreakMsgNoNode}{No node.}
\errmsg{ALDOR\_M\_BreakMsgNoStab}{No symbol table.}
\errmsg{ALDOR\_M\_BreakMsgNoTypeInfo}{No type info yet.}
\errmsg{ALDOR\_M\_BreakMsgAtTop}{At top.}
\errmsg{ALDOR\_M\_BreakMsgAtLeaf}{At leaf.}
\errmsg{ALDOR\_M\_BreakMsgNoPrev}{No prev.}
\errmsg{ALDOR\_M\_BreakMsgNoNext}{No next.}
\errmsg{ALDOR\_M\_BreakMsgCantSelect}{No such selection.}
\errmsg{ALDOR\_M\_BreakMsgNTypes}{The expression has \%d possible types.}
\errmsg{ALDOR\_M\_BreakMsg1Type}{The expression has the unique type: }
\errmsg{ALDOR\_M\_BreakMsgUsedContext}{Used in `\%s' context.}
\errmsg{ALDOR\_M\_BreakMsgNotId}{Can only ask for meanings of an identifier.}
\errmsg{ALDOR\_M\_BreakHelp}{Commands are:\newline
\hspace*{0.15in}  help  -- give this help\newline
\hspace*{0.15in}  getcomsg   -- get information on the current message \newline
\hspace*{0.15in}  notes   -- show the notes associated with the current message \newline
\hspace*{0.15in}  mselect i  -- select message i to be the current message \newline
\hspace*{0.15in}  mnext  -- select the next message in the list \newline
\hspace*{0.15in}  mprev  -- select the previous message in the list \newline
\hspace*{0.15in}  msg   -- display the error message again\newline
\hspace*{0.15in}  nice  -- show with pretty printed form\newline
\hspace*{0.15in}  ugly  -- show with more detailed, internal form\newline
\ \newline
\hspace*{0.15in}  show  -- show the current node\newline
\hspace*{0.15in}  means -- show the possible meanings of the current node\newline
\hspace*{0.15in}  use   -- show how the current node is used\newline
\hspace*{0.15in}  seman -- show the extra semantic information for the current node\newline
\hspace*{0.15in}  scope -- show information about the current scope\newline
\ \newline
\hspace*{0.15in}  up    -- use the parent as the current node\newline
\hspace*{0.15in}  down  -- use 0th child as the current node\newline
\hspace*{0.15in}  next  -- use the next sibling as the current node\newline
\hspace*{0.15in}  prev  -- use the previous sibling as the current node\newline
\hspace*{0.15in}  home  -- return to the original node\newline
\hspace*{0.15in}  where  -- returns the line and column location of the current node \newline
\ \newline
\hspace*{0.15in}  quit  -- exit the compiler, showing all messages so far}
\errmsg{ALDOR\_H\_HelpConfigOpt}{Configuration options:\newline
\hspace*{0.15in}  -N file=$<$file$>$       Specify name of config file.\newline
\hspace*{0.15in}  -N sys=$<$name$>$        Specify system name.\newline
}
\errmsg{ALDOR\_H\_HelpCppOpt}{C++ generation options:  Control the behaviour of `-Fc++'.\newline
\hspace*{0.15in}  -P basicfile=$<$bf$>$       if the filename $<$bf$>$ (absolute filename) is provided,\newline
\hspace*{1.95in}                          the standard basic types correspondence between C++ and Aldor\newline
\hspace*{1.95in}                          will be overridden.\newline
\hspace*{1.95in}                          If this option is not provided, the compiler uses 'basic.typ'\newline
\hspace*{1.95in}                          located in \$ALDORROOT/include.\newline
\ \newline
\hspace*{0.15in}  -P discrim-return       Will discriminate functions on the return type by changing\newline
\hspace*{1.95in}                          the name of the function to 'fnname\_return-type'.\newline
\ \newline
\hspace*{0.15in}  -P no-discrim-return    Won't discriminate functions on the return type.\newline
\hspace*{1.95in}                          Names of the functions will be as the original, however\newline
\hspace*{1.95in}                          the code generated for overloaded functions on the return\newline
\hspace*{1.95in}                          type only won't compile.\newline
\hspace*{1.95in}                          This option is the default.\newline
}
\errmsg{ALDOR\_H\_HelpProductInfo}{Contact infodesk@nag.co.uk for product support and information.\newline
Use the ALDORbug program for reporting any bugs.\newline
}
\errmsg{ALDOR\_E\_SigAbrt}{Program fault (abort process).}
\errmsg{ALDOR\_E\_SigBus}{Program fault (bus error).}
\errmsg{ALDOR\_E\_SigEmt}{Program fault (emulator instruction).}
\errmsg{ALDOR\_E\_SigFpe}{Program fault (arithmetic exception).}
\errmsg{ALDOR\_E\_SigHup}{User break (hangup).}
\errmsg{ALDOR\_E\_SigIll}{Program fault (illegal instruction).}
\errmsg{ALDOR\_E\_SigInt}{User break (interrupt).}
\errmsg{ALDOR\_E\_SigPipe}{Program fault (write on a pipe with no one to read it).}
\errmsg{ALDOR\_E\_SigDanger}{Program fault (paging space low)}
\errmsg{ALDOR\_E\_SigQuit}{User break (quit).}
\errmsg{ALDOR\_E\_SigSegv}{Program fault (segmentation violation).}
\errmsg{ALDOR\_E\_SigSys}{Program fault (bad argument to system call).}
\errmsg{ALDOR\_E\_SigTerm}{User break (software termination signal).}
\errmsg{ALDOR\_E\_SigTrap}{Program fault (trace trap).}
\errmsg{ALDOR\_E\_SigXcpu}{Exceeded time limit imposed by operating system.}
\errmsg{ALDOR\_E\_SigXfsz}{Exceeded file size limit imposed by operating system.}
\errmsg{ALDOR\_E\_SigUnknown}{Unexpected signal (\%d).}
\errmsg{ALDOR\_F\_SxAlreadyShare}{Share label \#nn= previously defined.}
\errmsg{ALDOR\_F\_SxBadArgumentTo}{Inappropriate argument to function `\%s'.}
\errmsg{ALDOR\_F\_SxBadChar}{Illegal character 0x\%x.}
\errmsg{ALDOR\_F\_SxBadCharName}{Improper character name after \#\\.}
\errmsg{ALDOR\_F\_SxBadComplexNum}{Improper complex number \#C....}
\errmsg{ALDOR\_F\_SxBadFeatureForm}{Improper feature form following \#+ or \#-.}
\errmsg{ALDOR\_F\_SxBadPotNum}{Meaningless potential number `\%s'.}
\errmsg{ALDOR\_F\_SxBadPunct}{Misplaced `\%s'.}
\errmsg{ALDOR\_F\_SxBadToken}{Missing escape in token.}
\errmsg{ALDOR\_F\_SxBadUninterned}{Package given with `\#:'}
\errmsg{ALDOR\_F\_SxCantMacroArg}{Macro \#\%c does not take a numeric argument.}
\errmsg{ALDOR\_F\_SxCantShare}{Share label \#nn= not previously defined.}
\errmsg{ALDOR\_F\_SxInternNeeds}{Intern requires a string.}
\errmsg{ALDOR\_F\_SxMacroIlleg}{Illegal macro character `\#\%c'.}
\errmsg{ALDOR\_F\_SxMacroUndef}{Undefined macro character `\#\%c'.}
\errmsg{ALDOR\_F\_SxMacroUnimp}{Unimplemented macro character `\#\%c'.}
\errmsg{ALDOR\_F\_SxMustMacroArg}{Macro \#n\%c requires a numeric argument.}
\errmsg{ALDOR\_F\_SxNReverseNeeds}{NReverse requires the last cdr of a list to be nil.}
\errmsg{ALDOR\_F\_SxNumDenNeeds}{\%s requires an integer or ratio.}
\errmsg{ALDOR\_F\_SxPackageExists}{A package with the name \%s already exists.}
\errmsg{ALDOR\_F\_SxReadEOF}{End of file during read.}
\errmsg{ALDOR\_F\_SxTooManyElts}{Number of elements greater than given size.}
\errmsg{ALDOR\_F\_StoCantBuild}{Storage allocation error (can't build internal structure).}
\errmsg{ALDOR\_F\_StoOutOfMemory}{Storage allocation error (out of memory).}
\errmsg{ALDOR\_F\_StoUsedNonalloc}{Storage allocation error (using non-allocated space).}
\errmsg{ALDOR\_F\_StoFreeBad}{Storage allocation error (atempt to free unknown space).}
\errmsg{ALDOR\_F\_CantOpen}{Could not open file `\%s'.}
\errmsg{ALDOR\_F\_CantOpenMode}{Could not open file `\%s' with mode `\%s'.}
\errmsg{ALDOR\_F\_CantFindTemp}{Could not find unused temporary file names.}
\errmsg{ALDOR\_W\_CantUseObject}{Could not use object file `\%s'.}
\errmsg{ALDOR\_W\_CantUseLibrary}{Could not use library file `\%s'.}
\errmsg{ALDOR\_W\_CantUseArchive}{Could not use archive file `\%s'.}
\errmsg{ALDOR\_W\_OverRideLibraryFile}{Current file over-rides existing library in\newline
\hspace*{0.075in} `\%s'.}
\errmsg{ALDOR\_F\_Bug}{Compiler bug: \%s.}
\errmsg{ALDOR\_W\_Bug}{Internal compiler warning: \%s}
\errmsg{ALDOR\_F\_BugExportSymeNotInit}{Compiler bug: I am trying to create a slot for the export `\%s:\%s'. However, gen0SymeSetInit() has not been used to initialise it so I can not continue (sorry).}
\errmsg{ALDOR\_I\_PreRelease}{This is a pre-release of \%s. `aldor -h info' for more details.}
\errmsg{ALDOR\_I\_DemoExpiry}{This is a demo version of \%s. `aldor -h info' for information.\newline
This program should not be used after \%s}
\errmsg{ALDOR\_S\_Syme\_Label}{label}
\errmsg{ALDOR\_S\_Syme\_Param}{parameter"    }
\errmsg{ALDOR\_S\_Syme\_LexVar}{lexical variable}
\errmsg{ALDOR\_S\_Syme\_LexConst}{lexical constant}
\errmsg{ALDOR\_S\_Syme\_Import}{import}
\errmsg{ALDOR\_S\_Syme\_Export}{export}
\errmsg{ALDOR\_S\_Syme\_Extend}{extend}
\errmsg{ALDOR\_S\_Syme\_Library}{library}
\errmsg{ALDOR\_S\_Syme\_Archive}{archive}
\errmsg{ALDOR\_S\_Syme\_Builtin}{builtin}
\errmsg{ALDOR\_S\_Syme\_Foreign}{foreign}
\errmsg{ALDOR\_S\_Syme\_Fluid}{fluid variable}
\errmsg{ALDOR\_S\_Syme\_Trigger}{trigger}
\errmsg{ALDOR\_S\_Syme\_Temp}{temporary}

