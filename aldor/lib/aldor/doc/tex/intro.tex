\section{Introduction}

\subsection*{What is \salli?}
\salli (the standard aldor library) is a compact, general-purpose
library for \aldor programs.
It provides a low-level interface between \aldor programmers and the
abstract machine, allowing its users to start programming with a minimal
set of basic types and data structures already built into the language.

\subsection*{How do I get and install \salli?}
\libaldor{} is now bundled with the
\htmladdnormallinkfoot{Aldor compiler}{http://www.aldor.org}
and is automatically installed with it. Older
versions of \salli{} for previous versions of the compiler can be obtained
from {\tt salli@sophia.inria.fr} or by
\htmladdnormallinkfoot{ftp}{ftp://ftp-sop/cafe/software/libaldor}.

\subsection*{How do I use \salli in my programs?}
To use \salli, just add the line {\tt \#include "aldor"}
in your \aldor sources (if you were previously an axllib user,
do {\em not} include {\tt "axllib"} anymore).
When building an executable, add the option
\begin{center}
{\tt -laldor}
\end{center}
to your compiler command line.
Check the subsection on~\alalias{\this}{sec:gmp}{using GMP}
for the options required if you want to use the GMP package,
and the subsection on~\alalias{\this}{sec:debug}{debugging}
for the options required if you want to use the debug version
of \libaldor.

If you are running \salli inside the compiler interactive loop, then
type the line
\begin{center}
{\tt \#include "aldorinterp"}
\end{center}
immediately after {\tt \#include "aldor"},
which will import various things for interactive use and make the interpreter
loop print values automatically.
Note that \altype{GMPInteger} and
\altype{GMPFloat} are not available in the interactive loop.
As with any \aldor program, do not forget
the {\tt -q} option in order to optimize your programs, specially
if performance is an issue.

Please report any installation problem or bugs you encounter
to {\tt salli@sophia.inria.fr}.
