\documentclass{article}

\usepackage{fancyheadings, epsfig, makeidx, html}

\newcommand{\astype}[1]{\altype{#1}}
\newcommand{\astarget}[1]{\altarget{#1}}
\newcommand{\aspage}[1]{\alpage{#1}}
\newcommand{\asfunc}[2]{\alfunc{#1}{#2}}
\newcommand{\asalias}[3]{\alalias{#1}{#2}{#3}}

\newcommand{\alhtmltarget}[2]{\label{#1:#2}{} \index{#2!#1}}
\newcommand{\altypes}[1]{\section*{#1}}
\newcommand{\albuiltin}[1]{\texttt{#1}}%
\newcommand{\alexttype}[2]{\htmlref{\texttt{#2}}{#1}}
\newcommand{\altype}[1]{\htmlref{\texttt{#1}}{#1}}
\newcommand{\alfunc}[2]{\alalias{#1}{#2}{#2}}
\newcommand{\alalias}[3]{\htmlref{\texttt{#3}}{#1:#2}}
\newcommand{\History}[3]{}
\newcommand{\category}[1]{\multicolumn{3}{l}{#1}}

\newenvironment{exports}{%
    \pagebreak[2]
    \par\bigskip\noindent
    \textbf{Exports}\nopagebreak\par
    \begin{list}{}{}
    \item{} \begin{tabular}{lll}
  }%
  {\end{tabular}\end{list}\smallskip}

\newenvironment{exports0}[1][]{%
    \pagebreak[2]
    \par\medskip\begin{list}{}{}
    \item{} #1 \vspace{-1ex}\nopagebreak
    \item{} \begin{tabular}{lll}
  }%
  {\end{tabular}\end{list}\smallskip}

\newenvironment{aswhere}{%
    \medskip
    \begin{list}{}{}
      \item{} where\nopagebreak
      \item \begin{tabular}{lcl}%
  }%
  {\end{tabular}\end{list}\smallskip}

\newcommand{\thistype}[1]{%
  \newpage\clearpage
  \subsection{#1}%
  \index{#1}%
  \label{#1}{}%
}

\newcommand{\thistypE}[2]{%
  \newpage\clearpage
  \subsection{#2}%
  \index{#1|see{#2}}
  \index{#2}%
  \label{#2}{}%
}

\newcommand{\alpage}[1]{%
    \newpage\clearpage
    \subsubsection{#1}%
  }

\newenvironment{aldocpar}[1]{%
    \pagebreak[2]\par\bigskip\noindent
    \textbf{#1}\nopagebreak\par
      \begin{list}{}{}%
        \item{}
  }%
  {\end{list}\smallskip}

\newcommand{\Aldocpar}[2]{%
    \pagebreak[2]\par\bigskip\noindent
    \textbf{#1}\nopagebreak\par
      \begin{quotation}%
  {#2}
  \end{quotation}\smallskip
}

\newenvironment{usage}{\begin{aldocpar}{Usage}}{\end{aldocpar}}
\newenvironment{descr}{\begin{aldocpar}{Description}}{\end{aldocpar}}
\newenvironment{retval}{\begin{aldocpar}{Returns}}{\end{aldocpar}}
\newenvironment{remarks}{\begin{aldocpar}{Remarks}}{\end{aldocpar}}
\newenvironment{errors}{\begin{aldocpar}{Errors}}{\end{aldocpar}}
\newenvironment{asex}{\begin{aldocpar}{Example}}{\end{aldocpar}}

\newcommand{\Usage}[1]{\Aldocpar{Usage}{#1}}
\newcommand{\Descr}[1]{\Aldocpar{Description}{#1}}
\newcommand{\Retval}[1]{\Aldocpar{Returns}{#1}}
\newcommand{\Remarks}[1]{\Aldocpar{Remarks}{#1}}
\newcommand{\Errors}[1]{\Aldocpar{Errors}{#1}}
\newcommand{\Asex}[1]{\Aldocpar{Example}{#1}}

\newcommand{\nsignature}[3]{\alsignature{#1}{{#2} $\to$ {#3}}}

\newcommand{\alsignature}[2]{%
    \pagebreak[2]\par\bigskip\noindent
    \textbf{Signature}\nopagebreak\par
    \begin{list}{}{}
      \item{}
        \begin{tabular}{ll} 
          {#1}: & {#2} \\
        \end{tabular}
    \end{list}%
}%

\newenvironment{signatures}{%
    \pagebreak[2]\par\bigskip\noindent
    \textbf{Signatures}\nopagebreak\par
    \begin{list}{}{}
      \item{}
        \begin{tabular}{ll}%
  }%
  {\end{tabular}\end{list}\smallskip}
\newcommand{\Signatures}[1]{%
    \pagebreak[2]\par\bigskip\noindent
    \textbf{Signatures}\nopagebreak\par
    \begin{list}{}{}
      \item{}
        \begin{tabular}{ll}#1\end{tabular}\end{list}\smallskip
}

\newenvironment{params}{%
    \par\bigskip\noindent
    \pagebreak[2]
    \begin{tabular}{lll}
      \textbf{Parameter} & \textbf{Type} & \textbf{Description}\\ \hline
  }%
  {\end{tabular}\smallskip}
\newcommand{\Params}[1]{%
    \par\bigskip\noindent
    \pagebreak[2]
    \begin{tabular}{lll}
      \textbf{Parameter} & \textbf{Type} & \textbf{Description}\\ \hline
#1\end{tabular}\smallskip
}

\newenvironment{asoutput}{%
    \pagebreak[2]
    \begin{tabbing} \hspace{10ex}\= \kill%
  }%
  {\end{tabbing}
}

\newcommand{\seealso}[1]{%
    \pagebreak[2]\par\bigskip\noindent
    \textbf{See Also}\nopagebreak\par
    \begin{list}{}{}
      \item{} #1
    \end{list}\smallskip%
}%

\pagestyle{fancyplain}

\makeindex
\usepackage{amssymb,xspace,graphics,url,html}


\begin{document}

\thispagestyle{empty}
\begin{center}
\vskip 5cm
\hrule
\vskip 0.5cm
{\LARGE\bf Salli User's Guide and Reference}\\
\vskip 1cm
{\Large Manuel Bronstein}\\
\vskip 1cm
%{\Large Version 0.1.12f -- \today}\\
{\Large Development $\beta$ Version -- \today}
\vskip 0.5cm
\hrule
\end{center}
\newpage

\tableofcontents

\newpage

\section{Reference Manual}

\altypes{Arithmetic}
\thistype{AldorInteger}
\History{Manuel Bronstein}{7/10/98}{created}
\Usage{import from AldorInteger}
\Descr{AldorInteger~provides an interface to the software (``infinite'' precision)
integers provided by the aldor virtual machine.}
\begin{exports}
\category{\astype{IntegerType}}\\
\end{exports}
\thistype{AdditiveType}
\History{Manuel Bronstein}{8/11/99}{created}
\Usage{AdditiveType: Category}
\Descr{AdditiveType~is the category of types with addition/substraction operations.}
\begin{exports}
\category{\astype{PrimitiveType}}\\
\alfunc{AdditiveType}{$0$} : & \% & zero\\
\alfunc{AdditiveType}{$+$} : & (\%, \%) $\to$ \% & addition\\
\alfunc{AdditiveType}{$-$} : & \% $\to$ \% & opposite\\
\alfunc{AdditiveType}{$-$} : & (\%, \%) $\to$ \% & substraction\\
\alfunc{AdditiveType}{add!}: & (\%, \%) $\to$ \% & In--place addition\\
\alfunc{AdditiveType}{minus!}: & \% $\to$ \% & In--place opposite\\
\alfunc{AdditiveType}{minus!}: & (\%, \%) $\to$ \% & In--place substraction\\
\alfunc{AdditiveType}{zero?}: & \% $\to$ \astype{Boolean} & test for $0$\\
\end{exports}
\aspage{$0$}
\alhtmltarget{AdditiveType}{$0$}
\Usage{$0$}
\alsignature{$0$}{\%}
\Retval{Return the $0$ constant of the type.}
\aspage{$+,-$}
\alhtmltarget{AdditiveType}{$+,-$}
\alhtmltarget{AdditiveType}{$+$}
\alhtmltarget{AdditiveType}{$-$}
\Usage{$x + y$\\ $x - y$\\ $-x$}
\Signatures{
$-$: & \% $\to$ \%\\
$+,-$: & (\%, \%) $\to$ \%\\
}
\Params{ {\em x,y} & \% & elements of the type\\ }
\Retval{$x + y, x - y$ return respectively
the sum and difference $x$ with $y$, while $-x$ returns
the opposite of $x$.}
\seealso{\alfunc{AdditiveType}{add!}, \alfunc{AdditiveType}{minus!}}
\aspage{add!,minus!}
\alhtmltarget{AdditiveType}{add!,minus!}
\alhtmltarget{AdditiveType}{add!}
\alhtmltarget{AdditiveType}{minus!}
\Usage{add!(x, y)\\ minus!(x, y)\\ minus!~x}
\Signatures{
minus!: & \% $\to$ \%\\
add!, minus!: & (\%, \%) $\to$ \%\\
}
\Params{ {\em x, y} & \% & Elements of the type\\ }
\Retval{add!($x,y$) and minus!($x,y$) returns respectively $x + y$
and $x-y$, while minus!~x returns the opposite of $x$.
In all cases, the storage used by x is allowed
to be destroyed or reused, so x is lost after this call.}
\Remarks{Those functions may cause x to be destroyed, so do not use them
unless x has been locally allocated, and is guaranteed not to share space
with other elements. Some functions are not necessarily copying their
arguments and can thus create memory aliases.}
\seealso{\alfunc{AdditiveType}{$+$},\alfunc{AdditiveType}{$-$}}
\aspage{zero?}
\alhtmltarget{AdditiveType}{zero?}
\Usage{zero?~x}
\nsignature{zero?}{\%}{\astype{Boolean}}
\Params{{\em x} & \% & an element of the type\\ }
\Retval{Returns the result of $x = 0$ using the semantics of $=$ of the type.}
\thistype{ArithmeticType}
\History{Manuel Bronstein}{28/9/98}{created}
\Usage{ArithmeticType: Category}
\Descr{ArithmeticType~is the category of types with standard arithmetic operations.}
\begin{exports}
\category{\astype{AdditiveType}}\\
\alfunc{ArithmeticType}{$1$}: & \% & one\\
\alfunc{ArithmeticType}{$*$}: & (\%, \%) $\to$ \% & product\\
\alalias{ArithmeticType}{$**$}{$\hat{}$}:
& (\%, \astype{MachineInteger}) $\to$ \% & exponentiation\\
\alfunc{ArithmeticType}{one?}: & \% $\to$ \astype{Boolean} & test for $1$\\
\alfunc{ArithmeticType}{times!}: & (\%, \%) $\to$ \% & In--place product\\
\end{exports}
\aspage{$1$}
\alhtmltarget{ArithmeticType}{$1$}
\Usage{$1$}
\alsignature{$1$}{\%}
\Retval{Return the $1$ constant of the type.}
\aspage{$*$,$**$}
\alhtmltarget{ArithmeticType}{$*$,$**$}
\alhtmltarget{ArithmeticType}{$*$}
\alhtmltarget{ArithmeticType}{$**$}
\Usage{$x \ast y$\\ $x$ \^{} $n$}
\Signatures{
$\ast$: & (\%, \%) $\to$ \%\\
\^{}: & (\%, \astype{MachineInteger}) $\to$ \%\\
}
\Params{
{\em x,y} & \% & elements of the type\\
{\em n} & \astype{MachineInteger} & an exponent\\
}
\Retval{$x \ast y$ returns the product of $x$ with $y$, while
$x$ \^{} $n$ returns $x$ to the power $n$.}
\seealso{\alfunc{ArithmeticType}{times!}}
\aspage{times!}
\alhtmltarget{ArithmeticType}{times!}
\Usage{times!(x, y)}
\nsignature{times!}{(\%, \%)}{\%}
\Params{ {\em x, y} & \% & Elements of the type\\ }
\Retval{Return $xy$, where the storage used by x is allowed
to be destroyed or reused, so x is lost after this call.}
\Remarks{This function may cause x to be destroyed, so do not use it unless
x has been locally allocated, and is guaranteed not to share space
with other elements. Some functions are not necessarily copying their
arguments and can thus create memory aliases.}
\seealso{\alfunc{ArithmeticType}{$*$}}
\aspage{one?}
\alhtmltarget{ArithmeticType}{one?}
\Usage{one?~x}
\nsignature{one?}{\%}{\astype{Boolean}}
\Params{{\em x} & \% & an element of the type\\ }
\Retval{Returns the result of $x = 1$ using the semantics of $=$ of the type.}

\end{document}
