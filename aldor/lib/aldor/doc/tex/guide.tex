\section{User Guide}

This guide introduces the common types and categories provided by \libaldor,
and presupposes some familiarity with programming in \aldor.
If you are unfamiliar with \aldor, then after installing \libaldor, go
through the tutorial
{\tt aldorlib/tutorial/tutorial.ps}, which will familiarize you both with
\aldor and \libaldor.
\libaldor{} provides basic categories and types in 3 areas:
arithmetic, data structures and input/output. Figure~\ref{fig:cats} shows the
general category hierarchy in those areas and a few of the types provided,
while Figure~\ref{fig:data} shows the category hierarchy for
data structures.
Note that all the category names end with the word {\tt Type}.

\begin{figure}[htb!]
\begin{latexonly}
\rotatebox{90}{
\scalebox{.75}[.9]{
\includegraphics{sallicat}
}
}
\end{latexonly}
\begin{htmlonly}
\includegraphics{sallicat}
\end{htmlonly}
\caption{The \libaldor{}~general category hierarchy}
\label{fig:cats}
\end{figure}

\subsection{Arithmetic}
\libaldor{} contains several integer types:
\altype{MachineInteger} provides full-word signed machine integers,
while \Integer{}
\footnote{{\tt Integer} is actually a macro that you should
use if you do not care about the specific big integer package that
will be used at runtime. If integer efficiency is an issue, then read
the section on \alalias{\this}{sec:gmp}{using GMP} to learn about the actual
integer types avalaible and how to select one.}
provides ``infinite'' precision software integers.
Both of them are of category \altype{IntegerType},
so the same functions can be applied to them.
In addition, \altype{MachineInteger} exports 3 useful
constants: \alfunc{MachineInteger}{min}
and \alfunc{MachineInteger}{max}, whose values are the smallest and
greatest machine integers,
and \alfunc{MachineInteger}{bytes}, whose values is the
machine wordsize in bytes.
The \alfunc{MachineInteger}{bytes} constant allows you to write
wordsize--dependent code, for example when generating half--word--size prime
numbers. A \altype{MachineInteger} value {\tt n}
can be coerced to an \Integer{} via {\tt n::Integer}, and the reverse
conversion is provided via the \alfunc{IntegerType}{machine} function,
\ie~{\tt n := machine m}. This last conversion can
of course lose precision if {\tt m} does not fit within a machine word.
The general form for iterating over a range of integers is\\
\centerline{ {\tt for} {\em variable} {\tt in} {\em from}{\tt..}{\em to}
{\tt by} {\em step} | {\em condition} {\tt repeat \{ \dots~\}} }
where the {\em to} parameter and the {\tt by} and ``| {\em condition}''
parts are optional.

Besides the integers, \libaldor{} also provides machine floating point numbers,
both in single precision (\altype{SingleFloat}) and double precision
(\altype{DoubleFloat}). Finally, if you are linking with the
GMP library, \libaldor{} provides \altype{GMPFloat}, which is an interface
to the ``infinite'' precision software floats of GMP.
The above 3 type share the category \altype{FloatType}.

\subsection{Data structures}
\libaldor{} provides both linear data structures (e.g.~arrays and lists)
and non-linear ones (e.g.~hash tables). The category hierarchy for
data structures is shown in Figure~\ref{fig:data}.

\begin{figure}[htb!]
\begin{latexonly}
\rotatebox{90}{
\scalebox{.75}[.9]{
\includegraphics{sallidata}
}
}
\end{latexonly}
\begin{htmlonly}
\includegraphics{sallidata}
\end{htmlonly}
\caption{The \libaldor~category hierarchy for data structures}
\label{fig:data}
\end{figure}

\libaldor{} provides some standard linear
data structures having very similar functionalities:
\altype{List}~{\tt T} and \altype{CheckingList}~{\tt T}
both provide linked lists
whose entries are of type {\tt T}, while
\altype{Array}~{\tt T} and \altype{CheckingArray}~{\tt T} both provide
arrays whose entries are of type {\tt T}.
Lower-level linear data structures are \altype{PrimitiveArray}~{\tt T}
and \altype{PackedPrimitiveArray}~{\tt T},
which provide simpler arrays whose entries are of type {\tt T}.
All of those data structures can be created by explicitly listing a
finite number of elements, for example\\
\centerline{{\tt l:\altype{List} \altype{String} := ["hello", "world"] }}
or by bracketing a generator, for example\\
\centerline{ {\tt a:\altype{PrimitiveArray} \altype{MachineInteger} :=
[n for n in 1..100 | odd?~n]}.}
In addition, the function \alfunc{FiniteLinearStructureType}{new}
also allows structures to be created, and the constant
\alfunc{FiniteLinearStructureType}{empty}
returns an empty structure. The individual elements can
be accessed via {\tt s.n} where {\tt s} is a data structure and
{\tt n} an index. The indexing scheme and the bound--checking scheme
both depend on the structure:
\altype{List} and \altype{CheckingList} are 1-indexed, while
\altype{Array}, \altype{CheckingArray}, \altype{PrimitiveArray}
and \altype{PackedPrimitiveArray} are 0-indexed.
If you need to know the indexing scheme at runtime,
or want to write index-independent code, the {\tt firstIndex} constant
returns the index of the first element of a structure.

The general form for iterating efficiently over a
\altype{BoundedFiniteLinearStructureType} is\\
\centerline{ {\tt for} {\em variable} {\tt in} {\em structure}
| {\em condition} {\tt repeat \{ \dots~\}} }
where ``| {\em condition}'' is optional. This is available
for all high-level linear data structures,
but not for those of category \altype{PrimitiveArrayType}.

There are classical tradeoffs between the various array types:
the only difference between \altype{CheckingArray} and \altype{Array}
is that \altype{CheckingArray} checks whether the index is within the range
of the array, while \altype{Array} does not. Similarly,
\altype{CheckingList} checks whether a list is empty before calling
\alfunc{ListType}{first} and \alfunc{ListType}{rest}, while \altype{List}
does not.
Since both pairs of types offer the same exports,
you can use one during the development
and testing phase, and then switch to the other for releasing your code.
An advantage of the lower-level arrays of category~\altype{PrimitiveArrayType}
is that they are compatible with C pointers and
generate more efficient code when accessing their elements.
In order to benefit from the advantages of all those types,
\libaldor{} provides the \alfunc{ArrayType}{data} function,
which returns the data of an \altype{Array} or \altype{CheckingArray}
as a \altype{PrimitiveArray} without copying or allocating memory.
It is thus possible to use \altype{Array} or \altype{CheckingArray}
in your code, making sure to
apply \alfunc{ArrayType}{data} to it before accessing elements in a loop.
For example, the following function efficiently computes the sum
of all the elements with even indices of an array of machine integers:
\begin{ttyout}
evenSum(a:Array MachineInteger):MachineInteger == {
    import from MachineInteger;               -- for the index i
    import from PrimitiveArray MachineInteger;-- for accessing the elements of p
    p := data a;                              -- for efficiency (optimized code)
    sum:MachineInteger := 0;
    for i in 0..#a by 2 repeat sum := sum + p.i;
    sum;
}
\end{ttyout}
\noindent
Conversely, the function \alfunc{ArrayType}{array}
creates an \altype{Array} or \altype{CheckingArray}
from a \altype{PrimitiveArray} without copying.
Note that the
\alfunc{BoundedFiniteDataStructureType}{generator} functions
in \altype{Array} and \altype{CheckingArray} use the \alfunc{ArrayType}{data}
function, so iterating an \altype{Array} or \altype{CheckingArray} {\tt a} via\\
\centerline{{\tt for x in a repeat \{ \dots~\}}}
is as efficient as using a \altype{PrimitiveArray}.
Note that the debug version\footnote{See the subsection
on \alalias{\this}{sec:debug}{debugging} for more information on using the
debug library.} of \libaldor{} performs bound checking on
accesses into all types of arrays, including \altype{String} and
all the low-level array types.

Since \altype{String} has the category
\altype{ArrayType}~\altype{Character},
all the array functionalities
are also applicable to strings. For example,\\
\centerline{{\tt for c in "hello" repeat \{ \dots~\}}}
assigns successively the characters 'h', 'e', 'l', 'l' and 'o' to {\tt c},
and strings can be created from \altype{Generator}~\altype{Character}.
Note that \altype{String} and \altype{PrimitiveArray} \altype{Character}
are not interchangeable, since the former is packed and not the latter.
\altype{String} is however interchangeable with
\altype{PackedPrimitiveArray}~\altype{Character}.
Similarly, chunks of memory viewed as byte arrays are provided by
either one of \altype{PrimitiveMemoryBlock},
\altype{MemoryBlock} or \altype{CheckingMemoryBlock}. Those types
are to be used for buffers rather than \altype{PrimitiveArray}~\altype{Byte},
which is not interchangeable with \altype{PrimitiveMemoryBlock}.

For types whose elements are not word-size, such as \altype{Byte},
\altype{Character}, \altype{SingleFloat} and \altype{DoubleFloat},
you can use \altype{PackedPrimitiveArray} as a packed alternative
to \altype{PrimitiveArray}, which is then compatible with the
corresponding C--arrays (see Table~\ref{tab:compatc}).

The type \altype{Stream}~{\tt T},
of category~\altype{LinearStructureType}~{\tt T},
provides infinite linear structures.
There are several ways to create streams, the easier ones being via an
unbounded iterator, or via a function that computes its $\sth{n}$ element.
For example,
\begin{ttyout}
    import from MachineInteger, Stream MachineInteger;
    sqr1 := [n^2 for n in 0..];
    sqr2 := stream(0, (n:MachineInteger):MachineInteger +-> n^2);
\end{ttyout}
\noindent
are two different ways to create the infinite stream $[0^2, 1^2, 2^2, \dots]$.
Streams can be iterated, yielding loops whose duration cannot be predicted
in advance, so using a parallel finite iterator or a termination condition
inside the loop is advisable. Finally, streams are lazy in that $s.0,\dots,s.n$
are computed only when $s.n$ has been specifically resquested, and elements
are never recomputed a second time.

\libaldor~also provides different table stuctures:
hash tables are provided by the \altype{HashTable} type and are created
via the \alfunc{TableType}{table} function, as in\\
\centerline{{\tt t:HashTable(String, MachineInteger) := table();}}
The hash function defaults to the one provided by the type of the keys
if it has \altype{HashType}, but can be overridden by providing your
own as last argument to the hash table type constructor, as in\\
\centerline{{\tt t:HashTable(SingleFloat, MachineInteger, h) := table();}}
where {\tt h} any function producing machine integers from machine
floats. Providing the hash function is required when the type of
the keys does not have \altype{HashType}. Sorted tables
are provided by the \altype{SortedAssociationSet} type,
those do not use hashing however.
Hash tables can also be used to make a function
\alfunc{HashTable}{remember} its previous
computed values: if $f: A \to B$ is any function, then
\centerline{{\tt g:HashTable(A, B) := remember~f;}}
creates a table {\tt g} that can be used everywhere instead of {\tt g}.
While {\tt f(a)} always recomputes $f(a)$, {\tt g(a)} computes $f(a)$
only once and stores the pair $[a,f(a)]$ in the table.

Finally, \libaldor{} also provides more specialized data structures
documented in the reference section: \altype{Set}, \altype{SortedSet},
\altype{SortedList} and \altype{Queue}.

\subsection{Input/Output}
\libaldor{} provides an I/O model inspired from the stream model of
Java and C++.
Data is written in text format on objects of type
\altype{TextWriter}, in binary format on objects of type
\altype{BinaryWriter}, read in text format from
objects of type \altype{TextReader}
and read in binary format from objects of type \altype{BinaryReader}.
The standard streams \alfunc{TextReader}{stdin},
\alfunc{TextWriter}{stdout} and \alfunc{TextWriter}{stderr}
are constants exported
by \altype{TextReader} and \altype{TextWriter} respectively,
while \alfunc{BinaryReader}{bin}, \alfunc{BinaryWriter}{bout}
and \alfunc{BinaryWriter}{berr} are their binary
counterparts.
In addition, any \altype{File} can be coerced into either
text or binary readers or writers, any \altype{String} can
be coerced into text readers or writers, any \altype{PrimitiveMemoryBlock}
can be coerced into binary readers or writers, and you can create
custom streams via the \alfunc{TextReader}{textReader},
\alfunc{TextWriter}{textWriter},
\alfunc{BinaryReader}{binaryReader} and \alfunc{BinaryWriter}{binaryWriter}
functions.

The function {\tt $<<$} is used for both input and output:
its binary version ``{\em writer} \alfunc{OutputType}{$<<$} {\em data}''
is for output,
and its unary version ``\alfunc{InputType}{$<<$} {\em reader}'' is for input,
in which
case the return type must be specified, either via an assigment to a variable,
\eg~{\tt n:MachineInteger := $<<$ stdin}, or via the {\tt @} construct, or via
the context. Whether you are reading/writing text or serializing data depends
on the reader/writer type (text or binary).
For example, a \libaldor{} version of the ``Hello world'' program is
\begin{ttyout}
#include "aldor"

import from TextWriter, String, Character;  -- Character needed for 'newline'
stdout << "Hello world!" << newline;
\end{ttyout}
\noindent
Text and binary writers can be flushed, either via the
\alfunc{TextWriter}{flush!}
function, or by sending the constant \alfunc{WriterManipulator}{flush},
exported by \altype{WriterManipulator}. Thus, the two lines\\
\centerline{\tt flush!(stdout $<<$ "Hello world" $<<$ newline);}
and\\
\centerline{\tt stdout $<<$ "Hello world" $<<$ newline $<<$ flush;}
are equivalent. The manipulator \alfunc{WriterManipulator}{endnl}
sends first a \alfunc{Character}{newline}
and then flushes the stream, so another alternative for the above is\\
\centerline{\tt stdout $<<$ "Hello world" $<<$ endnl;}

When coercing strings and buffers to readers or writers, you should assign the
resulting stream to a variable if you intend to read or write more than
one value from the stream, since the coercion resets the stream
to the beginning of the string. For example, if {\tt s} is the \altype{String}
``   12   56'', then
\begin{ttyout}
import from MachineInteger;
a:MachineInteger := << s::TextReader;             -- assigns 12 to a
b:MachineInteger := << s::TextReader;             -- assigns again 12 to b
\end{ttyout}
\noindent
while
\begin{ttyout}
import from MachineInteger;
p := s::TextReader;
a:MachineInteger := << p;                         -- assigns 12 to a
b:MachineInteger := << p;                         -- assigns 56 to b
\end{ttyout}
\noindent
A \altype{File} is however not reset when coerced, so if the file ``mydata''
contains ``[1,2,3]  [4,5,6]'', then both structures can be read by
\begin{ttyout}
import from File, String;
f := open("mydata", fileRead);
l:List MachineInteger := << f::TextReader;        -- assigns [1,2,3] to l
v:Array MachineInteger := << f::TextReader;       -- assigns [4,5,6] to v
close! f;
\end{ttyout}
\noindent
When coercing a \altype{String} to a text writer or a
\altype{PrimitiveMemoryBlock} to a binary writer, you must ensure
that the string or buffer is long enough to contain all the data
that will be written to
it, since \libaldor{} does not protect you against writing past the end of the
string or buffer, which is not extended by the write operation. You can
either use \alfunc{FiniteLinearStructureType}{new}
to create a large enough buffer,
or write into an existing string or buffer.
The type \altype{StringBuffer} provides string that grow when you write
into them, so use it when you do not know the amount of characters that
will be written. For example:
\begin{ttyout}
import from Integer, String;
strw:StringBuffer := new();
strw::TextWriter << "20! = " << factorial 20 << null;
str := string strw;
\end{ttyout}
produces the string ``20! = 2432902008176640000''.
You can reuse the same \altype{StringBuffer} in several calls in order
to reuse the corresponding buffer rather than allocating new ones.
When writing to a \altype{String} or \altype{StringBuffer},
you should in general
write \alfunc{Character}{null} (a constant exported by \altype{Character})
after your data in order to terminate the string properly.
Note that the debug version of \libaldor{} performs bound checking on
accesses into \altype{String} and \altype{PrimitiveMemoryBlock},
so it can verify whether you are writing past
the end of a string or buffer. See the subsection on debugging for more
information on using the debug library.
Coercing a \altype{File} to a reader or writer allocates memory, so it
is advisable to assign the resulting stream to a variable. Unlike the ones
for strings or buffers, those coercions do not reset the file to its beginning.

\libaldor{} provides 2 categories for text input/output: \altype{InputType}
is for types whose elements can be read from a \altype{TextReader},
and \altype{OutputType} is for types whose elements can be written
to a \altype{TextWriter}. In addition, the single category
\altype{SerializableType} is for types whose elements can be serialized
in binary format from a \altype{BinaryReader} and to a \altype{BinaryWriter}.
All the arithmetic types provided by \libaldor, as well as \altype{Byte},
\altype{Character}, \altype{String}, \altype{MemoryBlock} and
\altype{CheckingMemoryBlock}
are \altype{InputType}, \altype{OutputType} and \altype{SerializableType},
allowing you to read, write and serialize their elements. The data structures
\altype{List} {\tt T}, \altype{Array} {\tt T}, \altype{PrimitiveArray} {\tt T}
and \altype{HashTable}{\tt(K,V)} inherit whatever input
and output capabilities that their parameters have.

Programs that perform input or output tend to repeatedly import 
the various stream types and accessories (manipulators, characters and
strings). As an alternative to those imports,
you can use {\tt \#include "aldorio"} in
addition to {\tt \#include "aldor"}, which does a global import of
all the following: \altype{Character}, \altype{String}, \altype{File},
\altype{TextReader}, \altype{TextWriter}, \altype{BinaryReader},
\altype{BinaryWriter} and
\altype{WriterManipulator}.
So an alternative ``Hello world'' program would be:
\begin{ttyout}
#include "aldor"                              -- performs no import
#include "aldorio"                            -- imports all the I/O types

stdout << "Hello world!" << endnl;
\end{ttyout}

\subsection{Compatibility with C types}
Several of \libaldor's types are compatible with their C counterparts and
can be passed as arguments to C functions.
Table~\ref{tab:compatc}
lists those types that can safely be exchanged with C functions.
\begin{table}[ht]
\begin{center}
\begin{tabular}{|ll|} \hline
\libaldor & C \\ \hline
\altype{Boolean} & \tt{long} \\
\altype{MachineInteger} & \tt{long} \\
\altype{SingleFloat} & \tt{float} \\
\altype{Character} & \tt{char} \\
\altype{String} & \tt{char*} \\
\altype{PrimitiveMemoryBlock} & \tt{char*} \\
\altype{PackedPrimitiveArray} \altype{Byte} & \tt{char*} \\
\altype{PackedPrimitiveArray} \altype{Character} & \tt{char*} \\
\altype{PackedPrimitiveArray} \altype{SingleFloat} & \tt{float*} \\
\altype{PackedPrimitiveArray} \altype{DoubleFloat} & \tt{double*} \\
\altype{Pointer} & \tt{void*} \\
\altype{PrimitiveArray} & \tt{void**} \\
\altype{PrimitiveArray} \altype{MachineInteger} & \tt{long*} \\
\altype{File} & \tt{FILE*} \\
\altype{GMPFloat} & \tt{mpf\_t}\\
\altype{GMPInteger} & \tt{mpz\_t}\\ \hline
\end{tabular}
\caption{Compatibility between \libaldor{} and C types}
\label{tab:compatc}
\end{center}
\end{table}
Since \aldor does not provide a type that is guaranteed to be compatible 
with the C {\tt int} type on all platforms, in order to use C functions
having {\tt int} in their parameters, you must first write a C wrapper that
communicates only through the type {\tt long}.
Note that even though some type {\tt T} can be compatible
with a C type {\tt TC}, it is not always the case that
\altype{PrimitiveArray} {\tt T} is compatible with {\tt TC*}, for
example \altype{PrimitiveArray} \altype{Character} is not compatible
with {\tt char*} (nor is it compatible with \altype{String} in \libaldor).
It is always the case however that \altype{PackedPrimitiveArray} {\tt T}
is compatible with {\tt TC*}.
Note that \altype{DoubleFloat} is not compatible with
the C type {\tt double}, but with
\builtin{DFlo\$Machine} instead (there are coercions between
\builtin{DFlo\$Machine} and \altype{DoubleFloat}).
On the other hand, \altype{PackedPrimitiveArray} \altype{DoubleFloat}
is compatible with {\tt double*}.
When exchanging objects of type \altype{PrimitiveArray},
\altype{PrimitiveMemoryBlock} or \altype{String}
with C functions, always use the \alfunc{PrimitiveArrayType}{array},
\alfunc{PrimitiveArrayType}{pointer}, \alfunc{String}{string} and
\alfunc{String}{pointer} functions. While those have no effect and
no cost in the release version, they are necessary when linking with
the debug version in order to use the bound--checking features.
When doing this, you must use \altype{Pointer} in the prototypes for
the C functions rather than \altype{PrimitiveArray},
\altype{PrimitiveMemoryBlock} or \altype{String}.
For example, a function that uses the C--function {\tt unlink} to
remove a file could be written in the following way:
\begin{ttyout}
#include "aldor"
remove(filename:String):() == {
	import { unlink: Pointer -> MachineInteger } from Foreign C;
	unlink pointer filename;
}
\end{ttyout}
\noindent
While a function that uses the C--function {\tt mktemp} to get a
temporary file name could be written in the following way:
\begin{ttyout}
#include "aldor"

temporaryName():() == {
	import { mktemp: Pointer -> Pointer } from Foreign C;
	string mktemp pointer("/tmp/XXXXXX");
}
\end{ttyout}
\noindent
Note the use of \alfunc{String}{string} even though the buffer was
allocated within \libaldor{} and not within {\tt mktemp}.

\subsection{Using GMP}
\altarget{sec:gmp}
The type {\tt Integer} described in this document is actually a macro,
which defaults to \altype{AldorInteger}, the software integers provided
by the \aldor{} runtime. For efficiency or other reasons, you may prefer
to use the GMP\footnote{GMP (the GNU Multiple Precision library) is
a free library for software integers, rationals and floats, available
from \url{http://www.swox.com/gmp/}.}
library, which is supported by \libaldor. The easiest way
to use GMP is to compile all your source files with the options
\begin{center}
{\tt -dGMP -cruntime=foam-gmp,gmp}
\end{center}
The option {\tt -dGMP} makes {\tt Integer} point to the type
\altype{GMPInteger}, while the other option, which is necessary
only when linking an executable, ensure that GMP makes use
of the \aldor{} garbage collector. All you need is GMP 3.0 or later
installed in a file called {\tt libgmp.a} to produce executables.
Using GMP generally produces more efficient programs,
but programs calling GMP cannot be interpreted by the \aldor{}
compiler, nor can they run inside its interactive loop.

Using the {\tt -dGMP} option allows you to compile the same sources
either with or without GMP, which can be appreciable, but you must
ensure that you do mix files compiled with and without {\tt -dGMP} since
the macro {\tt Integer} would then be expanded into two different types.

An additional advantage of using GMP, is that \altype{GMPInteger}
exports and uses internally several of the in--place or higher--level
operations of GMP, which are not available with \altype{AldorInteger}.
In addition, variables of type \altype{GMPInteger} are compatible
with the C type {\tt mpz\_t} from GMP, so you can directly call
C programs that use GMP from your \aldor{} code.

\subsection{Profiling}
\libaldor{} provides a couple of tools to help you determine how much time is
spent in various sections of your programs. The simplest way is to use
the TIMESTART and TIME(\dots) macros in your code. Those macros have
absolutely no effect when compiling normally, but if you add the option
{\tt -dTIME} in the compiler command line when making the {\tt .ao}
file, then a CPU stopwatch is started when TIMESTART is encountered,
and read at each TIME(\dots) statement. Those readings are then written
to \alfunc{TextWriter}{stderr} together with whatever message appears inside
the TIME macro. For example, if a computation has 3 major parts, then the
following coding allows you to determine how much time is spent in each
part:
\begin{ttyout}
myComputation(...):... == {
    TIMESTART;                  -- has no effect on normal compilation
    ... part 1 of the computation ...
    TIME("myComputation: part one done at");
    ... part 2 of the computation ...
    TIME("myComputation: part two done at");
    ... part 3 of the computation ...
    TIME("myComputation: part three done at");
    ...                         -- do not forget to return the result here
}
\end{ttyout}
When profiling sections of a multi-file library, simply recompile the
desired {\tt .as} files with the {\tt -dTIME -fo -q3} options, then link
your executable with the local {\tt .o} files, which takes precedence over
the ones in the library.

Finer profiling, including obtaining information on the garbage collection
time, is possible via the use of the type \altype{Timer}, whose
elements are stopwatches that can be started and stopped at will. See the
reference section for more information on the use of timers.

\subsection{Exceptions}
\libaldor{} can throw the following exceptions:
\begin{itemize}
\item
\alfunc{File}{open}
throws the exception \altype{FileException} if the file cannot
be opened in the desired mode.
\item
\alfunc{InputType}{$<<$}
from a \altype{TextReader}
throws the exception \altype{SyntaxException} if the input does
not correspond to an element of the expected type.
\item
Accessing a \altype{CheckingArray}
or \altype{CheckingMemoryBlock}
out of bounds throws the exception \altype{ArrayException}.
\item
Accessing a \altype{CheckingList}
out of bounds throws the exception \altype{ListException}.
\item
Accessing a \altype{TableType} with \alfunc{TableType}{apply}
throws the exception \altype{TableException} if the given key
is not present in the table, unless the table was created with
a default function via either \alfunc{HashTable}{forget} or
\alfunc{HashTable}{remember}.
\item
Calling \alfunc{Generator}{next!} on an empty
generator throws the exception \altype{GeneratorException}.
\end{itemize}
No function in \libaldor{} makes a call to \alfunc{String}{error}.

\subsection{Debugging}
\altarget{sec:debug}
\libaldor{} comes with a debug version, which makes various assertions
about the arguments of the functions called. Its most important debugging
feature is probably the ability to bound check all accesses into
\altype{PrimitiveArray},
\altype{PackedPrimitiveArray},
\altype{PrimitiveMemoryBlock},
\altype{List}
and \altype{String} at runtime.
To use the debug version of the
library, just compile all your files with the
\begin{center}
{\tt -dDEBUG}
\end{center}
option, and add the option
\begin{center}
{\tt -laldord}
\end{center}
rather than {\tt -laldor} when building an executable.
It is also preferable when debugging to add the {\tt -q1} option
in order to prevent inlining.

\libaldor{} also provides a couple of macros to help you debug your
applications:
\begin{description}
\item[{\tt TRACE(message,value)}:] has no effect on normal compilation,
but sends {\em message} followed by {\em value} and a newline to
\alfunc{TextWriter}{stderr} when compiled with the {\tt -dTRACE} option.
\item[{\tt AGAT(stream,value)}:] has no effect on normal compilation,
but sends {\em value} to the Agat stream {\em stream} when compiled with
the {\tt -dAGAT} option. This option is only available if you have
Agat installed (see
\url{http://www-sop.inria.fr/cafe/Olivier.Arsac/agat/agat.html}).
\end{description}

Another useful tracing feature is the {\tt -Wdebug} option of the
compiler, which is now supported by \libaldor: if you compile
any source file, say {\tt foo.as}  with the {\tt -Wdebug} option,
and then call \alfunc{Trace}{traceActivate} in your program,
then any entry and exit of functions from {\tt foo.as} will be
traced, until you call \alfunc{Trace}{traceStop} in your program.
See \alfunc{Trace}{traceActivate} for an example.

